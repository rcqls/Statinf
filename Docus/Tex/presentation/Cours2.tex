
\documentclass[11pt]{beamer}
%Packages
\usepackage{multirow}
\usepackage{graphicx}
\usepackage[utf8x]{inputenc}
\usepackage{aeguill}
\usepackage{amssymb}
\usepackage[french]{babel}
\usepackage{pgf,pgfarrows,pgfnodes}
\usepackage{xmpmulti}
\usepackage{multimedia}
\usepackage{bbm}
\usepackage{bm}
\usepackage{mathrsfs,dsfont}
\usepackage{xcolor}
\usepackage{colortbl}
\usepackage{longtable}
\usepackage[T1]{fontenc}
\usepackage[scaled]{beramono}
\usepackage{float}
\usepackage{xkeyval,calc,listings,tikz}
\usepackage{fancyvrb}

%Preamble

\input Cours/cqlsInclude
\input Cours/testInclude


\mode<article>{\usepackage{fullpage}}
\usefonttheme{structureitalicserif}


\definecolor{VertFonce}{rgb}{0,.4,.0}

\mode<presentation>
{
  %\usetheme{Warsaw}
  % or ...
\usetheme{Boadilla}

  \setbeamercovered{transparent=5}
  % or whatever (possibly just delete it)
}
%\setbeamercovered{dynamic}


\subject{Talks}

\AtBeginSection[]
{
  
\begin{frame}<beamer>
	\frametitle{Plan}
    \tableofcontents[currentsection,currentsubsection]
\end{frame}

}


\definecolor{show}{rgb}{0.59,0.29,0.59}

\setbeamercovered{invisible}
\newcommand{\Sim}{{\star}}
\newcommand{\ok}{ \textcolor{green}{\large$\surd$}}
\newcommand{\nok}{ \textcolor{red}{\large X}}


\usetikzlibrary{arrows,%
  calc,%
  fit,%
  patterns,%
  plotmarks,%
  shapes.geometric,%
  shapes.misc,%
  shapes.symbols,%
  shapes.arrows,%
  shapes.callouts,%
  shapes.multipart,%
  shapes.gates.logic.US,%
  shapes.gates.logic.IEC,%
  er,%
  automata,%
  backgrounds,%
  chains,%
  topaths,%
  trees,%
  petri,%
  mindmap,%
  matrix,%
  calendar,%
  folding,%
  fadings,%
  through,%
  positioning,%
  scopes,%
  decorations.fractals,%
  decorations.shapes,%
  decorations.text,%
  decorations.pathmorphing,%
  decorations.pathreplacing,%
  decorations.footprints,%
  decorations.markings,%
  shadows}
\tikzset{
  every plot/.style={prefix=plots/pgf-},
  shape example/.style={
    color=black!30,
    draw,
    fill=yellow!30,
    line width=.5cm,
    inner xsep=2.5cm,
    inner ysep=0.5cm}
}

%Styles

%Title

\beamertemplateshadingbackground{green!50}{yellow!50}
\title[Problématiques Produits A et B]
{Cours de Statistiques Inférentielles}
\author{CQLS~: cqls@upmf-grenoble.fr}
\date{\today}

\begin{document}
\maketitle


%\begin{frame}
%  \titlepage
%\end{frame}

\section[Estimation]{Estimation~: Obtention et Qualité}

\begin{frame}
\frametitle{Problématique du Salaire Juste}
\begin{exampleblock}{Enoncé}
\begin{itemize}
\item  Une équipe de sociologues propose de réunir un comité d'experts pour la création d'un indicateur, appelé \textbf{Salaire Juste}, mesuré pour toute personne active et qui permettra de transformer les ressources individuelles réelles (souvent mesurées par un salaire) en tenant compte de critères aussi importants que les ressources locales, le partage de ces ressources, la pénibilité du travail, le niveau d'expérience, d'expertise et bien d'autres encore.
\item Cet indicateur est conçu de sorte qu'en théorie il devrait être équivalent (en fait égale à une valeur étalon 100) pour tout personne active dans le monde.
\item Après quelques mois de travail, un premier prototype (très perfectible) du \textbf{Salaire Juste} est élaboré par la fine équipe d'experts.
\end{itemize}
\end{exampleblock}
\end{frame}


\begin{frame}
\frametitle{Problématique du Salaire Juste}

\begin{alertblock}{Critère de pays civilisé}
Les sociologues s'accordent à dire qu'un pays peut se dire non civilisé si~:
\begin{enumerate} 
\item \textbf{Discrimination Mondiale}~: le Salaire Juste moyen dans le pays est très supérieur à  la valeur 100 de base. Un Salaire Juste moyen excédant un seuil de 150 est considéré comme intolérable.
\item \textbf{Discrimination Intérieure}~: les Salaires Justes dans le pays sont très dispersés. La variance des Salaires Justes dans le pays supérieur à 30 est considérée comme excessive et donc anormale.
\end{enumerate}
\end{alertblock}
\end{frame}


\begin{frame}
\frametitle{Problématique du Salaire Juste}

\begin{alertblock}{Mesures de discrimination}
Les experts sont aussi conseillés par des statisticiens pour proposer les mesures de discrimination au niveau du pays et mondialement. $\mathcal{Y}^{J}_i$ désigne le Salaire Juste du $i^{\grave eme}$ individu parmi les $N$ personnes actives du pays. $Y^{J}$ correspond au Salaire Juste d'un individu choisi au hasard.
\begin{enumerate} 
\item \textbf{Discrimination Mondiale}~:  le Salaire Juste moyen s'écrit~:
$$\mu^{J}=\meanEmp[N]{\mathcal{Y}^{J}}=\frac1N\sum_{i=1}^N \mathcal{Y}^{J}_i=\EEE{Y^{J}}$$
\item \textbf{Discrimination Intérieure}~: la variance des Salaires Justes s'écrit~:
$$\sigma^2_{J}=\Big(\sdEmp[N]{\mathcal{Y}^{J}}\Big)^2=\frac1N\sum_{i=1}^N \Big(\mathcal{Y}^{J}_i-\meanEmp[N]{\mathcal{Y}^{J}}\Big)^2=\VVV{Y^{J}}$$
\end{enumerate}
\end{alertblock}
\end{frame}












%\beamertemplateshadingbackground{green!50}{yellow!50}
\begin{frame}<1->
\setbeamercolor{header}{fg=black,bg=blue!40!white}
 \hspace*{2.5cm}\begin{beamerboxesrounded}[width=6cm,shadow=true,lower=header]{}
  \pgfsetxvec{\pgfpoint{6cm}{0cm}}
\pgfsetyvec{\pgfpoint{0cm}{0.5cm}}
\begin{pgfpicture}{0cm}{0cm}{6cm}{0.5cm}

  \only<1-4>{
\pgfputat{\pgfxy(0.5,0.5)}{\pgfbox[center,center]{\textbf{\large Estimation}}}}
\only<5-8>{
\pgfputat{\pgfxy(0.5,0.5)}{\pgfbox[center,center]{\textbf{\large Qualité}}}}
\only<9-10>{
\pgfputat{\pgfxy(0.5,0.5)}{\pgfbox[center,center]{\textbf{\large Erreur standard}}}}
\only<11-16>{
\pgfputat{\pgfxy(0.5,0.5)}{\pgfbox[center,center]{\textbf{\large Mesure d'écart standardisé}}}}
\only<17-17>{
\pgfputat{\pgfxy(0.5,0.5)}{\pgfbox[center,center]{\textbf{\large A.E.P. ecart standardisé}}}}

  \end{pgfpicture}

\end{beamerboxesrounded}

\setbeamercolor{postit}{fg=black,bg=green!40!white}
%\begin{beamercolorbox}[sep=1em,wd=12cm]{postit}
\begin{beamerboxesrounded}[shadow=true,lower=postit]{}
\pgfsetxvec{\pgfpoint{11cm}{0cm}}
\pgfsetyvec{\pgfpoint{0cm}{2.1cm}}
\begin{pgfpicture}{0cm}{0cm}{11cm}{2.1cm}

\only<1-4>{
\pgfputat{\pgfxy(0.5,0.5)}{\pgfbox[center,center]{\begin{minipage}{11cm}Les paramètres (d'intérêt) $\mu^{J}$ et $\sigma^2_{J}$ (appelé, $\theta^\bullet$ dans un cadre général) sont donc supposés \textbf{inconnus} car la taille $N$ de la population est trop grande. Proposez les estimations~?
\end{minipage}}}}
\only<5-8>{
\pgfputat{\pgfxy(0.5,0.5)}{\pgfbox[center,center]{\begin{minipage}{11cm} Quelles sont les qualités souhaitables pour l'estimation d'un paramètre d'intérêt~? Pouvez-vous les traduire à partir des $m$ estimations potentielles~? (\textit{N.B.: l'estimation du jour J est choisi parmi celles-ci})
\end{minipage}}}}
\only<9-10>{
\pgfputat{\pgfxy(0.5,0.5)}{\pgfbox[center,center]{\begin{minipage}{11cm} Les statisticiens (mathématiciens) proposent généralement l'estimation $\Est{\theta}{y}$ d'un paramètre inconnu $\theta^\bullet$ accompagnée de l'estimation $\Est{\sigma_{\widehat{\theta^\bullet}}}{y}$ de sa qualité $\sigma_{\widehat{\theta^\bullet}}$.
\end{minipage}}}}
\only<11-16>{
\pgfputat{\pgfxy(0.5,0.5)}{\pgfbox[center,center]{\begin{minipage}{11cm}\only<11>{\textbf{Objectif~:} Etude de la loi de l'écart $\Est{\theta^\bullet}{Y}-\theta^\bullet$ (potentiellement, fort utile pour construction d'outils statistiques)}\only<12>{\textbf{Problème:} Loi de $\Est{\theta^\bullet}{Y}-\theta^\bullet$ généralement inconnue car dépendant d'un paramètre de nuisance inconnu ${\color{red}\sigma_{\widehat\theta^\bullet}}$.}\only<13>{\textbf{La solution~:} La loi de l'écart standardisé (centrage et réduction)\\
\centerline{$\displaystyle\delta_{\widehat\theta^\bullet,\theta^\bullet}(\Vect{Y}):=\frac{{\color{purple}\Est{\theta^\bullet}{Y}}-\theta^\bullet}{{\color{purple}\Est{\sigma_{\widehat{\theta^\bullet}}}{Y}}}\SuitApprox {\color{blue}\mathcal{N}(0,1)}$}}
\only<14>{\textbf{La solution~:} La loi de l'écart standardisé (centrage et réduction)\\ 
\centerline{$\displaystyle\delta_{\widehat\mu^J,\mu^J}(\Vect{Y}):=\frac{{\color{purple}\Est{\mu^J}{Y}}-\mu^J}{\color{purple}\Est{\sigma_{\widehat{\mu^J}}}{Y}}\SuitApprox {\color{blue}\mathcal{N}(0,1)}$ avec ${\color{purple}\Est{\sigma_{\widehat{\mu^J}}}{Y}}:=\displaystyle\frac{{\color{purple}\Est{\sigma_J}{Y}}}{\sqrt{n}}$}}\only<15>{\textbf{Application~:} Salaire Juste avec population fictive fixée expérimentalement à $\mu^J=100$ et $\sigma_J=10$ avec taille d'échantillon $n=1000$.}\only<16>{\textbf{Remarque~:} Chaque ligne du tableau serait un résultat possible pour le \textbf{jour~J}. Observez notamment la $1^{\grave ere}$ et $3^{\grave eme}$ colonnes (estimations des paramètres $\mu^J$ et $\sigma_J$ puis l'erreur standard pour $\mu^J$)!}
\end{minipage}}}}
\only<17-17>{
\pgfputat{\pgfxy(0.5,0.5)}{\pgfbox[center,center]{\begin{minipage}{11cm}\textbf{Expérimentation~:} Relation entre A.E.P. et A.M.P sur
\centerline{$\Delta:=\displaystyle\delta_{\widehat\mu^J,\mu^J}(\Vect{Y}):=\frac{{\color{purple}\Est{\mu^J}{Y}}-\mu^J}{\color{purple}\Est{\sigma_{\widehat{\mu^J}}}{Y}}\SuitApprox {\color{blue}\mathcal{N}(0,1)}$}
\end{minipage}}}}

\end{pgfpicture}

\end{beamerboxesrounded}
%\end{beamercolorbox}

\setbeamercolor{postex}{fg=black,bg=yellow!50!white}
%\begin{beamercolorbox}[sep=1em,wd=12cm]{postex}
\begin{beamerboxesrounded}[shadow=true,lower=postex]{}
\pgfsetxvec{\pgfpoint{11cm}{0cm}}
\pgfsetyvec{\pgfpoint{0cm}{5cm}}
\begin{pgfpicture}{0cm}{0cm}{11cm}{5cm}

\only<1-4>{
\pgfputat{\pgfxy(0.5,1.0)}{\pgfbox[center,top]{\begin{minipage}{11cm}\only<2-4>{\underline{\only<2>{Future estimation $\Est{\theta^\bullet}{Y}$}\only<3>{Estimation $\Est{\theta^\bullet}{y}$ du Jour J}\only<4>{Estimations potentielles $\Est{\theta^\bullet}{y_{[k]}}$}:} (à partir \only<2>{d'un futur échantillon $\Vect{Y}$}\only<3>{de l'échantillon $\Vect{y}$ du Jour J}\only<4>{d'échantillons $\Vect{y_{[k]}}$}) $$\Est{\mu^{J}}{\only<2>{Y}\only<3>{y}\only<4>{y_{[k]}}}:=\meanEmp[n]{\only<2>{Y}\only<3>{y}\only<4>{y_{[k]}}}:=\frac1{{\color{blue}}n}\sum_{i=1}^{\color{blue}n} \only<2>{Y_i}\only<3>{y_i}\only<4>{y_{i,[k]}}$$ et $$\Est{\sigma^2_{J}}{\only<2>{Y}\only<3>{y}\only<4>{y_{[k]}}}:=\frac1{\color{red}n-1}\sum_{i=1}^{{\color{blue}n}} \Big(\only<2>{Y_i}\only<3>{y_i}\only<4>{y_{i,[k]}}-\meanEmp[n]{\only<2>{Y}\only<3>{y}\only<4>{y_{[k]}}}\Big)^2$$}
\end{minipage}}}}
\only<5-8>{
\pgfputat{\pgfxy(0.5,1.0)}{\pgfbox[center,top]{\begin{minipage}{11cm}\begin{itemize}
\item<6-8> \textbf{Autour du paramètre}~: \alt<6>{\color{purple}leur moyenne égale au paramètre inconnu $\theta^\bullet$ (\textbf{A.M.P.~:} estimateur sans biais)\\
\centerline{$\displaystyle\meanEmp[m]{\Est{\theta^\bullet}{y_{[\cdot]}}}\simeq \meanEmp[\infty]{\Est{\theta^\bullet}{y_{[\cdot]}}}=\EEE{\Est{\theta^\bullet}{Y}}=\theta^\bullet$} 
}{\color{gray}leur moyenne égale au paramètre inconnu $\theta^\bullet$ (\textbf{A.M.P.~:} estimateur sans biais)\\
\centerline{$\displaystyle\meanEmp[m]{\Est{\theta^\bullet}{y_{[\cdot]}}}\simeq \meanEmp[\infty]{\Est{\theta^\bullet}{y_{[\cdot]}}}=\EEE{\Est{\theta^\bullet}{Y}}=\theta^\bullet$} 
} 
\item<6-8> \textbf{Faible dispersion}~: \alt<7>{\color{purple}leur écart-type (ou variance) d'autant plus petit que $n$ grandit (\textbf{A.M.P.~:} estimateur convergent)\\
\centerline{$\displaystyle\sdEmp[m]{\Est{\theta^\bullet}{y_{[\cdot]}}}\simeq \sdEmp[\infty]{\Est{\theta^\bullet}{y_{[\cdot]}}}=\sigma(\Est{\theta^\bullet}{Y})\mathop{\longrightarrow}_{n\rightarrow +\infty} 0$} 
}{\color{gray}leur écart-type (ou variance) d'autant plus petit que $n$ grandit (\textbf{A.M.P.~:} estimateur convergent)\\
\centerline{$\displaystyle\sdEmp[m]{\Est{\theta^\bullet}{y_{[\cdot]}}}\simeq \sdEmp[\infty]{\Est{\theta^\bullet}{y_{[\cdot]}}}=\sigma(\Est{\theta^\bullet}{Y})\mathop{\longrightarrow}_{n\rightarrow +\infty} 0$} 
}
\end{itemize}
\only<8>{
$\Rightarrow$ \noindent \textbf{Pb}~: qualité d'estimation ${\color{red}\sigma_{\widehat{\theta^\bullet}}}:=\sigma(\Est{\theta^\bullet}{Y})$  est un paramètre \textbf{inconnu}! Peut-on espérer l'estimer à partir de l'échantillon~$\Vect{y}$~?}
\end{minipage}}}}
\only<9-10>{
\pgfputat{\pgfxy(0.5,1.0)}{\pgfbox[center,top]{\begin{minipage}{11cm} (Voir le tableau dans votre caisse à outils pour la liste de toutes les erreurs standard associées aux différents paramètres!)\\
Pour illustrer comment cela est possible, étudions le paramètre moyenne $\mu^\bullet$~:\\
\centerline{${\color{red}\sigma_{\widehat{\mu^\bullet}}}=\frac{{\color{red}\sigma_\bullet}}{\sqrt{n}}
\invisible<9>{\mbox{ estimé par } {\color{blue}\Est{\sigma_{\widehat{\mu^\bullet}}}{Y}}=\frac{\only<10>{{\color{blue}\Est{\sigma_\bullet}{Y}}}}{\sqrt{n}}}$} 
\end{minipage}}}}
\only<11-16>{
\pgfputat{\pgfxy(0.5,1.0)}{\pgfbox[center,top]{\begin{minipage}{11cm}
{\small \only<11>{\begin{tabular}{c|cccc}
        $\Vect{Y}$
         & 
    
        $\Est{\theta^\bullet}{Y}$
         & 
    
        $\Est{\sigma_{\widehat{\theta^\bullet}}}{Y}$
         & 
    
        \phantom{$\Est{\theta^\bullet}{Y}-\theta^\bullet$
        } & 
    
        \phantom{$\delta_{\widehat\theta^\bullet,\theta^\bullet}(\Vect{Y})$
        }
    \\ \cline{1-3}

    
        $\Vect{y_{[1]}}$
         & 
    
        $\Est{\theta^\bullet}{y_{[1]}}$
         & 
    
        $\Est{\sigma_{\widehat{\theta^\bullet}}}{y_{[1]}}$
         & 
    
        \phantom{$\Est{\theta^\bullet}{y_{[1]}}-\theta^\bullet$
        } & 
    
        \phantom{$\delta_{\widehat\theta^\bullet,\theta^\bullet}(\Vect{y_{[1]}})$
        }
    \\ 

    
        $\Vect{y_{[2]}}$
         & 
    
        $\Est{\theta^\bullet}{y_{[2]}}$
         & 
    
        $\Est{\sigma_{\widehat{\theta^\bullet}}}{y_{[2]}}$
         & 
    
        \phantom{$\Est{\theta^\bullet}{y_{[2]}}-\theta^\bullet$
        } & 
    
        \phantom{$\delta_{\widehat\theta^\bullet,\theta^\bullet}(\Vect{y_{[2]}})$
        }
    \\ 

    
        \vdots
         & 
    
        \vdots
         & 
    
        \vdots
         & 
    
        \phantom{\vdots
        } & 
    
        \phantom{\vdots
        }
    \\ 

    
        $\Vect{y_{[m-1]}}$
         & 
    
        $\Est{\theta^\bullet}{y_{[m-1]}}$
         & 
    
        $\Est{\sigma_{\widehat{\theta^\bullet}}}{y_{[m-1]}}$
         & 
    
        \phantom{$\Est{\theta^\bullet}{y_{[m-1]}}-\theta^\bullet$
        } & 
    
        \phantom{$\delta_{\widehat\theta^\bullet,\theta^\bullet}(\Vect{y_{[m-1]}})$
        }
    \\ 

    
        $\Vect{y_{[m]}}$
         & 
    
        $\Est{\theta^\bullet}{y_{[m]}}$
         & 
    
        $\Est{\sigma_{\widehat{\theta^\bullet}}}{y_{[m]}}$
         & 
    
        \phantom{$\Est{\theta^\bullet}{y_{[m]}}-\theta^\bullet$
        } & 
    
        \phantom{$\delta_{\widehat\theta^\bullet,\theta^\bullet}(\Vect{y_{[m]}})$
        }
    \\ 

    
        \vdots
         & 
    
        \vdots
         & 
    
        \vdots
         & 
    
        \phantom{\vdots
        } & 
    
        \phantom{\vdots
        }
    \\ \cline{1-3}

    
        \phantom{Loi ($n\geq 30$)
        } & 
    
        \phantom{$\mathcal{N}(\theta^\bullet,{\color{red}\sigma_{\widehat\theta^\bullet}})$
        } & 
    
        \phantom{
        } & 
    
        \phantom{$\mathcal{N}(0,{\color{red}\sigma_{\widehat\theta^\bullet}})$
        } & 
    
        \phantom{{\color{blue}$\mathcal{N}(0,1)$}
        }
    \\ 

    \end{tabular}
}\only<12>{\begin{tabular}{c|cccc}
        $\Vect{Y}$
         & 
    
        $\Est{\theta^\bullet}{Y}$
         & 
    
        $\Est{\sigma_{\widehat{\theta^\bullet}}}{Y}$
         & 
    
        $\Est{\theta^\bullet}{Y}-\theta^\bullet$
         & 
    
        \phantom{$\delta_{\widehat\theta^\bullet,\theta^\bullet}(\Vect{Y})$
        }
    \\ \cline{1-4}

    
        $\Vect{y_{[1]}}$
         & 
    
        $\Est{\theta^\bullet}{y_{[1]}}$
         & 
    
        $\Est{\sigma_{\widehat{\theta^\bullet}}}{y_{[1]}}$
         & 
    
        $\Est{\theta^\bullet}{y_{[1]}}-\theta^\bullet$
         & 
    
        \phantom{$\delta_{\widehat\theta^\bullet,\theta^\bullet}(\Vect{y_{[1]}})$
        }
    \\ 

    
        $\Vect{y_{[2]}}$
         & 
    
        $\Est{\theta^\bullet}{y_{[2]}}$
         & 
    
        $\Est{\sigma_{\widehat{\theta^\bullet}}}{y_{[2]}}$
         & 
    
        $\Est{\theta^\bullet}{y_{[2]}}-\theta^\bullet$
         & 
    
        \phantom{$\delta_{\widehat\theta^\bullet,\theta^\bullet}(\Vect{y_{[2]}})$
        }
    \\ 

    
        \vdots
         & 
    
        \vdots
         & 
    
        \vdots
         & 
    
        \vdots
         & 
    
        \phantom{\vdots
        }
    \\ 

    
        $\Vect{y_{[m-1]}}$
         & 
    
        $\Est{\theta^\bullet}{y_{[m-1]}}$
         & 
    
        $\Est{\sigma_{\widehat{\theta^\bullet}}}{y_{[m-1]}}$
         & 
    
        $\Est{\theta^\bullet}{y_{[m-1]}}-\theta^\bullet$
         & 
    
        \phantom{$\delta_{\widehat\theta^\bullet,\theta^\bullet}(\Vect{y_{[m-1]}})$
        }
    \\ 

    
        $\Vect{y_{[m]}}$
         & 
    
        $\Est{\theta^\bullet}{y_{[m]}}$
         & 
    
        $\Est{\sigma_{\widehat{\theta^\bullet}}}{y_{[m]}}$
         & 
    
        $\Est{\theta^\bullet}{y_{[m]}}-\theta^\bullet$
         & 
    
        \phantom{$\delta_{\widehat\theta^\bullet,\theta^\bullet}(\Vect{y_{[m]}})$
        }
    \\ 

    
        \vdots
         & 
    
        \vdots
         & 
    
        \vdots
         & 
    
        \vdots
         & 
    
        \phantom{\vdots
        }
    \\ \cline{1-4}

    
        Loi ($n\geq 30$)
         & 
    
        $\mathcal{N}(\theta^\bullet,{\color{red}\sigma_{\widehat\theta^\bullet}})$
         & 
    
        
         & 
    
        $\mathcal{N}(0,{\color{red}\sigma_{\widehat\theta^\bullet}})$
         & 
    
        \phantom{{\color{blue}$\mathcal{N}(0,1)$}
        }
    \\ 

    \end{tabular}
}\only<13>{\begin{tabular}{c|cccc}
        $\Vect{Y}$
         & 
    
        $\Est{\theta^\bullet}{Y}$
         & 
    
        $\Est{\sigma_{\widehat{\theta^\bullet}}}{Y}$
         & 
    
        $\Est{\theta^\bullet}{Y}-\theta^\bullet$
         & 
    
        $\delta_{\widehat\theta^\bullet,\theta^\bullet}(\Vect{Y})$
        
    \\ \cline{1-5}

    
        $\Vect{y_{[1]}}$
         & 
    
        $\Est{\theta^\bullet}{y_{[1]}}$
         & 
    
        $\Est{\sigma_{\widehat{\theta^\bullet}}}{y_{[1]}}$
         & 
    
        $\Est{\theta^\bullet}{y_{[1]}}-\theta^\bullet$
         & 
    
        $\delta_{\widehat\theta^\bullet,\theta^\bullet}(\Vect{y_{[1]}})$
        
    \\ 

    
        $\Vect{y_{[2]}}$
         & 
    
        $\Est{\theta^\bullet}{y_{[2]}}$
         & 
    
        $\Est{\sigma_{\widehat{\theta^\bullet}}}{y_{[2]}}$
         & 
    
        $\Est{\theta^\bullet}{y_{[2]}}-\theta^\bullet$
         & 
    
        $\delta_{\widehat\theta^\bullet,\theta^\bullet}(\Vect{y_{[2]}})$
        
    \\ 

    
        \vdots
         & 
    
        \vdots
         & 
    
        \vdots
         & 
    
        \vdots
         & 
    
        \vdots
        
    \\ 

    
        $\Vect{y_{[m-1]}}$
         & 
    
        $\Est{\theta^\bullet}{y_{[m-1]}}$
         & 
    
        $\Est{\sigma_{\widehat{\theta^\bullet}}}{y_{[m-1]}}$
         & 
    
        $\Est{\theta^\bullet}{y_{[m-1]}}-\theta^\bullet$
         & 
    
        $\delta_{\widehat\theta^\bullet,\theta^\bullet}(\Vect{y_{[m-1]}})$
        
    \\ 

    
        $\Vect{y_{[m]}}$
         & 
    
        $\Est{\theta^\bullet}{y_{[m]}}$
         & 
    
        $\Est{\sigma_{\widehat{\theta^\bullet}}}{y_{[m]}}$
         & 
    
        $\Est{\theta^\bullet}{y_{[m]}}-\theta^\bullet$
         & 
    
        $\delta_{\widehat\theta^\bullet,\theta^\bullet}(\Vect{y_{[m]}})$
        
    \\ 

    
        \vdots
         & 
    
        \vdots
         & 
    
        \vdots
         & 
    
        \vdots
         & 
    
        \vdots
        
    \\ \cline{1-5}

    
        Loi ($n\geq 30$)
         & 
    
        $\mathcal{N}(\theta^\bullet,{\color{red}\sigma_{\widehat\theta^\bullet}})$
         & 
    
        
         & 
    
        $\mathcal{N}(0,{\color{red}\sigma_{\widehat\theta^\bullet}})$
         & 
    
        {\color{blue}$\mathcal{N}(0,1)$}
        
    \\ 

    \end{tabular}
}\only<14>{\begin{tabular}{c|cccc}
        $\Vect{Y}$
         & 
    
        $\Est{\mu^{J}}{Y}$
         & 
    
        $\Est{\sigma_{\widehat{\mu^{J}}}}{Y}$
         & 
    
        $\Est{\mu^{J}}{Y}-\mu^{J}$
         & 
    
        $\delta_{\widehat\mu^{J},\mu^{J}}(\Vect{Y})$
        
    \\ \cline{1-5}

    
        $\Vect{y_{[1]}}$
         & 
    
        $\Est{\mu^J}{y_{[1]}}$
         & 
    
        $\Est{\sigma_{\widehat{\mu^J}}}{y_{[1]}}$
         & 
    
        $\Est{\mu^J}{y_{[1]}}-\mu^J$
         & 
    
        $\delta_{\widehat\mu^J,\mu^J}(\Vect{y_{[1]}})$
        
    \\ 

    
        $\Vect{y_{[2]}}$
         & 
    
        $\Est{\mu^J}{y_{[2]}}$
         & 
    
        $\Est{\sigma_{\widehat{\mu^J}}}{y_{[2]}}$
         & 
    
        $\Est{\mu^J}{y_{[2]}}-\mu^J$
         & 
    
        $\delta_{\widehat\mu^J,\mu^J}(\Vect{y_{[2]}})$
        
    \\ 

    
        \vdots
         & 
    
        \vdots
         & 
    
        \vdots
         & 
    
        \vdots
         & 
    
        \vdots
        
    \\ 

    
        $\Vect{y_{[m-1]}}$
         & 
    
        $\Est{\mu^J}{y_{[m-1]}}$
         & 
    
        $\Est{\sigma_{\widehat{\mu^J}}}{y_{[m-1]}}$
         & 
    
        $\Est{\mu^J}{y_{[m-1]}}-\mu^J$
         & 
    
        $\delta_{\widehat\mu^J,\mu^J}(\Vect{y_{[m-1]}})$
        
    \\ 

    
        $\Vect{y_{[m]}}$
         & 
    
        $\Est{\mu^J}{y_{[m]}}$
         & 
    
        $\Est{\sigma_{\widehat{\mu^J}}}{y_{[m]}}$
         & 
    
        $\Est{\mu^J}{y_{[m]}}-\mu^J$
         & 
    
        $\delta_{\widehat\mu^J,\mu^J}(\Vect{y_{[m]}})$
        
    \\ 

    
        \vdots
         & 
    
        \vdots
         & 
    
        \vdots
         & 
    
        \vdots
         & 
    
        \vdots
        
    \\ \cline{1-5}

    
        Loi ($n\geq 30$)
         & 
    
        $\mathcal{N}(\mu^J,{\color{red}\sigma_{\widehat\mu^J}})$
         & 
    
        
         & 
    
        $\mathcal{N}(0,{\color{red}\sigma_{\widehat\mu^J}})$
         & 
    
        {\color{blue}$\mathcal{N}(0,1)$}
        
    \\ 

    \end{tabular}
}\only<15-16>{\begin{tabular}{c|ccccc}
        $\Vect{Y}$
         & 
    
        $\Est{\mu^{J}}{Y}$
         & 
    
        $\Est{\sigma_{J}}{Y}$
         & 
    
        $\Est{\sigma_{\widehat{\mu^{J}}}}{Y}$
         & 
    
        $\Est{\mu^{J}}{Y}-\mu^{J}$
         & 
    
        $\delta_{\widehat\mu^{J},\mu^{J}}(\Vect{Y})$
        
    \\ \cline{1-6}

    
        $\Vect{y_{[1]}}$
         & 
    
        $99.91$
         & 
    
        $10.0231$
         & 
    
        $0.317$
         & 
    
        $-0.09$
         & 
    
        $-0.29$
        
    \\ 

    
        $\Vect{y_{[2]}}$
         & 
    
        $99.65$
         & 
    
        $9.2615$
         & 
    
        $0.2929$
         & 
    
        $-0.35$
         & 
    
        $-1.19$
        
    \\ 

    
        \vdots
         & 
    
        \vdots
         & 
    
        \vdots
         & 
    
        \vdots
         & 
    
        \vdots
         & 
    
        \vdots
        
    \\ 

    
        \texttt{y} en \texttt{R}
         & 
    
        {\tiny\texttt{mean(y)}}
         & 
    
        {\tiny\texttt{sd(y)}}
         & 
    
        {\tiny\texttt{seMean(y)}}
         & 
    
        {\tiny\texttt{mean(y)-100}}
         & 
    
        {\tiny\texttt{(mean(y)-100)/seMean(y)}}
        
    \\ 

    
        \vdots
         & 
    
        \vdots
         & 
    
        \vdots
         & 
    
        \vdots
         & 
    
        \vdots
         & 
    
        \vdots
        
    \\ 

    
        $\Vect{y_{[9999]}}$
         & 
    
        $100.2$
         & 
    
        $9.9372$
         & 
    
        $0.3142$
         & 
    
        $0.2$
         & 
    
        $0.63$
        
    \\ 

    
        $\Vect{y_{[10000]}}$
         & 
    
        $99.94$
         & 
    
        $9.7991$
         & 
    
        $0.3099$
         & 
    
        $-0.06$
         & 
    
        $-0.21$
        
    \\ \cline{1-6}

    
        mean
         & 
    
        $100.0043$
         & 
    
        $10.034$
         & 
    
        $0.3173$
         & 
    
        $0.0043$
         & 
    
        $-0.0294$
        
    \\ 

    
        sd
         & 
    
        $0.3178$
         & 
    
        $0.63$
         & 
    
        $0.0199$
         & 
    
        $0.3178$
         & 
    
        $1.0058$
        
    \\ 

    \end{tabular}
}
}
\end{minipage}}}}
\only<17-17>{
\pgfputat{\pgfxy(0.5,1.0)}{\pgfbox[center,top]{\begin{minipage}{11cm}
{\scriptsize 
\begin{center}\begin{tabular}{|c|c|c|c|}\hline
\phantom{$\Big($}$\meanEmp[m]{\delta_{[\cdot]}<-3}$&\phantom{$\Big($}$\meanEmp[m]{\delta_{[\cdot]}\in [-3,-1.5[}$&\phantom{$\Big($}$\meanEmp[m]{\delta_{[\cdot]}\in [-1.5,-0.5]}$&\phantom{$\Big($}$\meanEmp[m]{\delta_{[\cdot]}\in [-0.5,0.5[}$
\\\hline
\phantom{$\Big($}$0.23\%$&\phantom{$\Big($}$7.4\%$&\phantom{$\Big($}$23.94\%$&\phantom{$\Big($}$37.55\%$
\\\hline
\phantom{$\Big($}$\PPP{\Delta<-3}$&\phantom{$\Big($}$\PPP{\Delta\in [-3,-1.5[}$&\phantom{$\Big($}$\PPP{\Delta\in [-1.5,-0.5]}$&\phantom{$\Big($}$\PPP{\Delta\in [-0.5,0.5[}$
\\\hline
\phantom{$\Big($}$0.13\%$&\phantom{$\Big($}$6.55\%$&\phantom{$\Big($}$24.17\%$&\phantom{$\Big($}$38.29\%$
\\\hline
\end{tabular}\end{center}

\begin{center}\begin{tabular}{|c|c|c|c|c|}\hline
\phantom{$\Big($}$\meanEmp[m]{\delta_{[\cdot]}\in [0.5,1.5[}$&\phantom{$\Big($}$\meanEmp[m]{\delta_{[\cdot]}\in [1.5,3[}$&\phantom{$\Big($}$\meanEmp[m]{\delta_{[\cdot]}\geq3}$&\phantom{$\Big($}$\meanEmp[m]{\delta_{[\cdot]}}$&\phantom{$\Big($}$\sdEmp[m]{\delta_{[\cdot]}}$
\\\hline
\phantom{$\Big($}$25.05\%$&\phantom{$\Big($}$5.73\%$&\phantom{$\Big($}$0.1\%$&\phantom{$\Big($}$-0.0294$&\phantom{$\Big($}$1.0058$
\\\hline
\phantom{$\Big($}$\PPP{\Delta\in [0.5,1.5[}$&\phantom{$\Big($}$\PPP{\Delta\in [1.5,3[}$&\phantom{$\Big($}$\PPP{\Delta\geq3}$&\phantom{$\Big($}$\EEE{\Delta}$&\phantom{$\Big($}$\sigma(\Delta)$
\\\hline
\phantom{$\Big($}$24.17\%$&\phantom{$\Big($}$6.55\%$&\phantom{$\Big($}$0.13\%$&\phantom{$\Big($}$0$&\phantom{$\Big($}$1$
\\\hline
\end{tabular}\end{center}
}
\end{minipage}}}}

\end{pgfpicture}

\end{beamerboxesrounded}
%\end{beamercolorbox}
\begin{tikzpicture}[remember picture,overlay]
  \node [rotate=30,scale=10,text opacity=0.05]
    at (current page.center) {CQLS};
\end{tikzpicture}
\end{frame}





\end{document}


