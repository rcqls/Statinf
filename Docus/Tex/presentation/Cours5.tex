
\documentclass[11pt]{beamer}
%Packages
\usepackage{multirow}
\usepackage{graphicx}
\usepackage[utf8x]{inputenc}
\usepackage{aeguill}
\usepackage{amssymb}
\usepackage[french]{babel}
\usepackage{pgf,pgfarrows,pgfnodes}
\usepackage{xmpmulti}
\usepackage{multimedia}
\usepackage{bbm}
\usepackage{bm}
\usepackage{mathrsfs,dsfont}
\usepackage{xcolor}
\usepackage{colortbl}
\usepackage{longtable}
\usepackage[T1]{fontenc}
\usepackage[scaled]{beramono}
\usepackage{float}
\usepackage{xkeyval,calc,listings,tikz}
\usepackage{fancyvrb}

%Preamble

\input Cours/cqlsInclude
\input Cours/testInclude


\mode<article>{\usepackage{fullpage}}
\usefonttheme{structureitalicserif}


\definecolor{VertFonce}{rgb}{0,.4,.0}

\mode<presentation>
{
  %\usetheme{Warsaw}
  % or ...
\usetheme{Boadilla}

  \setbeamercovered{transparent=5}
  % or whatever (possibly just delete it)
}
%\setbeamercovered{dynamic}


\subject{Talks}

\AtBeginSection[]
{
  
\begin{frame}<beamer>
	\frametitle{Plan}
    \tableofcontents[currentsection,currentsubsection]
\end{frame}

}


\definecolor{show}{rgb}{0.59,0.29,0.59}

\setbeamercovered{invisible}
\newcommand{\Sim}{{\star}}
\newcommand{\ok}{ \textcolor{green}{\large$\surd$}}
\newcommand{\nok}{ \textcolor{red}{\large X}}


\usetikzlibrary{arrows,%
  calc,%
  fit,%
  patterns,%
  plotmarks,%
  shapes.geometric,%
  shapes.misc,%
  shapes.symbols,%
  shapes.arrows,%
  shapes.callouts,%
  shapes.multipart,%
  shapes.gates.logic.US,%
  shapes.gates.logic.IEC,%
  er,%
  automata,%
  backgrounds,%
  chains,%
  topaths,%
  trees,%
  petri,%
  mindmap,%
  matrix,%
  calendar,%
  folding,%
  fadings,%
  through,%
  positioning,%
  scopes,%
  decorations.fractals,%
  decorations.shapes,%
  decorations.text,%
  decorations.pathmorphing,%
  decorations.pathreplacing,%
  decorations.footprints,%
  decorations.markings,%
  shadows}
\tikzset{
  every plot/.style={prefix=plots/pgf-},
  shape example/.style={
    color=black!30,
    draw,
    fill=yellow!30,
    line width=.5cm,
    inner xsep=2.5cm,
    inner ysep=0.5cm}
}


\definecolor{darkgreen}{rgb}{0,.4,.0}

%Styles

%Title

\beamertemplateshadingbackground{green!50}{yellow!50}
\title[Problématiques Produits A et B]
{Cours de Statistiques Inférentielles}
\author{CQLS~: cqls@upmf-grenoble.fr}
\date{\today}

\begin{document}
\maketitle


%\begin{frame}
%  \titlepage
%\end{frame}

\section[A.E.P.]{Approche Expérimentale des Probabilités}

\begin{frame}<1->[label=aep]
\frametitle<1->{\textbf{A}pproche \textbf{E}xpérimentale des \textbf{P}robabilités : l'Expérimentateur versus le Matheux}
\begin{columns}
\begin{column}{5cm}
      \begin{enumerate}
     \item[] \textbf{L'Expérimentateur ~:} 
        \item<1-10| alert@1-4>
         Réaliser $m$ expériences
        \item<5-10| alert@5-8>
          Répartition des $\Est{\mu^\bullet}{y^\bullet_{[j]}}$ {\small représentées par $m$ briques de surface $1/m$ et de largeur $1/n$ empilées l'une après l'autre en les centrant en abscisse en leur valeur.}
          )
  \item<9-10>[] \textbf{Le Matheux :}      
  \item<9-10| alert@9>
          Je le savais \textbf{à l'avance} pour $m\to+\infty$
           \item<10| alert@10>
           $\Est{\mu^\bullet}{Y^\bullet}\SuitApprox \mathcal{N}(\mu^\bullet,\frac{\sigma_\bullet}{\sqrt{n}})$
      \end{enumerate}
     
\end{column}    
\begin{column}{6.5cm}







\begin{pgfpicture}{-0.5cm}{0cm}{6cm}{6cm}
      \only<2-10>{\pgfputat{\pgfxy(0,4.4)}{\pgfbox[center,center]{$14.6\%$}}}
       \only<3-10>{\pgfputat{\pgfxy(2,4.4)}{\pgfbox[center,center]{$16.3\%$}}}
       \only<4-5>{\pgfputat{\pgfxy(5.5,4.4)}{\pgfbox[center,center]{$14.1\%$}}}
      \only<6>{\pgfputat{\pgfxy(5.5,4.4)}{\pgfbox[center,center]{$15.1\%$}}}
      \only<7>{\pgfputat{\pgfxy(5.5,4.4)}{\pgfbox[center,center]{$14.7\%$}}}
      \only<8>{\pgfputat{\pgfxy(5.5,4.4)}{\pgfbox[center,center]{$16.3\%$}}}
      \only<9-10>{\pgfputat{\pgfxy(5.5,4.4)}{\pgfbox[center,center]{$\cdots$}}}
     \pgfnodebox{EstY}[stroke]{\pgfxy(3,6.3)}{Future $\Est{\mu^\bullet}{Y^\bullet}$}{2pt}{2pt}
      \only<2-10>{\pgfnodebox{esty1}[stroke]{\pgfxy(0,5)}{$\Est{\mu^\bullet}{y^\bullet_{[1]}}$}{2pt}{2pt}}
       \only<3-10>{\pgfnodebox{esty2}[stroke]{\pgfxy(2,5)}{$\Est{\mu^\bullet}{y^\bullet_{[2]}}$}{2pt}{2pt}}
     \only<4-10>{\pgfputat{\pgfxy(3.75,5)}{\pgfbox[center,center]{$\cdots\cdots$}}
        \pgfnodebox{estym}[stroke]{\pgfxy(5.5,5)}{$\Est{\mu^\bullet}{y^\bullet_{[m]}}$}{2pt}{2pt}}
  \pgfsetendarrow{\pgfarrowto}
\only<2,5-10>{\pgfnodeconncurve{EstY}{esty1}{-90}{90}{0.5cm}{0.5cm}}
\only<2>{\pgfnodelabel{EstY}{esty1}[0.5][-3pt]{\pgfbox[right,base]{$1^{\grave eme}$ réalisation}}
}
\only<3,5-10>{\pgfnodeconncurve{EstY}{esty2}{-90}{90}{0.5cm}{0.5cm}}
\only<3>{\pgfnodelabel{EstY}{esty2}[0.5][-3pt]{\pgfbox[center,base]{$2^{\grave eme}$ réalisation}}
}
\only<4-10>{\pgfnodeconncurve{EstY}{estym}{-90}{90}{0.5cm}{0.5cm}}
\only<4>{\pgfnodelabel{EstY}{estym}[0.5][-3pt]{\pgfbox[center,base]{$m^{\grave eme}$ réalisation}}
}
\only<5>{
\pgfdeclareimage[interpolate=true,height=5cm]{image1}{img/hist100A15}
\pgfputat{\pgfxy(3,1.5)}{\pgfbox[center,center]{\pgfuseimage{image1}}}
\pgfputat{\pgfxy(3.5,3.75)}{\pgfbox[center,center]{$m=100$}}
}
\only<6>{
\pgfdeclareimage[interpolate=true,height=5cm]{image2}{img/hist1000A15}
\pgfputat{\pgfxy(3,1.5)}{\pgfbox[center,center]{\pgfuseimage{image2}}}
\pgfputat{\pgfxy(3.5,3.75)}{\pgfbox[center,center]{$m=1000$}}
}
\only<7>{
\pgfdeclareimage[interpolate=true,height=5cm]{image3}{img/hist5000A15}
\pgfputat{\pgfxy(3,1.5)}{\pgfbox[center,center]{\pgfuseimage{image3}}}
\pgfputat{\pgfxy(3.5,3.75)}{\pgfbox[center,center]{$m=5000$}}
}
\only<8>{
\pgfdeclareimage[interpolate=true,height=5cm]{image4}{img/hist10000A15}
\pgfputat{\pgfxy(3,1.5)}{\pgfbox[center,center]{\pgfuseimage{image4}}}
\pgfputat{\pgfxy(3.5,3.75)}{\pgfbox[center,center]{$m=10000$}}
}
\only<9>{
\pgfdeclareimage[interpolate=true,height=5cm]{image5}{img/hist10000A15Norm}
\pgfputat{\pgfxy(3,1.5)}{\pgfbox[center,center]{\pgfuseimage{image5}}}
\pgfputat{\pgfxy(3.5,3.75)}{\pgfbox[center,center]{$m=+\infty$}}
}
\only<10>{
\pgfdeclareimage[interpolate=true,height=5cm]{image6}{img/hist10000A15NormContour}
\pgfputat{\pgfxy(3,1.5)}{\pgfbox[center,center]{\pgfuseimage{image6}}}
\pgfputat{\pgfxy(3.5,3.75)}{\pgfbox[center,center]{$m=10000$ vs $m=+\infty$}}
}
\end{pgfpicture}
\end{column}
    
\end{columns} 
\end{frame}

\begin{frame}[label=recap]
\frametitle<1-3>{Réalisation d'une future estimation par l'\textbf{Expérimentateur}}
\frametitle<4-6>{Réalisation d'une future estimation par le \textbf{Matheux}}

\begin{columns}
\begin{column}{5cm}
\begin{center}
\only<1,4>{

\begin{tabular}{|c|c|}\hline
\multicolumn{2}{|c|}{ 
 
} \\\hline
j & $\Est{\mu^\bullet}{y^\bullet_{[j]}}$\\\hline
$\vdots$ & $\vdots$\\



4150

& 

14.4\%

\\



4151

& 

17.2\%

\\



4152

& 

15\%

\\


{\color[rgb]{1,0,0}
4153
}
& 
{\color[rgb]{1,0,0}
14.9\%
}
\\



4154

& 

13.7\%

\\



4155

& 

15.8\%

\\



4156

& 

14.6\%

\\

$\vdots$ & $\vdots$\\
\hline
\end{tabular}

}\only<2,5>{

\begin{tabular}{|c|c|}\hline
\multicolumn{2}{|c|}{ 
 
} \\\hline
j & $\Est{\mu^\bullet}{y^\bullet_{[j]}}$\\\hline
$\vdots$ & $\vdots$\\



2105

& 

15.3\%

\\



2106

& 

14.1\%

\\



2107

& 

13.2\%

\\


{\color[rgb]{1,0,0}
2108
}
& 
{\color[rgb]{1,0,0}
15.5\%
}
\\



2109

& 

16.7\%

\\



2110

& 

15.5\%

\\



2111

& 

14.5\%

\\

$\vdots$ & $\vdots$\\
\hline
\end{tabular}

}\only<3,6>{

\begin{tabular}{|c|c|}\hline
\multicolumn{2}{|c|}{ 
 
} \\\hline
j & $\Est{\mu^\bullet}{y^\bullet_{[j]}}$\\\hline
$\vdots$ & $\vdots$\\



3728

& 

14.9\%

\\



3729

& 

14.4\%

\\



3730

& 

14.8\%

\\


{\color[rgb]{1,0,0}
3731
}
& 
{\color[rgb]{1,0,0}
13.4\%
}
\\



3732

& 

14.9\%

\\



3733

& 

14.4\%

\\



3734

& 

16.4\%

\\

$\vdots$ & $\vdots$\\
\hline
\end{tabular}

}
\end{center}
\end{column}
\begin{column}{5cm}






\begin{pgfpicture}{0cm}{0cm}{6cm}{6cm}
\only<1>{
\pgfdeclareimage[interpolate=true,height=7.5cm]{brique1}{img/brique1}
\pgfputat{\pgfxy(2.5,3)}{\pgfbox[center,center]{\pgfuseimage{brique1}}}
}
\only<2>{
\pgfdeclareimage[interpolate=true,height=7.5cm]{brique2}{img/brique2}
\pgfputat{\pgfxy(2.5,3)}{\pgfbox[center,center]{\pgfuseimage{brique2}}}
}
\only<3>{
\pgfdeclareimage[interpolate=true,height=7.5cm]{brique3}{img/brique3}
\pgfputat{\pgfxy(2.5,3)}{\pgfbox[center,center]{\pgfuseimage{brique3}}}
}
\only<4>{
\pgfdeclareimage[interpolate=true,height=7.5cm]{point1}{img/point1}
\pgfputat{\pgfxy(2.5,3)}{\pgfbox[center,center]{\pgfuseimage{point1}}}
}
\only<5>{
\pgfdeclareimage[interpolate=true,height=7.5cm]{point2}{img/point2}
\pgfputat{\pgfxy(2.5,3)}{\pgfbox[center,center]{\pgfuseimage{point2}}}
}
\only<6>{
\pgfdeclareimage[interpolate=true,height=7.5cm]{point3}{img/point3}
\pgfputat{\pgfxy(2.5,3)}{\pgfbox[center,center]{\pgfuseimage{point3}}}
}
\end{pgfpicture}
\end{column}
\end{columns}

\end{frame}

\begin{frame}
\frametitle{Comment l'industriel doit-il utiliser ces informations~?}
\begin{exampleblock}{Réalisation d'une future estimation}
\begin{itemize}[<+-| alert@+>]
\item[$\to$] L'industriel s'imagine être le \textbf{jour~J} dans la situation où $\mu^\bullet=0.15$ (juste pas le marché)
\item[$\to$] Il prend alors conscience que ce qui peut lui arriver \textbf{le jour~J}, c'est équivalent (ou presque) à~:
\begin{enumerate}
\item Choisir au hasard une brique (i.e un $\Est{\mu^\bullet}{y_{[j]}}$ parmi les $m$)
\item Choisir au hasard un point sous la ``courbe $\mathcal{N}(\mu^\bullet,\frac{\sigma_\bullet}{\sqrt{n}})$'' associé à son abscisse représentant une réalisation au hasard de $\Est{\mu^\bullet}{Y}$ choisie parmi une infinité. 
\end{enumerate}
\item[$\Rightarrow$] Il voit clairement  la ``courbe $\mathcal{N}(\mu^\bullet,\frac{\sigma_\bullet}{\sqrt{n}})$'' comme un empilement d'une infinité de briques (``devenues des points'') associées à une infinité de réalisations possibles  de $\Est{\mu^\bullet}{Y}$.
\end{itemize}
\end{exampleblock}

\end{frame}




\pgfdeclareimage[width=11cm,height=5cm,interpolate=true]{quant}{img/quant}

\pgfdeclareimage[width=11cm,height=5cm,interpolate=true]{quantNorm}{img/quantNorm}

\pgfdeclareimage[width=11cm,height=5cm,interpolate=true]{quantExpNorm}{img/quantExpNorm}



%\beamertemplateshadingbackground{green!50}{yellow!50}
\begin{frame}<1->
\setbeamercolor{header}{fg=black,bg=blue!40!white}
 \hspace*{2.5cm}\begin{beamerboxesrounded}[width=6cm,shadow=true,lower=header]{}
  \pgfsetxvec{\pgfpoint{6cm}{0cm}}
\pgfsetyvec{\pgfpoint{0cm}{0.5cm}}
\begin{pgfpicture}{0cm}{0cm}{6cm}{0.5cm}

  \only<1-3>{
\pgfputat{\pgfxy(0.5,0.5)}{\pgfbox[center,center]{
\textbf{\large Produit~A~: Risque 1ère espèce}}}}

  \end{pgfpicture}

\end{beamerboxesrounded}

\setbeamercolor{postit}{fg=black,bg=green!40!white}
%\begin{beamercolorbox}[sep=1em,wd=12cm]{postit}
\begin{beamerboxesrounded}[shadow=true,lower=postit]{}
\pgfsetxvec{\pgfpoint{11cm}{0cm}}
\pgfsetyvec{\pgfpoint{0cm}{2.1cm}}
\begin{pgfpicture}{0cm}{0cm}{11cm}{2.1cm}

\only<1-3>{
\pgfputat{\pgfxy(0.5,1.0)}{\pgfbox[center,top]{\begin{minipage}{11cm}
{\small
$\begin{array}{l}
\only<2-3>{{\color{red}P(\Est{p^A}{Y^A}> 16.9\%)}}\only<2>{=}\only<3>{\simeq}{\color{blue}\meanEmp[\only<1,3>{m}\only<2>{\infty}]{\Est{p^A}{y^A_{[\cdot]} } >16.9\%}}\only<1>{=\mbox{Prop. des $\Big(\Est{p^A}{y^A_{[\cdot]} }\Big)_{\only<1,3>{10000}\only<2>{\infty} } $ supérieurs à 16.9\%}}\\
=\only<2>{\displaystyle\lim_{m\to\infty}}\frac1m\times\Big(\mbox{Nbre des $\Big(\Est{p^A}{y^A_{[\cdot]}}\Big)_{\only<1,3>{m}\only<2>{\infty}}$ supérieurs à 16.9\%}\Big)\\
\only<1,3>{=}\only<2>{\simeq} \mbox{Surface des \only<1,3>{\textbf{briques} associées}\only<2>{\textbf{points} associés}  aux $\Big(\Est{p^A}{y^A_{[\cdot]}}\Big)_{\only<1,3>{m}\only<2>{\infty}}$ supérieurs à 16.9\%}
\end{array}$}\end{minipage}}}}

\end{pgfpicture}

\end{beamerboxesrounded}
%\end{beamercolorbox}

\setbeamercolor{postex}{fg=black,bg=yellow!50!white}
%\begin{beamercolorbox}[sep=1em,wd=12cm]{postex}
\begin{beamerboxesrounded}[shadow=true,lower=postex]{}
\pgfsetxvec{\pgfpoint{11cm}{0cm}}
\pgfsetyvec{\pgfpoint{0cm}{5cm}}
\begin{pgfpicture}{0cm}{0cm}{11cm}{5cm}

\only<1>{
\pgfputat{\pgfxy(0.05,0.0)}{\pgfbox[left,bottom]{\pgfuseimage{quant}}}}
\only<2>{
\pgfputat{\pgfxy(0.05,0.0)}{\pgfbox[left,bottom]{\pgfuseimage{quantNorm}}}}
\only<3>{
\pgfputat{\pgfxy(0.05,0.0)}{\pgfbox[left,bottom]{\pgfuseimage{quantExpNorm}}}}

\end{pgfpicture}

\end{beamerboxesrounded}
%\end{beamercolorbox}
\begin{tikzpicture}[remember picture,overlay]
  \node [rotate=30,scale=10,text opacity=0.05]
    at (current page.center) {CQLS};
\end{tikzpicture}
\end{frame}

\section{P-valeur}

\pgfdeclareimage[width=11cm,height=5cm,interpolate=true]{diet}{img/diet}

\pgfdeclareimage[width=11cm,height=5cm,interpolate=true]{diet2}{img/diet2}

\pgfdeclareimage[width=11cm,height=5cm,interpolate=true]{diet3}{img/diet3}

\pgfdeclareimage[width=11cm,height=5cm,interpolate=true]{diet4}{img/diet4}













\pgfdeclareimage[width=11cm,height=5cm,interpolate=true]{diet5}{img/diet5}

\pgfdeclareimage[width=11cm,height=5cm,interpolate=true]{diet6}{img/diet6}








\pgfdeclareimage[width=11cm,height=5cm,interpolate=true]{diet7}{img/diet7}








%\beamertemplateshadingbackground{green!50}{yellow!50}
\begin{frame}<1->
\setbeamercolor{header}{fg=black,bg=blue!40!white}
 \hspace*{2.5cm}\begin{beamerboxesrounded}[width=6cm,shadow=true,lower=header]{}
  \pgfsetxvec{\pgfpoint{6cm}{0cm}}
\pgfsetyvec{\pgfpoint{0cm}{0.5cm}}
\begin{pgfpicture}{0cm}{0cm}{6cm}{0.5cm}

  \only<1-5,6>{
\pgfputat{\pgfxy(0.5,0.5)}{\pgfbox[center,center]{
\textbf{\large Exemple diététicien}}}}
\only<7>{
\pgfputat{\pgfxy(0.5,0.5)}{\pgfbox[center,center]{\textbf{\large Exemple diététicien (fin)}}}}

  \end{pgfpicture}

\end{beamerboxesrounded}

\setbeamercolor{postit}{fg=black,bg=green!40!white}
%\begin{beamercolorbox}[sep=1em,wd=12cm]{postit}
\begin{beamerboxesrounded}[shadow=true,lower=postit]{}
\pgfsetxvec{\pgfpoint{11cm}{0cm}}
\pgfsetyvec{\pgfpoint{0cm}{2.1cm}}
\begin{pgfpicture}{0cm}{0cm}{11cm}{2.1cm}

\only<1-5>{
\pgfputat{\pgfxy(0.5,1.0)}{\pgfbox[center,top]{\begin{minipage}{11cm}
\visible<1-5>{\textbf{Assertion d'intérêt}~: Le régime permet une perte de poids de 2 kilos par semaine $\Leftrightarrow$ $\mathbf{H_1}:\mu^D>4$ (en 2 semaines)}\\
\only<2-4>{\textbf{Décision} ({\small au vu des $\!n\!=\!50\!$ données})~: Accepter $\mathbf{H_1}$  si $\!{\color<2-4>{orange}\Est{\delta_{\mu^D,4}}{y^D}}\!>\!{\color<2-4>{darkgreen}\delta_{lim,\alpha}^+}\!$\\}
\only<2-4>{\textbf{Question}~: Conclure pour $\alpha=\only<2>{5}\only<3>{10}\only<4>{1}\%$~?}
\only<5>{\textbf{Question}~: Quel est le plus petit $\alpha$ (i.e. risque maximal de décider 
à tort $\mathbf{H_1}$) à encourir pour accepter $\mathbf{H_1}$ (i.e. l'assertion d'intérêt) au vu des $n=50$ données~?}\end{minipage}}}}
\only<6>{
\pgfputat{\pgfxy(0.5,1.0)}{\pgfbox[center,top]{\begin{minipage}{11cm}
\textbf{Assertion d'intérêt}~: Le régime permet une perte de poids de 2 kilos par semaine $\Leftrightarrow$ $\mathbf{H_1}:\mu^D>4$ (en 2 semaines)\\
\textbf{Réponse}~: \textbf{\color{orange}p$-$valeur}= \textbf{\color{orange}le} plus petit \textbf{\color{orange}risque} maximal de décider 
à tort $\mathbf{H_1}$ à encourir \textbf{\color{orange}pour accepter $\mathbf{H_1}$} (i.e. l'assertion d'intérêt).\end{minipage}}}}
\only<7>{
\pgfputat{\pgfxy(0.5,1.0)}{\pgfbox[center,top]{\begin{minipage}{11cm}
\textbf{Assertion d'intérêt}~: Le régime permet une perte de poids de 2 kilos par semaine $\Leftrightarrow$ $\mathbf{H_1}:\mu^D>4$ (en 2 semaines)\\
\textbf{Décision} (au vu des $\!n\!=\!50\!$ données)~: Accepter $\mathbf{H_1}$  si $\!{\color<7>{orange}\bf p-valeur}\!<\!{\color<7>{darkgreen}\alpha}\!$\\
i.e. si {\color{orange}\bf le risque pour accepter $\mathbf{H_1}$} est {\color{darkgreen}\bf raisonnablement petit}\end{minipage}}}}

\end{pgfpicture}

\end{beamerboxesrounded}
%\end{beamercolorbox}

\setbeamercolor{postex}{fg=black,bg=yellow!50!white}
%\begin{beamercolorbox}[sep=1em,wd=12cm]{postex}
\begin{beamerboxesrounded}[shadow=true,lower=postex]{}
\pgfsetxvec{\pgfpoint{11cm}{0cm}}
\pgfsetyvec{\pgfpoint{0cm}{5cm}}
\begin{pgfpicture}{0cm}{0cm}{11cm}{5cm}

\only<1>{
\pgfputat{\pgfxy(0.05,0.0)}{\pgfbox[left,bottom]{\pgfuseimage{diet}}}}
\only<2>{
\pgfputat{\pgfxy(0.05,0.0)}{\pgfbox[left,bottom]{\pgfuseimage{diet2}}}}
\only<3>{
\pgfputat{\pgfxy(0.05,0.0)}{\pgfbox[left,bottom]{\pgfuseimage{diet3}}}}
\only<4-5>{
\pgfputat{\pgfxy(0.05,0.0)}{\pgfbox[left,bottom]{\pgfuseimage{diet4}}}}
\only<1-5>{
\pgfputat{\pgfxy(0.5636363636363637,0.9536580194337094)}{\pgfbox[center,center]{
\textbf{Loi de $\Est{\delta_{\mu^D,4}}{Y^D}$ sous $\mathbf{H}_0:\mu^D=4\Leftrightarrow \delta_{\mu^D,4}=0$} }}}
\only<2-3>{
 
\pgfsetendarrow{\pgfarrowto}
\pgfline{\pgfxy(0.9065082797505997,0.5130431867558191)}{\pgfxy(0.8105041432386135,0.24189559741557892)}}
\only<4-5>{
 
\pgfsetendarrow{\pgfarrowto}
\pgfline{\pgfxy(0.9065082797505997,0.5130431867558191)}{\pgfxy(0.9065082797505997,0.18258206224740137)}}
\only<2>{
\pgfputat{\pgfxy(0.9065082797505997,0.5130431867558191)}{\pgfbox[center,bottom]{$\alpha=5\%$}}}
\only<3>{
\pgfputat{\pgfxy(0.9065082797505997,0.5130431867558191)}{\pgfbox[center,bottom]{$\alpha=10\%$}}}
\only<4-5>{
\pgfputat{\pgfxy(0.9065082797505997,0.5130431867558191)}{\pgfbox[center,bottom]{$\alpha=1\%$}}}
\only<2>{
\pgfputat{\pgfxy(0.5636363636363637,0.05548162974416376)}{\pgfbox[center,center]{
{\scriptsize {\color{orange}$\Est{\delta_{\mu^D,4}}{y^D}\stackrel{R}{=}$\texttt{(mean(yD)-4)/seMean(yD)=2.045}}$>${\color{darkgreen}$\delta_{lim,5\%}^+\stackrel{R}{=}$\texttt{qnorm(.95)=1.645}}}}}}
\only<3>{
\pgfputat{\pgfxy(0.5636363636363637,0.05548162974416376)}{\pgfbox[center,center]{
{\scriptsize {\color{orange}$\Est{\delta_{\mu^D,4}}{y^D}\stackrel{R}{=}$\texttt{(mean(yD)-4)/seMean(yD)=2.045}}$>${\color{darkgreen}$\delta_{lim,10\%}^+\stackrel{R}{=}$\texttt{qnorm(.9)=1.282}}}}}}
\only<4>{
\pgfputat{\pgfxy(0.5636363636363637,0.05548162974416376)}{\pgfbox[center,center]{
{\scriptsize {\color{orange}$\Est{\delta_{\mu^D,4}}{y^D}\stackrel{R}{=}$\texttt{(mean(yD)-4)/seMean(yD)=2.045}}$\ngtr${\color{darkgreen}$\delta_{lim,1\%}^+\stackrel{R}{=}$\texttt{qnorm(.99)=2.326}}}}}}
\only<6>{
\pgfputat{\pgfxy(0.05,0.0)}{\pgfbox[left,bottom]{\pgfuseimage{diet5}}}}
\only<6>{
\pgfputat{\pgfxy(0.05,0.0)}{\pgfbox[left,bottom]{\pgfuseimage{diet6}}}}
\only<6>{
\pgfputat{\pgfxy(0.5636363636363637,0.9536580194337094)}{\pgfbox[center,center]{
\textbf{Loi de $\Est{\delta_{\mu^D,4}}{Y^D}$ sous $\mathbf{H}_0:\mu^D=4\Leftrightarrow \delta_{\mu^D,4}=0$} }}}
\only<6>{
 
\pgfsetendarrow{\pgfarrowto}
\pgfline{\pgfxy(0.9065082797505997,0.5130431867558191)}{\pgfxy(0.8653636498168914,0.19105542441428391)}}
\only<6>{
\pgfputat{\pgfxy(0.9065082797505997,0.5130431867558191)}{\pgfbox[center,bottom]{p$-$valeur}}}
\only<6>{
\pgfputat{\pgfxy(0.5636363636363637,0.05548162974416376)}{\pgfbox[center,center]{
{\scriptsize {\color{orange}p$-$valeur(droite)$\stackrel{R}{=}$ \texttt{1-pnorm((mean(yD)-4)/seMean(yD))=2.04\%}  }}}}}
\only<7>{
\pgfputat{\pgfxy(0.05,0.0)}{\pgfbox[left,bottom]{\pgfuseimage{diet7}}}}
\only<7>{
\pgfputat{\pgfxy(0.5636363636363637,0.9536580194337094)}{\pgfbox[center,center]{\textbf{Loi de $\Est{\delta_{\mu^D,4}}{Y^D}$ sous $\mathbf{H}_0:\mu^D=4\Leftrightarrow \delta_{\mu^D,4}=0$} }}}
\only<7>{
 
\pgfsetendarrow{\pgfarrowto}
\pgfline{\pgfxy(0.9065082797505997,0.5130431867558191)}{\pgfxy(0.8653636498168914,0.19105542441428391)}
\pgfline{\pgfxy(0.8242190198831831,0.6825104300934691)}{\pgfxy(0.803646704916329,0.28426240824999144)}}
\only<7>{
\pgfputat{\pgfxy(0.9065082797505997,0.5130431867558191)}{\pgfbox[center,bottom]{p$-$valeur}}}
\only<7>{
\pgfputat{\pgfxy(0.8105041432386135,0.6825104300934691)}{\pgfbox[center,bottom]{$\alpha=5\%$}}}
\only<7>{
\pgfputat{\pgfxy(0.5636363636363637,0.05548162974416376)}{\pgfbox[center,center]{
{\scriptsize {\color{orange}p$-$valeur(droite)$\stackrel{R}{=}$ \texttt{1-pnorm((mean(yD)-4)/seMean(yD))=2.04\%}  $<${\color{darkgreen}$\alpha=5\%$}  }}}}}

\end{pgfpicture}

\end{beamerboxesrounded}
%\end{beamercolorbox}
\begin{tikzpicture}[remember picture,overlay]
  \node [rotate=30,scale=10,text opacity=0.05]
    at (current page.center) {CQLS};
\end{tikzpicture}
\end{frame}





\end{document}


