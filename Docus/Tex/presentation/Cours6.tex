
\documentclass[11pt]{beamer}
%Packages
\usepackage{multirow}
\usepackage{graphicx}
\usepackage[utf8x]{inputenc}
\usepackage{aeguill}
\usepackage{amssymb}
\usepackage[french]{babel}
\usepackage{pgf,pgfarrows,pgfnodes}
\usepackage{xmpmulti}
\usepackage{multimedia}
\usepackage{bbm}
\usepackage{bm}
\usepackage{mathrsfs,dsfont}
\usepackage{xcolor}
\usepackage{colortbl}
\usepackage{longtable}
\usepackage[T1]{fontenc}
\usepackage[scaled]{beramono}
\usepackage{float}
\usepackage{xkeyval,calc,listings,tikz}
\usepackage{slashbox}
\usepackage{fancyvrb}

%Preamble

\input Cours/cqlsInclude
\input Cours/testInclude


\mode<article>{\usepackage{fullpage}}
\usefonttheme{structureitalicserif}


\definecolor{VertFonce}{rgb}{0,.4,.0}

\mode<presentation>
{
  %\usetheme{Warsaw}
  % or ...
\usetheme{Boadilla}

  \setbeamercovered{transparent=5}
  % or whatever (possibly just delete it)
}
%\setbeamercovered{dynamic}


\subject{Talks}

\AtBeginSection[]
{
  
\begin{frame}<beamer>
	\frametitle{Plan}
    \tableofcontents[currentsection,currentsubsection]
\end{frame}

}


\definecolor{show}{rgb}{0.59,0.29,0.59}

\setbeamercovered{invisible}
\newcommand{\Sim}{{\star}}
\newcommand{\ok}{ \textcolor{green}{\large$\surd$}}
\newcommand{\nok}{ \textcolor{red}{\large X}}


\usetikzlibrary{arrows,%
  calc,%
  fit,%
  patterns,%
  plotmarks,%
  shapes.geometric,%
  shapes.misc,%
  shapes.symbols,%
  shapes.arrows,%
  shapes.callouts,%
  shapes.multipart,%
  shapes.gates.logic.US,%
  shapes.gates.logic.IEC,%
  er,%
  automata,%
  backgrounds,%
  chains,%
  topaths,%
  trees,%
  petri,%
  mindmap,%
  matrix,%
  calendar,%
  folding,%
  fadings,%
  through,%
  positioning,%
  scopes,%
  decorations.fractals,%
  decorations.shapes,%
  decorations.text,%
  decorations.pathmorphing,%
  decorations.pathreplacing,%
  decorations.footprints,%
  decorations.markings,%
  shadows}
\tikzset{
  every plot/.style={prefix=plots/pgf-},
  shape example/.style={
    color=black!30,
    draw,
    fill=yellow!30,
    line width=.5cm,
    inner xsep=2.5cm,
    inner ysep=0.5cm}
}

\definecolor{darkgreen}{rgb}{0,.4,.0}


\definecolor{darkgreen}{rgb}{0,.4,.0}

\definecolor{darkgreen}{rgb}{0,.4,.0}

%Styles

%Title

\beamertemplateshadingbackground{green!50}{yellow!50}
\title[Problématiques Produits A et B]
{Cours de Statistiques Inférentielles}
\author{CQLS~: cqls@upmf-grenoble.fr}
\date{\today}

\begin{document}
\maketitle


%\begin{frame}
%  \titlepage
%\end{frame}

\section{Rédaction standard}
\begin{frame}
\frametitle{Rédaction standard}

\begin{minipage}{11cm}

\centerline{\fbox{ \large \textit{  R{\'e}daction standard d'un test d'hypoth{\`e}ses param{\'e}trique  }} }\vspace*{.2cm}  

\fbox{
\begin{minipage}{11cm}
{\small
\noindent \textbf{Hypoth{\`e}ses de test~:} 
$$\mathbf{H_0}: \theta=\theta_0\mbox{ versus }\mathbf{H_1}:\left\{
\begin{array}{ll} 
\theta>\theta_0 & \mbox{\textbf{(a)}: {\it unilatéral droit}}\\ 
\theta<\theta_0 & \mbox{\textbf{(b)}: {\it unilatéral gauche}} \\ 
\theta\neq\theta_0 & \mbox{\textbf{(c)}: {\it bilatéral}}
\end{array}\right.
$$

\noindent \textbf{Statistique de test sous $\mathbf{H_0}$}~:
$$\Est{\delta_{\theta,\theta_0}}{Y}\leadsto\mathcal{L}_0 \mbox{ ({\`a} pr{\'e}ciser selon probl{\'e}matique)}$$ 

\noindent \textbf{R{\`e}gle de d{\'e}cision~:}
on accepte $\mathbf{H_1}$ si\\
$\left\{ 
\begin{array}{l} 
\mbox{\textbf{(a): }} 
\Est{\delta_{\theta,\theta_0}}{y}>\delta^{+}_{\lim,\alpha} \mbox{ ou p-valeur(droite)}<\alpha \\ 
\mbox{\textbf{(b): }} 
\Est{\delta_{\theta,\theta_0}}{y}<\delta^{-}_{\lim,\alpha}\mbox{ ou p-valeur(gauche)}<\alpha \\ 
 \mbox{\textbf{(c): }}  
\left( \Est{\delta_{\theta,\theta_0}} {y} < \delta^{-}_{\lim,\alpha/2} \mbox{ ou } \Est{\delta_{\theta,\theta_0}} {y} > \delta^{+}_{\lim,\alpha/2} \right)\mbox{ ou  p-valeur(bi)}<\alpha
\end{array}\right.$
}
\end{minipage}}\end{minipage}
\end{frame}

\begin{frame}
\frametitle{Un tableau récapitulatif pour les instructions \texttt{R}}
{\small \hspace*{-.3cm} \begin{tabular}{|c|c|c|}
\hline
 \multicolumn{3}{|c|}{\color{blue} \large Assertion d'intérêt} \\ 
$\mathbf{H_1}: \theta<\theta_0$ & $\mathbf{H_1}: \theta>\theta_0$&$\mathbf{H_1}: \theta\neq\theta_0$ \\
\hline \hline
\multicolumn{3}{|c|}{\color{blue} \large Statistique de test sous $H_0$} \\
\multicolumn{3}{|c|}{$\Est{\delta_{\theta,\theta_0}}{Y} \leadsto \mathcal{L}_0 \NotR \mathtt{{\color{purple}loi}({\color{purple}...})}$} \\
\hline \hline
\multicolumn{3}{|c|}{\color{blue} \large Le jour J avec les données $\Vect{y}$} \\ \multicolumn{3}{|c|}{$\Est{\delta_{\theta,\theta_0}}{y} \NotR \mathtt{deltaEst.H0}$} \\
\hline \hline
\multicolumn{3}{|c|}{\color{blue} \large Quantile(s)} \\
$\delta_{lim,\alpha}^- \NotR \mathtt{q{\color{purple}loi}(\alpha,{\color{purple}...})}$ &  $\delta_{lim,\alpha}^+ \NotR \mathtt{q{\color{purple}loi}(1-\alpha,{\color{purple}...})}$& $\delta_{lim,\frac{\alpha}2}^- \NotR \mathtt{q{\color{purple}loi}(\alpha,{\color{purple}...})}$\\
&& $\delta_{lim,\frac{\alpha}2}^+ \NotR \mathtt{q{\color{purple}loi}(1-\alpha/2,{\color{purple}...})}$\\
\hline \hline
\multicolumn{3}{|c|}{\color{blue} \large P$-$valeur}\\
p$-$val(gauche)$\NotR$&p$-$val(droite)$\NotR$ & p$-$val(bi)$\NotR$ \\
{\scriptsize$\mathtt{p{\color{purple}loi}(deltaEst.H0,{\color{purple}...})}$}&{\scriptsize $\mathtt{1-p{\color{purple}loi}(deltaEst.H0,{\color{purple}...})}$} & 
{\tiny$ \mathtt{2*p{\color{purple}loi}(deltaEst.H0,{\color{purple}...})}$ si p$-$val(g)$<$p$-$val(d)} \\
&& {\tiny$ \mathtt{2*(1-p{\color{purple}loi}(deltaEst.H0,{\color{purple}...}))}$ sinon}\\
\hline
\end{tabular}}
\end{frame}


\section{Problématiques avec 1 paramètre}
\begin{frame}

\begin{pgfpicture}{0cm}{0cm}{11cm}{8cm}
\pgfsetendarrow{\pgfarrowto}
\pgfnodebox{CritInt}[stroke]{\pgfxy(5.5,8)}{\color[rgb]{0,0,1}Nbre de paramètres pour décrire $\mathbf{H_1}$~?}{2pt}{2pt}
\pgfnodebox{Par1}[stroke]{\pgfxy(3.5,6.4)}{1 paramètre}{2pt}{2pt}
\pgfnodeconncurve{CritInt}{Par1}{-90}{90}{0.5cm}{0.5cm}
\pgfnodebox{Par2}[stroke]{\pgfxy(7.5,7.2)}{2 paramètres}{2pt}{2pt}
\pgfnodeconncurve{CritInt}{Par2}{-90}{90}{0.5cm}{0.5cm}

\end{pgfpicture}
\pgftext[top,right,at={\pgfxy(0,4.0)}]{\begin{minipage}{11cm}


\end{minipage}}

\end{frame}

\begin{frame}

\begin{pgfpicture}{0cm}{0cm}{11cm}{8cm}
\pgfsetendarrow{\pgfarrowto}
\pgfnodebox{CritInt}[stroke]{\pgfxy(5.5,8)}{\color[rgb]{0,0,1}Nbre de paramètres pour décrire $\mathbf{H_1}$~?}{2pt}{2pt}
\pgfnodebox{Par1}[stroke]{\pgfxy(3.5,6.4)}{1 paramètre}{2pt}{2pt}
\pgfnodeconncurve{CritInt}{Par1}{-90}{90}{0.5cm}{0.5cm}

\end{pgfpicture}
\pgftext[top,right,at={\pgfxy(0,4.0)}]{\begin{minipage}{11cm}


\end{minipage}}

\end{frame}

\begin{frame}

\begin{pgfpicture}{0cm}{0cm}{11cm}{8cm}
\pgfsetendarrow{\pgfarrowto}
\pgfnodebox{CritInt}[stroke]{\pgfxy(5.5,8)}{\color[rgb]{0,0,1}Nbre de paramètres pour décrire $\mathbf{H_1}$~?}{2pt}{2pt}
\pgfnodebox{Par1}[stroke]{\pgfxy(3.5,6.4)}{1 paramètre}{2pt}{2pt}
\pgfnodeconncurve{CritInt}{Par1}{-90}{90}{0.5cm}{0.5cm}

  \pgfnodebox{Asympt}[stroke]{\pgfxy(3,4.4)}{\color[rgb]{0,0.6,0}Asymptotique}{2pt}{2pt}
\pgfnodebox{Gauss}[stroke]{\pgfxy(8,4.4)}{\color[rgb]{0,0.6,0}Gaussien}{2pt}{2pt}
\pgfnodeconncurve{Par1}{Asympt}{-90}{90}{0.5cm}{0.5cm}
\pgfnodeconncurve{Par1}{Gauss}{-90}{90}{2cm}{0.5cm} 

\end{pgfpicture}
\pgftext[top,right,at={\pgfxy(0,4.0)}]{\begin{minipage}{11cm}


\end{minipage}}

\end{frame}

\begin{frame}

\begin{pgfpicture}{0cm}{0cm}{11cm}{8cm}
\pgfsetendarrow{\pgfarrowto}
\pgfnodebox{CritInt}[stroke]{\pgfxy(5.5,8)}{\color[rgb]{0,0,1}Nbre de paramètres pour décrire $\mathbf{H_1}$~?}{2pt}{2pt}
\pgfnodebox{Par1}[stroke]{\pgfxy(3.5,6.4)}{1 paramètre}{2pt}{2pt}
\pgfnodeconncurve{CritInt}{Par1}{-90}{90}{0.5cm}{0.5cm}

  \pgfnodebox{Asympt}[stroke]{\pgfxy(3,4.4)}{\color[rgb]{0,0.6,0}Asymptotique}{2pt}{2pt}
\pgfnodeconncurve{Par1}{Asympt}{-90}{90}{0.5cm}{0.5cm}

\end{pgfpicture}
\pgftext[top,right,at={\pgfxy(0,4.0)}]{\begin{minipage}{11cm}
\hrulefill\\
%%% 1 param et asympt
\noindent\textbf{Données~:} $\Vect{Y}=(Y_1,\cdots,Y_n)$ avec {\color{blue}$n\geq 30$} ($n$ grand).\\
\noindent \textbf{Statistique de test sous $\mathbf{H_0}$~:}\\
\centerline{\begin{tabular}{|c|c|}\hline
$p^\bullet$ & $\scriptsize\Est{\delta_{p^\bullet,p_0}}{Y}:=\frac{\Est{p^\bullet}{Y}-p_0}{\sqrt{\frac{p_0\times(1-p_0)}n}}\SuitApprox \mathcal{N}(0,1)$\\\hline
$\mu^\bullet$ &  $\scriptsize\Est{\delta_{\mu^\bullet,\mu_0}}{Y}:=\frac{\Est{\mu^\bullet}{Y}-\mu_0}{\Est{\sigma_{\widehat{\mu^\bullet}}}{Y}}\SuitApprox \mathcal{N}(0,1)$\\\hline
$\sigma^2_\bullet$ & $\scriptsize\Est{\delta_{\sigma^2_\bullet,\sigma^2_0}}{Y}:=\frac{\Est{\sigma^2_\bullet}{Y}-\sigma^2_0}{\Est{\sigma_{\widehat{\sigma^2_\bullet}}}{Y}}\SuitApprox \mathcal{N}(0,1)$\\\hline
\end{tabular}}


\end{minipage}}

\end{frame}

\begin{frame}

\begin{pgfpicture}{0cm}{0cm}{11cm}{8cm}
\pgfsetendarrow{\pgfarrowto}
\pgfnodebox{CritInt}[stroke]{\pgfxy(5.5,8)}{\color[rgb]{0,0,1}Nbre de paramètres pour décrire $\mathbf{H_1}$~?}{2pt}{2pt}
\pgfnodebox{Par1}[stroke]{\pgfxy(3.5,6.4)}{1 paramètre}{2pt}{2pt}
\pgfnodeconncurve{CritInt}{Par1}{-90}{90}{0.5cm}{0.5cm}
\pgfnodebox{Gauss}[stroke]{\pgfxy(8,4.4)}{\color[rgb]{0,0.6,0}Gaussien}{2pt}{2pt}
\pgfnodeconncurve{Par1}{Gauss}{-90}{90}{2cm}{0.5cm} 

\end{pgfpicture}
\pgftext[top,right,at={\pgfxy(0,4.0)}]{\begin{minipage}{11cm}
\hrulefill\\
%%% 1 param et gaussien 
\noindent\textbf{Données~:} $\Vect{Y}=(Y_1,\cdots,Y_n)$ avec $Y{\color{blue}\leadsto \mathcal{N}(\mu^\bullet,\sigma_\bullet)}$\\
\noindent\textbf{Statistique de test sous $\mathbf{H_0}$~:}\\ 
\centerline{\begin{tabular}{|c|c|}\hline
$\mu^\bullet$ &  $\scriptsize\Est{\delta_{\mu^\bullet,\mu_0}}{Y}:=\frac{\Est{\mu^\bullet}{Y}-\mu_0}{\Est{\sigma_{\widehat{\mu^\bullet}}}{Y}}\leadsto \mathcal{S}t(n-1)$\\\hline
$\sigma^2_\bullet$ & $\scriptsize\Est{\delta_{\sigma^2_\bullet,\sigma^2_0}}{Y}:=(n-1)\times\frac{\Est{\sigma^2_\bullet}{Y}}{\sigma^2_0}\leadsto \chi^2(n-1)$\\\hline
\end{tabular}}



\end{minipage}}

\end{frame}

\section{P-valeur (suite)}
\pgfdeclareimage[width=11cm,height=5cm,interpolate=true]{alf2}{img/alf2}

\pgfdeclareimage[width=11cm,height=5cm,interpolate=true]{alf7}{img/alf7}

\pgfdeclareimage[width=11cm,height=5cm,interpolate=true]{alf8}{img/alf8}









\pgfdeclareimage[width=11cm,height=5cm,interpolate=true]{dict}{img/dict}

\pgfdeclareimage[width=11cm,height=5cm,interpolate=true]{dict2}{img/dict2}






%\beamertemplateshadingbackground{green!50}{yellow!50}
\begin{frame}<1->
\setbeamercolor{header}{fg=black,bg=blue!40!white}
 \hspace*{2.5cm}\begin{beamerboxesrounded}[width=6cm,shadow=true,lower=header]{}
  \pgfsetxvec{\pgfpoint{6cm}{0cm}}
\pgfsetyvec{\pgfpoint{0cm}{0.5cm}}
\begin{pgfpicture}{0cm}{0cm}{6cm}{0.5cm}

  \only<1-4>{
\pgfputat{\pgfxy(0.5,0.5)}{\pgfbox[center,center]{\textbf{\large Exemple compétence Alfred}}}}
\only<5-6>{
\pgfputat{\pgfxy(0.5,0.5)}{\pgfbox[center,center]{\textbf{\large Problématique de la dictée}}}}

  \end{pgfpicture}

\end{beamerboxesrounded}

\setbeamercolor{postit}{fg=black,bg=green!40!white}
%\begin{beamercolorbox}[sep=1em,wd=12cm]{postit}
\begin{beamerboxesrounded}[shadow=true,lower=postit]{}
\pgfsetxvec{\pgfpoint{11cm}{0cm}}
\pgfsetyvec{\pgfpoint{0cm}{2.1cm}}
\begin{pgfpicture}{0cm}{0cm}{11cm}{2.1cm}

\only<1-2>{
\pgfputat{\pgfxy(0.5,1.0)}{\pgfbox[center,top]{\begin{minipage}{11cm}
\textbf{Assertion d'intérêt}~: Alfred est compétent $\Leftrightarrow$ $\mathbf{H_1}:\sigma_A^2<0.1$.\\
\textbf{Décision} (au vu des $n=20$ données)~: \\
\hspace*{1cm}Accepter $\mathbf{H_1}$  si
\only<1>{${\color<1>{orange}\Est{\delta_{\sigma_A^2,0.1}}{y^A}}<{\color<1>{darkgreen}\delta_{lim,\alpha}^-}$}
\only<2>{${\color<2>{orange}p-valeur(gauche)}<{\color<2>{darkgreen}\alpha}$}\end{minipage}}}}
\only<3-4>{
\pgfputat{\pgfxy(0.5,1.0)}{\pgfbox[center,top]{\begin{minipage}{11cm}
\textbf{Question}~: Peut-on pour autant plutôt penser au vu de ce même jeu de données qu'Alfred n'est pas compétent (i.e. $\mathbf{H_1}:\sigma^2_A>0.1$)~?\\
\textbf{Réponse}~: p-valeur droite = \only<3>{?}\only<4>{1-(p-valeur gauche)=1-21.16\%=78.84\%}\\
\visible<4>{car \textbf{la somme des p-valeurs droite et gauche est égale à 1!}}\end{minipage}}}}
\only<5-6>{
\pgfputat{\pgfxy(0.5,1.0)}{\pgfbox[center,top]{\begin{minipage}{11.3cm}
\textbf{Assertion d'intérêt}~: Il y a un effet sur le niveau des bacheliers en orthographe $\Leftrightarrow$ $\mathbf{H_1}:\mu^D \neq 6.3$.\\
\textbf{Décision} (au vu des $n=25$ données)~: \\
Accepter $\mathbf{H_1}$  si
{\small
\only<5>{${\color<5>{orange}\Est{\delta_{\mu^D,6.3}}{y^D}}<{\color<5>{darkgreen}\delta_{lim,\alpha/2}^-}$ ou ${\color<5>{orange}\Est{\delta_{\mu^D,6.3}}{y^D}}>{\color<5>{darkgreen}\delta_{lim,\alpha/2}^+}$ }
\only<6>{${\color<6>{orange}\mbox{p-valeur (bi)}\!=\!2\!\times\!\mbox{min(p-valeur gauche,p-valeur droite)}}\!<\!{\color<6>{darkgreen}\alpha}$}
}\end{minipage}}}}

\end{pgfpicture}

\end{beamerboxesrounded}
%\end{beamercolorbox}

\setbeamercolor{postex}{fg=black,bg=yellow!50!white}
%\begin{beamercolorbox}[sep=1em,wd=12cm]{postex}
\begin{beamerboxesrounded}[shadow=true,lower=postex]{}
\pgfsetxvec{\pgfpoint{11cm}{0cm}}
\pgfsetyvec{\pgfpoint{0cm}{5cm}}
\begin{pgfpicture}{0cm}{0cm}{11cm}{5cm}

\only<1>{
\pgfputat{\pgfxy(0.05,0.0)}{\pgfbox[left,bottom]{\pgfuseimage{alf2}}}}
\only<2-3>{
\pgfputat{\pgfxy(0.05,0.0)}{\pgfbox[left,bottom]{\pgfuseimage{alf7}}}}
\only<4>{
\pgfputat{\pgfxy(0.05,0.0)}{\pgfbox[left,bottom]{\pgfuseimage{alf8}}}}
\only<1-4>{
\pgfputat{\pgfxy(0.5721648758217379,0.9714797908880797)}{\pgfbox[center,center]{
\textbf{Loi de $\Est{\delta_{\sigma^2_A,0.1}}{Y^A}$ sous $\mathbf{H}_0:\sigma_A^2=0.1\Leftrightarrow \delta_{\sigma_A^2,0.1}=19$} }}}
\only<1-3>{
\pgfputat{\pgfxy(0.5721648758217379,0.05611716230336712)}{\pgfbox[center,center]{
{\scriptsize \only<1>{{\color{orange}$\Est{\delta_{\sigma^2_A,0.1}}{y^A}\stackrel{R}{=}$\texttt{19*var(yA)/0.1=13.919}}$\nless${\color{darkgreen}$\delta_{lim,5\%}^-\stackrel{R}{=}$\texttt{qchisq(.05,19)=10.117}}}
\only<2-3>{{\color{orange}$p-valeur (gauche)\stackrel{R}{=}$\texttt{pchisq(19*var(yA)/0.1,19)=21.16\%}}$\nless${\color{darkgreen}$\alpha=5\%$}}
}}}}
\only<4>{
\pgfputat{\pgfxy(0.5721648758217379,0.05611716230336712)}{\pgfbox[center,center]{
{\scriptsize  {\color{orange}$p-valeur (droite)\stackrel{R}{=}$\texttt{1-pchisq(19*var(yA)/0.1,19)=78.84\%}}$\nless${\color{darkgreen}$\alpha=5\%$}
}}}}
\only<5>{
\pgfputat{\pgfxy(0.05,0.0)}{\pgfbox[left,bottom]{\pgfuseimage{dict}}}}
\only<6>{
\pgfputat{\pgfxy(0.05,0.0)}{\pgfbox[left,bottom]{\pgfuseimage{dict2}}}}
\only<5-6>{
\pgfputat{\pgfxy(0.5636363636363636,0.9615726728634971)}{\pgfbox[center,center]{
\textbf{Loi de $\Est{\delta_{\mu^D,6.3}}{Y^D}$ sous $\mathbf{H}_0:\mu^D=6.3\Leftrightarrow \delta_{\mu^D,6.3}=0$} }}}
\only<5-6>{
\pgfputat{\pgfxy(0.5636363636363636,0.07307827151614543)}{\pgfbox[center,center]{
{\scriptsize \only<5>{{\color{orange}$\Est{\delta_{\mu^D,6.3}}{y^D}\stackrel{R}{=}$\texttt{(mean(yD)-6.3)/seMean(yD)=-1.985}}
$\left\{\begin{array}{l}
\nless{\color{darkgreen}\delta_{lim,2.5\%}^-\simeq -2.063899}\\
\ngtr{\color{darkgreen}\delta_{lim,2.5\%}^+\simeq 2.063899}
\end{array}\right.$
}
\only<6>{{\color{orange}$p-valeur (bi)\stackrel{R}{=}$ \texttt{2*pt((mean(yD)-6.3)/seMean(yD),24) =5.87\%}}$\nless${\color{darkgreen}$\alpha=5\%$}}
}}}}

\end{pgfpicture}

\end{beamerboxesrounded}
%\end{beamercolorbox}
\begin{tikzpicture}[remember picture,overlay]
  \node [rotate=30,scale=10,text opacity=0.05]
    at (current page.center) {CQLS};
\end{tikzpicture}
\end{frame}

\section{Exercices (1 échantillon)}
\begin{frame}

\begin{pgfpicture}{0cm}{0cm}{11cm}{8cm}
\pgfsetendarrow{\pgfarrowto}
\pgfnodebox{CritInt}[stroke]{\pgfxy(5.5,8)}{\color[rgb]{0,0,1}Nbre de paramètres pour décrire $\mathbf{H_1}$~?}{2pt}{2pt}
\pgfnodebox{Par1}[stroke]{\pgfxy(3.5,6.4)}{1 paramètre}{2pt}{2pt}
\pgfnodeconncurve{CritInt}{Par1}{-90}{90}{0.5cm}{0.5cm}

  \pgfnodebox{Asympt}[stroke]{\pgfxy(3,4.4)}{\color[rgb]{0,0.6,0}Asymptotique}{2pt}{2pt}
\pgfnodeconncurve{Par1}{Asympt}{-90}{90}{0.5cm}{0.5cm}

\end{pgfpicture}
\pgftext[top,right,at={\pgfxy(0,4.0)}]{\begin{minipage}{11cm}
\hrulefill\\
%%% 1 param et asympt
\noindent\textbf{Données~:} $\Vect{Y}=(Y_1,\cdots,Y_n)$ avec {\color{blue}$n\geq 30$} ($n$ grand).\\
\noindent \textbf{Statistique de test sous $\mathbf{H_0}$~:}\\
\centerline{\begin{tabular}{|c|c|}\hline
$p^\bullet$ & $\scriptsize\Est{\delta_{p^\bullet,p_0}}{Y}:=\frac{\Est{p^\bullet}{Y}-p_0}{\sqrt{\frac{p_0\times(1-p_0)}n}}\SuitApprox \mathcal{N}(0,1)$\\\hline
$\mu^\bullet$ &  $\scriptsize\Est{\delta_{\mu^\bullet,\mu_0}}{Y}:=\frac{\Est{\mu^\bullet}{Y}-\mu_0}{\Est{\sigma_{\widehat{\mu^\bullet}}}{Y}}\SuitApprox \mathcal{N}(0,1)$\\\hline
$\sigma^2_\bullet$ & $\scriptsize\Est{\delta_{\sigma^2_\bullet,\sigma^2_0}}{Y}:=\frac{\Est{\sigma^2_\bullet}{Y}-\sigma^2_0}{\Est{\sigma_{\widehat{\sigma^2_\bullet}}}{Y}}\SuitApprox \mathcal{N}(0,1)$\\\hline
\end{tabular}}


\end{minipage}}

\end{frame}

\pgfdeclareimage[width=8cm,height=4cm,interpolate=true]{pvalmuDietn50sq}{img/pvalmuDietn50sq}

\pgfdeclareimage[width=8cm,height=4cm,interpolate=true]{pvalmuDietn50sp}{img/pvalmuDietn50sp}




\pgfdeclareimage[width=8cm,height=4cm,interpolate=true]{pvalalfredn50sq}{img/pvalalfredn50sq}

\pgfdeclareimage[width=8cm,height=4cm,interpolate=true]{pvalalfredn50sp}{img/pvalalfredn50sp}




%\beamertemplateshadingbackground{green!50}{yellow!50}
\begin{frame}<1->
\setbeamercolor{header}{fg=black,bg=blue!40!white}
 \hspace*{2.5cm}\begin{beamerboxesrounded}[width=6cm,shadow=true,lower=header]{}
  \pgfsetxvec{\pgfpoint{6cm}{0cm}}
\pgfsetyvec{\pgfpoint{0cm}{0.5cm}}
\begin{pgfpicture}{0cm}{0cm}{6cm}{0.5cm}

  \only<1-1
>{
\pgfputat{\pgfxy(0.5,0.5)}{\pgfbox[center,center]{\textbf{\large Chomage (abr. quant)}}}}
\only<2-2
>{
\pgfputat{\pgfxy(0.5,0.5)}{\pgfbox[center,center]{\textbf{\large Chomage (abr. p-val)}}}}
\only<3-13>{
\pgfputat{\pgfxy(0.5,0.5)}{\pgfbox[center,center]{\textbf{\large Diététicien (quantile)}}}}
\only<14-24>{
\pgfputat{\pgfxy(0.5,0.5)}{\pgfbox[center,center]{\textbf{\large Diététicien (p-valeur)}}}}
\only<25-25
>{
\pgfputat{\pgfxy(0.5,0.5)}{\pgfbox[center,center]{\textbf{\large Diététicien (abr. quant)}}}}
\only<26-26
>{
\pgfputat{\pgfxy(0.5,0.5)}{\pgfbox[center,center]{\textbf{\large Diététicien (abr. p-val)}}}}
\only<27-37>{
\pgfputat{\pgfxy(0.5,0.5)}{\pgfbox[center,center]{\textbf{\large Alfred (quantile)}}}}
\only<38-48>{
\pgfputat{\pgfxy(0.5,0.5)}{\pgfbox[center,center]{\textbf{\large Alfred (p-valeur)}}}}
\only<49-49
>{
\pgfputat{\pgfxy(0.5,0.5)}{\pgfbox[center,center]{\textbf{\large Alfred (abr. quant)}}}}
\only<50-50
>{
\pgfputat{\pgfxy(0.5,0.5)}{\pgfbox[center,center]{\textbf{\large Alfred (abr. p-val)}}}}

  \end{pgfpicture}

\end{beamerboxesrounded}

\setbeamercolor{postit}{fg=black,bg=green!40!white}
%\begin{beamercolorbox}[sep=1em,wd=12cm]{postit}
\begin{beamerboxesrounded}[shadow=true,lower=postit]{}
\pgfsetxvec{\pgfpoint{11cm}{0cm}}
\pgfsetyvec{\pgfpoint{0cm}{2.1cm}}
\begin{pgfpicture}{0cm}{0cm}{11cm}{2.1cm}

\only<1-1
>{
\pgfputat{\pgfxy(0.5,0.5)}{\pgfbox[center,center]{\begin{minipage}{11cm}
\textbf{Question } Peut-on penser que le taux de chômage en France serait cette année inférieur à $10\%$
au vu des données \texttt{yC} en \texttt{R}~?\\
\textbf{Indic \texttt{R}}~: \texttt{deltaEst.H0}$\simeq$\texttt{-0.942809}\\\phantom{\textbf{Indic \texttt{R}}~: }\texttt{qnorm(0.95)}$\simeq$ \texttt{1.644854} 

\end{minipage}}}}
\only<2-2
>{
\pgfputat{\pgfxy(0.5,0.5)}{\pgfbox[center,center]{\begin{minipage}{11cm}
\textbf{Question } Peut-on penser que le taux de chômage en France serait cette année inférieur à $10\%$
au vu des données \texttt{yC} en \texttt{R}~?\\
\textbf{Indic \texttt{R}}~: \texttt{pnorm(deltaEst.H0)}$\simeq$ \texttt{0.1728893}
\end{minipage}}}}
\only<3-4>{
\pgfputat{\pgfxy(0.5,1.0)}{\pgfbox[center,top]{\begin{minipage}{11cm}\textbf{Question } Comment s'écrit l'assertion d'intérêt $\mathbf{H_1}$ en fonction des paramètres d'intérêt et d'écart~?\only<3>{\\\textbf{Assertion d'intérêt}~: le régime alimentaire du diététicien permet une perte de poids de 2 kilos par semaine}\only<4>{\\ \centerline{$\mathbf{H_1}$:$\mu^{D}>4$ $\Longleftrightarrow$ $\delta_{\mu^{D},4}:={\displaystyle \frac{\mu^{D}-4}{
\sigma_{\cqlshat{\mu^{D}}}
}} >0$}}
\end{minipage}}}}
\only<5-6>{
\pgfputat{\pgfxy(0.5,1.0)}{\pgfbox[center,top]{\begin{minipage}{11cm}\textbf{Question }: Quelle est la pire des situations, i.e. parmi toutes les situations quelle est celle qui engendre le plus grand risque d'erreur de première espèce~?\end{minipage}}}}
\only<7-8>{
\pgfputat{\pgfxy(0.5,1.0)}{\pgfbox[center,top]{\begin{minipage}{11cm}\textbf{Question }: Quelle est l'information du mathématicien quant au comportement de $\Est{\delta_{\mu^{D},4}}{Y^{D}}$ dans la pire des situations~?\end{minipage}}}}
\only<9-10>{
\pgfputat{\pgfxy(0.5,1.0)}{\pgfbox[center,top]{\begin{minipage}{11cm}\textbf{Question }: Comment s'écrit la règle de décision ne produisant pas plus de 5\% d'erreur de première espèce~? \\
\textbf{Indic \texttt{R}}~: \texttt{deltaEst.H0}$\simeq$\texttt{2.044722}\\\phantom{\textbf{Indic \texttt{R}}~: }\texttt{qnorm(0.95)}$\simeq$ \texttt{1.644854}
\end{minipage}}}}
\only<11-13>{
\pgfputat{\pgfxy(0.5,1.0)}{\pgfbox[center,top]{\begin{minipage}{11cm}\textbf{Question }: Comment conclueriez-vous au vu des données \texttt{yD} en \texttt{R}~?\\
\textbf{Indic \texttt{R}}~: \texttt{deltaEst.H0}$\simeq$\texttt{2.044722}\\\phantom{\textbf{Indic \texttt{R}}~: }\texttt{qnorm(0.95)}$\simeq$ \texttt{1.644854} 
\end{minipage}}}}
\only<14-15>{
\pgfputat{\pgfxy(0.5,1.0)}{\pgfbox[center,top]{\begin{minipage}{11cm}\textbf{Question } Comment s'écrit l'assertion d'intérêt $\mathbf{H_1}$ en fonction des paramètres d'intérêt et d'écart~?\only<14>{\\\textbf{Assertion d'intérêt}~: le régime alimentaire du diététicien permet une perte de poids de 2 kilos par semaine}\only<15>{\\ \centerline{$\mathbf{H_1}$:$\mu^{D}>4$ $\Longleftrightarrow$ $\delta_{\mu^{D},4}:={\displaystyle \frac{\mu^{D}-4}{
\sigma_{\cqlshat{\mu^{D}}}
}} >0$}}
\end{minipage}}}}
\only<16-17>{
\pgfputat{\pgfxy(0.5,1.0)}{\pgfbox[center,top]{\begin{minipage}{11cm}\textbf{Question }: Quelle est la pire des situations, i.e. parmi toutes les situations quelle est celle qui engendre le plus grand risque d'erreur de première espèce~?\end{minipage}}}}
\only<18-19>{
\pgfputat{\pgfxy(0.5,1.0)}{\pgfbox[center,top]{\begin{minipage}{11cm}\textbf{Question }: Quelle est l'information du mathématicien quant au comportement de $\Est{\delta_{\mu^{D},4}}{Y^{D}}$ dans la pire des situations~?\end{minipage}}}}
\only<20-21>{
\pgfputat{\pgfxy(0.5,1.0)}{\pgfbox[center,top]{\begin{minipage}{11cm}\textbf{Question }: Comment s'écrit la règle de décision ne produisant pas plus de 5\% d'erreur de première espèce~? \\
\textbf{Indic \texttt{R}}~: \texttt{pnorm(deltaEst.H0)}$\simeq$ \texttt{0.9795589}\end{minipage}}}}
\only<22-24>{
\pgfputat{\pgfxy(0.5,1.0)}{\pgfbox[center,top]{\begin{minipage}{11cm}\textbf{Question }: Comment conclueriez-vous au vu des données \texttt{yD} en \texttt{R}~?\\
\textbf{Indic \texttt{R}}~: \texttt{pnorm(deltaEst.H0)}$\simeq$ \texttt{0.9795589}\end{minipage}}}}
\only<25-25
>{
\pgfputat{\pgfxy(0.5,0.5)}{\pgfbox[center,center]{\begin{minipage}{11cm}
\textbf{Question } Peut-on penser que le régime alimentaire du diététicien permet une perte de poids de 2 kilos par semaine
au vu des données \texttt{yD} en \texttt{R}~?\\
\textbf{Indic \texttt{R}}~: \texttt{deltaEst.H0}$\simeq$\texttt{2.044722}\\\phantom{\textbf{Indic \texttt{R}}~: }\texttt{qnorm(0.95)}$\simeq$ \texttt{1.644854} 

\end{minipage}}}}
\only<26-26
>{
\pgfputat{\pgfxy(0.5,0.5)}{\pgfbox[center,center]{\begin{minipage}{11cm}
\textbf{Question } Peut-on penser que le régime alimentaire du diététicien permet une perte de poids de 2 kilos par semaine
au vu des données \texttt{yD} en \texttt{R}~?\\
\textbf{Indic \texttt{R}}~: \texttt{pnorm(deltaEst.H0)}$\simeq$ \texttt{0.9795589}
\end{minipage}}}}
\only<27-28>{
\pgfputat{\pgfxy(0.5,1.0)}{\pgfbox[center,top]{\begin{minipage}{11cm}\textbf{Question } Comment s'écrit l'assertion d'intérêt $\mathbf{H_1}$ en fonction des paramètres d'intérêt et d'écart~?\only<27>{\\\textbf{Assertion d'intérêt}~: Alfred est compétent}\only<28>{\\ \centerline{$\mathbf{H_1}$:$\sigma^2_{A}<0.1$ $\Longleftrightarrow$ $\delta_{\sigma^2_{A},0.1}:={\displaystyle \frac{\sigma^2_{A}-0.1}{
\sigma_{\cqlshat{\sigma^2_{A}}}
}} <0$}}
\end{minipage}}}}
\only<29-30>{
\pgfputat{\pgfxy(0.5,1.0)}{\pgfbox[center,top]{\begin{minipage}{11cm}\textbf{Question }: Quelle est la pire des situations, i.e. parmi toutes les situations quelle est celle qui engendre le plus grand risque d'erreur de première espèce~?\end{minipage}}}}
\only<31-32>{
\pgfputat{\pgfxy(0.5,1.0)}{\pgfbox[center,top]{\begin{minipage}{11cm}\textbf{Question }: Quelle est l'information du mathématicien quant au comportement de $\Est{\delta_{\sigma^2_{A},0.1}}{Y^{A}}$ dans la pire des situations~?\end{minipage}}}}
\only<33-34>{
\pgfputat{\pgfxy(0.5,1.0)}{\pgfbox[center,top]{\begin{minipage}{11cm}\textbf{Question }: Comment s'écrit la règle de décision ne produisant pas plus de 5\% d'erreur de première espèce~? \\
\textbf{Indic \texttt{R}}~: \texttt{deltaEst.H0}$\simeq$\texttt{-2.762438}\\\phantom{\textbf{Indic \texttt{R}}~: }\texttt{qnorm(0.95)}$\simeq$ \texttt{1.644854}
\end{minipage}}}}
\only<35-37>{
\pgfputat{\pgfxy(0.5,1.0)}{\pgfbox[center,top]{\begin{minipage}{11cm}\textbf{Question }: Comment conclueriez-vous au vu des données \texttt{yA} en \texttt{R}~?\\
\textbf{Indic \texttt{R}}~: \texttt{deltaEst.H0}$\simeq$\texttt{-2.762438}\\\phantom{\textbf{Indic \texttt{R}}~: }\texttt{qnorm(0.95)}$\simeq$ \texttt{1.644854} 
\end{minipage}}}}
\only<38-39>{
\pgfputat{\pgfxy(0.5,1.0)}{\pgfbox[center,top]{\begin{minipage}{11cm}\textbf{Question } Comment s'écrit l'assertion d'intérêt $\mathbf{H_1}$ en fonction des paramètres d'intérêt et d'écart~?\only<38>{\\\textbf{Assertion d'intérêt}~: Alfred est compétent}\only<39>{\\ \centerline{$\mathbf{H_1}$:$\sigma^2_{A}<0.1$ $\Longleftrightarrow$ $\delta_{\sigma^2_{A},0.1}:={\displaystyle \frac{\sigma^2_{A}-0.1}{
\sigma_{\cqlshat{\sigma^2_{A}}}
}} <0$}}
\end{minipage}}}}
\only<40-41>{
\pgfputat{\pgfxy(0.5,1.0)}{\pgfbox[center,top]{\begin{minipage}{11cm}\textbf{Question }: Quelle est la pire des situations, i.e. parmi toutes les situations quelle est celle qui engendre le plus grand risque d'erreur de première espèce~?\end{minipage}}}}
\only<42-43>{
\pgfputat{\pgfxy(0.5,1.0)}{\pgfbox[center,top]{\begin{minipage}{11cm}\textbf{Question }: Quelle est l'information du mathématicien quant au comportement de $\Est{\delta_{\sigma^2_{A},0.1}}{Y^{A}}$ dans la pire des situations~?\end{minipage}}}}
\only<44-45>{
\pgfputat{\pgfxy(0.5,1.0)}{\pgfbox[center,top]{\begin{minipage}{11cm}\textbf{Question }: Comment s'écrit la règle de décision ne produisant pas plus de 5\% d'erreur de première espèce~? \\
\textbf{Indic \texttt{R}}~: \texttt{pnorm(deltaEst.H0)}$\simeq$ \texttt{0.002868572}\end{minipage}}}}
\only<46-48>{
\pgfputat{\pgfxy(0.5,1.0)}{\pgfbox[center,top]{\begin{minipage}{11cm}\textbf{Question }: Comment conclueriez-vous au vu des données \texttt{yA} en \texttt{R}~?\\
\textbf{Indic \texttt{R}}~: \texttt{pnorm(deltaEst.H0)}$\simeq$ \texttt{0.002868572}\end{minipage}}}}
\only<49-49
>{
\pgfputat{\pgfxy(0.5,0.5)}{\pgfbox[center,center]{\begin{minipage}{11cm}
\textbf{Question } Peut-on penser que Alfred est compétent
au vu des données \texttt{yA} en \texttt{R}~?\\
\textbf{Indic \texttt{R}}~: \texttt{deltaEst.H0}$\simeq$\texttt{-2.762438}\\\phantom{\textbf{Indic \texttt{R}}~: }\texttt{qnorm(0.95)}$\simeq$ \texttt{1.644854} 

\end{minipage}}}}
\only<50-50
>{
\pgfputat{\pgfxy(0.5,0.5)}{\pgfbox[center,center]{\begin{minipage}{11cm}
\textbf{Question } Peut-on penser que Alfred est compétent
au vu des données \texttt{yA} en \texttt{R}~?\\
\textbf{Indic \texttt{R}}~: \texttt{pnorm(deltaEst.H0)}$\simeq$ \texttt{0.002868572}
\end{minipage}}}}

\end{pgfpicture}

\end{beamerboxesrounded}
%\end{beamercolorbox}

\setbeamercolor{postex}{fg=black,bg=yellow!50!white}
%\begin{beamercolorbox}[sep=1em,wd=12cm]{postex}
\begin{beamerboxesrounded}[shadow=true,lower=postex]{}
\pgfsetxvec{\pgfpoint{11cm}{0cm}}
\pgfsetyvec{\pgfpoint{0cm}{5cm}}
\begin{pgfpicture}{0cm}{0cm}{11cm}{5cm}

\only<1-1
>{
\pgfputat{\pgfxy(0.5,1.0)}{\pgfbox[center,top]{\begin{minipage}{11cm}
{\small
\noindent \textbf{Assertion d'intérêt} :  $\mathbf{H}_1:$ $p^{C}<10\%$ \\
\textbf{Application numérique} :  puisqu'au vu des données, 
  \begin{eqnarray*}
\Est{\delta_{p^{C},10\%}}{y^{C}} &\NotR&\mathtt{(16/200-0.1)/sqrt(0.1*(1-0.1)/200)}\simeq -0.942809\\&\nless & \delta^-_{lim,5\%} \NotR \mathtt{-qnorm(1-.05)}\simeq-1.644854
\end{eqnarray*}
  
on ne peut pas plutôt penser (avec un risque de 5\%) que le taux de chômage en France serait cette année inférieur à $10\%$. }
\end{minipage}}}}
\only<2-2
>{
\pgfputat{\pgfxy(0.5,1.0)}{\pgfbox[center,top]{\begin{minipage}{11cm}
{\small
\noindent \textbf{Assertion d'intérêt} :  $\mathbf{H}_1:$ $p^{C}<10\%$ \\
\textbf{Application numérique} :  puisqu'au vu des données, 
  \[
p-valeur\NotR\mathtt{pnorm((16/200-0.1)/sqrt(0.1*(1-0.1)/200))} \simeq 17.29\%\nless5\%,
\]
on ne peut pas plutôt penser (avec un risque de 5\%) que le taux de chômage en France serait cette année inférieur à $10\%$. }
\end{minipage}}}}
\only<3-13>{
\pgfputat{\pgfxy(0.5,1.0)}{\pgfbox[center,top]{\begin{minipage}{11cm}\only<3>{\textbf{Indications \texttt{R}}~:\\\texttt{  > length(yD)\\
$[1]$ 50\\
> mean(yD)\\
$[1]$ 4.5\\
  > deltaEst.H0 \# instruction R à fournir\\
$[1]$ 2.044722\\
> qnorm(0.95)\\
$[1]$ 1.644854}}\visible<4-13>{\noindent\textbf{Hypothèses de test~:}}\visible<6-13>{ {\small $\mathbf{H}_0:$ $\mu^{D}=4$} vs } \visible<4-13>{$\mathbf{H}_1:$ $\mu^{D}>4$}\\\only<4>{\\}\visible<8-13>{\noindent\textbf{Statistique de test sous $\mathbf{H_0}$~:}\\
\centerline{$\Est{\delta_{\mu^{D},4}}{Y^{D}}= {\displaystyle \frac{\Est{\mu^{D}}{Y^{D}}-4}{
\Est{\sigma_{\cqlshat{\mu^{D}}}}{Y^{D}}
}} 
  \SuitApprox\mathcal{N}(0,1) $}\newline}
\visible<10-13>{\noindent\textbf{Règle de Décision}~:\\
\centerline{Accepter $\mathbf{H_1}$ si   ${\color<12>{orange}\Est{\delta_{\mu^{D},4}}{y^{D}}} > {\color<12>{darkgreen}\delta^+_{lim,5\%}}$}\newline}
\visible<12-13>{\small\noindent\textbf{Conclusion}~: puisqu'au vu des données,
\\$\Est{\delta_{\mu^{D},4}}{y^{D}}\!\NotR\! {\color<12>{orange}\mbox{\scriptsize\texttt{(mean(yD)-4)/seMean(yD)}}\!\simeq\! 2.045}$\\\phantom{$\!\Est{\delta_{\mu^{D},4}}{y^{D}}\!$}$\!> \delta^+_{lim,5\%}\!
\NotR\!{\color<12>{darkgreen}\mathtt{qnorm(1-.05)\!\simeq\!1.645}}\!$\\
 on peut plutôt penser (avec un risque de 5\%) que le régime alimentaire du diététicien permet une perte de poids de 2 kilos par semaine.}\end{minipage}}}}
\only<12>{
\pgfputat{\pgfxy(0.15,0.65)}{\pgfbox[left,bottom]{\pgfuseimage{pvalmuDietn50sq}}}}
\only<14-24>{
\pgfputat{\pgfxy(0.5,1.0)}{\pgfbox[center,top]{\begin{minipage}{11cm}\only<14>{\textbf{Indications \texttt{R}}~:\\\texttt{  > length(yD)\\
$[1]$ 50\\
> mean(yD)\\
$[1]$ 4.5\\
  > pnorm(deltaEst.H0)\\
$[1]$ 0.9795589}}\visible<15-24>{\noindent\textbf{Hypothèses de test~:}}\visible<17-24>{ {\small $\mathbf{H}_0:$ $\mu^{D}=4$} vs } \visible<15-24>{$\mathbf{H}_1:$ $\mu^{D}>4$}\\\only<15>{\\}\visible<19-24>{\noindent\textbf{Statistique de test sous $\mathbf{H_0}$~:}\\
\centerline{$\Est{\delta_{\mu^{D},4}}{Y^{D}}= {\displaystyle \frac{\Est{\mu^{D}}{Y^{D}}-4}{
\Est{\sigma_{\cqlshat{\mu^{D}}}}{Y^{D}}
}} 
  \SuitApprox\mathcal{N}(0,1) $}\newline}
\visible<21-24>{\noindent\textbf{Règle de Décision}~:\\
\centerline{Accepter $\mathbf{H_1}$ si   {\color<23>{orange}p-valeur (droite)} < {\color<23>{blue}5\%}}\newline}
\visible<23-24>{\small\noindent\textbf{Conclusion}~: puisqu'au vu des données,
\\ \texttt{p-valeur}$\NotR{\color<23>{orange}\mathtt{1-pnorm((mean(yD)-4)/seMean(yD))}}$\\ \phantom{\texttt{p-valeur}}${\color<23>{orange}\simeq 2.04\%}<{\color<23>{blue}5\%}$\\
 on peut plutôt penser (avec un risque de 5\%) que le régime alimentaire du diététicien permet une perte de poids de 2 kilos par semaine.}\end{minipage}}}}
\only<23>{
\pgfputat{\pgfxy(0.15,0.65)}{\pgfbox[left,bottom]{\pgfuseimage{pvalmuDietn50sp}}}}
\only<25-25
>{
\pgfputat{\pgfxy(0.5,1.0)}{\pgfbox[center,top]{\begin{minipage}{11cm}
{\small
\noindent \textbf{Assertion d'intérêt} :  $\mathbf{H}_1:$ $\mu^{D}>4$ \\
\textbf{Application numérique} :  puisqu'au vu des données, 
  \begin{eqnarray*}
\Est{\delta_{\mu^{D},4}}{y^{D}} &\NotR&\mathtt{(mean(yD)-4)/seMean(yD)}\simeq 2.044722\\& >  & \delta^+_{lim,5\%} \NotR \mathtt{qnorm(1-.05)}\simeq1.644854
\end{eqnarray*}
  
on peut plutôt penser (avec un risque de 5\%) que le régime alimentaire du diététicien permet une perte de poids de 2 kilos par semaine. }
\end{minipage}}}}
\only<26-26
>{
\pgfputat{\pgfxy(0.5,1.0)}{\pgfbox[center,top]{\begin{minipage}{11cm}
{\small
\noindent \textbf{Assertion d'intérêt} :  $\mathbf{H}_1:$ $\mu^{D}>4$ \\
\textbf{Application numérique} :  puisqu'au vu des données, 
  \[
p-valeur\NotR\mathtt{1-pnorm((mean(yD)-4)/seMean(yD))} \simeq 2.04\% < 5\%,
\]
on peut plutôt penser (avec un risque de 5\%) que le régime alimentaire du diététicien permet une perte de poids de 2 kilos par semaine. }
\end{minipage}}}}
\only<27-37>{
\pgfputat{\pgfxy(0.5,1.0)}{\pgfbox[center,top]{\begin{minipage}{11cm}\only<27>{\textbf{Indications \texttt{R}}~:\\\texttt{  > length(yA)\\
$[1]$ 50\\
> var(yA)\\
$[1]$ 0.06362229\\
  > deltaEst.H0 \# instruction R à fournir\\
$[1]$ -2.762438\\
> qnorm(0.95)\\
$[1]$ 1.644854}}\visible<28-37>{\noindent\textbf{Hypothèses de test~:}}\visible<30-37>{ {\small $\mathbf{H}_0:$ $\sigma^2_{A}=0.1$} vs } \visible<28-37>{$\mathbf{H}_1:$ $\sigma^2_{A}<0.1$}\\\only<28>{\\}\visible<32-37>{\noindent\textbf{Statistique de test sous $\mathbf{H_0}$~:}\\
\centerline{$\Est{\delta_{\sigma^2_{A},0.1}}{Y^{A}}= {\displaystyle \frac{\Est{\sigma^2_{A}}{Y^{A}}-0.1}{
\Est{\sigma_{\cqlshat{\sigma^2_{A}}}}{Y^{A}}
}} 
  \SuitApprox\mathcal{N}(0,1) $}\newline}
\visible<34-37>{\noindent\textbf{Règle de Décision}~:\\
\centerline{Accepter $\mathbf{H_1}$ si   ${\color<36>{orange}\Est{\delta_{\sigma^2_{A},0.1}}{y^{A}}} < {\color<36>{darkgreen}\delta^-_{lim,5\%}}$}\newline}
\visible<36-37>{\small\noindent\textbf{Conclusion}~: puisqu'au vu des données,
\\$\Est{\delta_{\sigma^2_{A},0.1}}{y^{A}}\!\NotR\! {\color<36>{orange}\mbox{\scriptsize\texttt{(var(yA)-0.1)/seVar(yA)}}\!\simeq\! -2.762}$\\\phantom{$\!\Est{\delta_{\sigma^2_{A},0.1}}{y^{A}}\!$}$\!<  \delta^-_{lim,5\%}\!
\NotR\!{\color<36>{darkgreen}\mathtt{-qnorm(1-.05)\!\simeq\!-1.645}}\!$\\
 on peut plutôt penser (avec un risque de 5\%) que Alfred est compétent.}\end{minipage}}}}
\only<36>{
\pgfputat{\pgfxy(0.15,0.65)}{\pgfbox[left,bottom]{\pgfuseimage{pvalalfredn50sq}}}}
\only<38-48>{
\pgfputat{\pgfxy(0.5,1.0)}{\pgfbox[center,top]{\begin{minipage}{11cm}\only<38>{\textbf{Indications \texttt{R}}~:\\\texttt{  > length(yA)\\
$[1]$ 50\\
> var(yA)\\
$[1]$ 0.06362229\\
  > pnorm(deltaEst.H0)\\
$[1]$ 0.002868572}}\visible<39-48>{\noindent\textbf{Hypothèses de test~:}}\visible<41-48>{ {\small $\mathbf{H}_0:$ $\sigma^2_{A}=0.1$} vs } \visible<39-48>{$\mathbf{H}_1:$ $\sigma^2_{A}<0.1$}\\\only<39>{\\}\visible<43-48>{\noindent\textbf{Statistique de test sous $\mathbf{H_0}$~:}\\
\centerline{$\Est{\delta_{\sigma^2_{A},0.1}}{Y^{A}}= {\displaystyle \frac{\Est{\sigma^2_{A}}{Y^{A}}-0.1}{
\Est{\sigma_{\cqlshat{\sigma^2_{A}}}}{Y^{A}}
}} 
  \SuitApprox\mathcal{N}(0,1) $}\newline}
\visible<45-48>{\noindent\textbf{Règle de Décision}~:\\
\centerline{Accepter $\mathbf{H_1}$ si   {\color<47>{orange}p-valeur (gauche)} < {\color<47>{blue}5\%}}\newline}
\visible<47-48>{\small\noindent\textbf{Conclusion}~: puisqu'au vu des données,
\\ \texttt{p-valeur}$\NotR{\color<47>{orange}\mathtt{pnorm((var(yA)-0.1)/seVar(yA))}}$\\ \phantom{\texttt{p-valeur}}${\color<47>{orange}\simeq 0.29\%}<{\color<47>{blue}5\%}$\\
 on peut plutôt penser (avec un risque de 5\%) que Alfred est compétent.}\end{minipage}}}}
\only<47>{
\pgfputat{\pgfxy(0.15,0.65)}{\pgfbox[left,bottom]{\pgfuseimage{pvalalfredn50sp}}}}
\only<49-49
>{
\pgfputat{\pgfxy(0.5,1.0)}{\pgfbox[center,top]{\begin{minipage}{11cm}
{\small
\noindent \textbf{Assertion d'intérêt} :  $\mathbf{H}_1:$ $\sigma^2_{A}<0.1$ \\
\textbf{Application numérique} :  puisqu'au vu des données, 
  \begin{eqnarray*}
\Est{\delta_{\sigma^2_{A},0.1}}{y^{A}} &\NotR&\mathtt{(var(yA)-0.1)/seVar(yA)}\simeq -2.762438\\& <  & \delta^-_{lim,5\%} \NotR \mathtt{-qnorm(1-.05)}\simeq-1.644854
\end{eqnarray*}
  
on peut plutôt penser (avec un risque de 5\%) que Alfred est compétent. }
\end{minipage}}}}
\only<50-50
>{
\pgfputat{\pgfxy(0.5,1.0)}{\pgfbox[center,top]{\begin{minipage}{11cm}
{\small
\noindent \textbf{Assertion d'intérêt} :  $\mathbf{H}_1:$ $\sigma^2_{A}<0.1$ \\
\textbf{Application numérique} :  puisqu'au vu des données, 
  \[
p-valeur\NotR\mathtt{pnorm((var(yA)-0.1)/seVar(yA))} \simeq 0.29\% < 5\%,
\]
on peut plutôt penser (avec un risque de 5\%) que Alfred est compétent. }
\end{minipage}}}}

\end{pgfpicture}

\end{beamerboxesrounded}
%\end{beamercolorbox}
\begin{tikzpicture}[remember picture,overlay]
  \node [rotate=30,scale=10,text opacity=0.05]
    at (current page.center) {CQLS};
\end{tikzpicture}
\end{frame}

\begin{frame}

\begin{pgfpicture}{0cm}{0cm}{11cm}{8cm}
\pgfsetendarrow{\pgfarrowto}
\pgfnodebox{CritInt}[stroke]{\pgfxy(5.5,8)}{\color[rgb]{0,0,1}Nbre de paramètres pour décrire $\mathbf{H_1}$~?}{2pt}{2pt}
\pgfnodebox{Par1}[stroke]{\pgfxy(3.5,6.4)}{1 paramètre}{2pt}{2pt}
\pgfnodeconncurve{CritInt}{Par1}{-90}{90}{0.5cm}{0.5cm}
\pgfnodebox{Gauss}[stroke]{\pgfxy(8,4.4)}{\color[rgb]{0,0.6,0}Gaussien}{2pt}{2pt}
\pgfnodeconncurve{Par1}{Gauss}{-90}{90}{2cm}{0.5cm} 

\end{pgfpicture}
\pgftext[top,right,at={\pgfxy(0,4.0)}]{\begin{minipage}{11cm}
\hrulefill\\
%%% 1 param et gaussien 
\noindent\textbf{Données~:} $\Vect{Y}=(Y_1,\cdots,Y_n)$ avec $Y{\color{blue}\leadsto \mathcal{N}(\mu^\bullet,\sigma_\bullet)}$\\
\noindent\textbf{Statistique de test sous $\mathbf{H_0}$~:}\\ 
\centerline{\begin{tabular}{|c|c|}\hline
$\mu^\bullet$ &  $\scriptsize\Est{\delta_{\mu^\bullet,\mu_0}}{Y}:=\frac{\Est{\mu^\bullet}{Y}-\mu_0}{\Est{\sigma_{\widehat{\mu^\bullet}}}{Y}}\leadsto \mathcal{S}t(n-1)$\\\hline
$\sigma^2_\bullet$ & $\scriptsize\Est{\delta_{\sigma^2_\bullet,\sigma^2_0}}{Y}:=(n-1)\times\frac{\Est{\sigma^2_\bullet}{Y}}{\sigma^2_0}\leadsto \chi^2(n-1)$\\\hline
\end{tabular}}



\end{minipage}}

\end{frame}

\pgfdeclareimage[width=8cm,height=4cm,interpolate=true]{pvalalfredn20sq}{img/pvalalfredn20sq}

\pgfdeclareimage[width=8cm,height=4cm,interpolate=true]{pvalalfredn20sp}{img/pvalalfredn20sp}


%\beamertemplateshadingbackground{green!50}{yellow!50}
\begin{frame}<1->
\setbeamercolor{header}{fg=black,bg=blue!40!white}
 \hspace*{2.5cm}\begin{beamerboxesrounded}[width=6cm,shadow=true,lower=header]{}
  \pgfsetxvec{\pgfpoint{6cm}{0cm}}
\pgfsetyvec{\pgfpoint{0cm}{0.5cm}}
\begin{pgfpicture}{0cm}{0cm}{6cm}{0.5cm}

  \only<1-1
>{
\pgfputat{\pgfxy(0.5,0.5)}{\pgfbox[center,center]{\textbf{\large Diététicien (abr. quant)}}}}
\only<2-2
>{
\pgfputat{\pgfxy(0.5,0.5)}{\pgfbox[center,center]{\textbf{\large Diététicien (abr. p-val)}}}}
\only<3-13>{
\pgfputat{\pgfxy(0.5,0.5)}{\pgfbox[center,center]{\textbf{\large Alfred (quantile)}}}}
\only<14-24>{
\pgfputat{\pgfxy(0.5,0.5)}{\pgfbox[center,center]{\textbf{\large Alfred (p-valeur)}}}}

  \end{pgfpicture}

\end{beamerboxesrounded}

\setbeamercolor{postit}{fg=black,bg=green!40!white}
%\begin{beamercolorbox}[sep=1em,wd=12cm]{postit}
\begin{beamerboxesrounded}[shadow=true,lower=postit]{}
\pgfsetxvec{\pgfpoint{11cm}{0cm}}
\pgfsetyvec{\pgfpoint{0cm}{2.1cm}}
\begin{pgfpicture}{0cm}{0cm}{11cm}{2.1cm}

\only<1-1
>{
\pgfputat{\pgfxy(0.5,0.5)}{\pgfbox[center,center]{\begin{minipage}{11cm}
\textbf{Question } Peut-on penser que le régime alimentaire du diététicien permet une perte de poids
au vu des données \texttt{AV-AP} en \texttt{R}~?\\
\textbf{Indic \texttt{R}}~: \texttt{deltaEst.H0}$\simeq$\texttt{5.25}\\\phantom{\textbf{Indic \texttt{R}}~: }\texttt{qt(0.95,9)}$\simeq$ \texttt{1.833113} 

\end{minipage}}}}
\only<2-2
>{
\pgfputat{\pgfxy(0.5,0.5)}{\pgfbox[center,center]{\begin{minipage}{11cm}
\textbf{Question } Peut-on penser que le régime alimentaire du diététicien permet une perte de poids
au vu des données \texttt{AV-AP} en \texttt{R}~?\\
\textbf{Indic \texttt{R}}~: \texttt{pt(deltaEst.H0,9)}$\simeq$ \texttt{0.9997362}
\end{minipage}}}}
\only<3-4>{
\pgfputat{\pgfxy(0.5,1.0)}{\pgfbox[center,top]{\begin{minipage}{11cm}\textbf{Question } Comment s'écrit l'assertion d'intérêt $\mathbf{H_1}$ en fonction des paramètres d'intérêt et d'écart~?\only<3>{\\\textbf{Assertion d'intérêt}~: Alfred est compétent}\only<4>{\\ \centerline{$\mathbf{H_1}$:$\sigma^2_{A}<0.1$ $\Longleftrightarrow$ $\delta_{\sigma^2_{A},0.1}:={\displaystyle (20-1)\frac{\sigma^2_{A}}{0.1}}<20-1$}}
\end{minipage}}}}
\only<5-6>{
\pgfputat{\pgfxy(0.5,1.0)}{\pgfbox[center,top]{\begin{minipage}{11cm}\textbf{Question }: Quelle est la pire des situations, i.e. parmi toutes les situations quelle est celle qui engendre le plus grand risque d'erreur de première espèce~?\end{minipage}}}}
\only<7-8>{
\pgfputat{\pgfxy(0.5,1.0)}{\pgfbox[center,top]{\begin{minipage}{11cm}\textbf{Question }: Quelle est l'information du mathématicien quant au comportement de $\Est{\delta_{\sigma^2_{A},0.1}}{Y^{A}}$ dans la pire des situations~?\end{minipage}}}}
\only<9-10>{
\pgfputat{\pgfxy(0.5,1.0)}{\pgfbox[center,top]{\begin{minipage}{11cm}\textbf{Question }: Comment s'écrit la règle de décision ne produisant pas plus de 5\% d'erreur de première espèce~? \\
\textbf{Indic \texttt{R}}~: \texttt{deltaEst.H0}$\simeq$\texttt{13.91858}\\\phantom{\textbf{Indic \texttt{R}}~: }\texttt{qchisq(0.95,19)}$\simeq$ \texttt{30.14353}
\end{minipage}}}}
\only<11-13>{
\pgfputat{\pgfxy(0.5,1.0)}{\pgfbox[center,top]{\begin{minipage}{11cm}\textbf{Question }: Comment conclueriez-vous au vu des données \texttt{yA} en \texttt{R}~?\\
\textbf{Indic \texttt{R}}~: \texttt{deltaEst.H0}$\simeq$\texttt{13.91858}\\\phantom{\textbf{Indic \texttt{R}}~: }\texttt{qchisq(0.95,19)}$\simeq$ \texttt{30.14353} 
\end{minipage}}}}
\only<14-15>{
\pgfputat{\pgfxy(0.5,1.0)}{\pgfbox[center,top]{\begin{minipage}{11cm}\textbf{Question } Comment s'écrit l'assertion d'intérêt $\mathbf{H_1}$ en fonction des paramètres d'intérêt et d'écart~?\only<14>{\\\textbf{Assertion d'intérêt}~: Alfred est compétent}\only<15>{\\ \centerline{$\mathbf{H_1}$:$\sigma^2_{A}<0.1$ $\Longleftrightarrow$ $\delta_{\sigma^2_{A},0.1}:={\displaystyle (20-1)\frac{\sigma^2_{A}}{0.1}}<20-1$}}
\end{minipage}}}}
\only<16-17>{
\pgfputat{\pgfxy(0.5,1.0)}{\pgfbox[center,top]{\begin{minipage}{11cm}\textbf{Question }: Quelle est la pire des situations, i.e. parmi toutes les situations quelle est celle qui engendre le plus grand risque d'erreur de première espèce~?\end{minipage}}}}
\only<18-19>{
\pgfputat{\pgfxy(0.5,1.0)}{\pgfbox[center,top]{\begin{minipage}{11cm}\textbf{Question }: Quelle est l'information du mathématicien quant au comportement de $\Est{\delta_{\sigma^2_{A},0.1}}{Y^{A}}$ dans la pire des situations~?\end{minipage}}}}
\only<20-21>{
\pgfputat{\pgfxy(0.5,1.0)}{\pgfbox[center,top]{\begin{minipage}{11cm}\textbf{Question }: Comment s'écrit la règle de décision ne produisant pas plus de 5\% d'erreur de première espèce~? \\
\textbf{Indic \texttt{R}}~: \texttt{pchisq(deltaEst.H0,19)}$\simeq$ \texttt{0.2115835}\end{minipage}}}}
\only<22-24>{
\pgfputat{\pgfxy(0.5,1.0)}{\pgfbox[center,top]{\begin{minipage}{11cm}\textbf{Question }: Comment conclueriez-vous au vu des données \texttt{yA} en \texttt{R}~?\\
\textbf{Indic \texttt{R}}~: \texttt{pchisq(deltaEst.H0,19)}$\simeq$ \texttt{0.2115835}\end{minipage}}}}

\end{pgfpicture}

\end{beamerboxesrounded}
%\end{beamercolorbox}

\setbeamercolor{postex}{fg=black,bg=yellow!50!white}
%\begin{beamercolorbox}[sep=1em,wd=12cm]{postex}
\begin{beamerboxesrounded}[shadow=true,lower=postex]{}
\pgfsetxvec{\pgfpoint{11cm}{0cm}}
\pgfsetyvec{\pgfpoint{0cm}{5cm}}
\begin{pgfpicture}{0cm}{0cm}{11cm}{5cm}

\only<1-1
>{
\pgfputat{\pgfxy(0.5,1.0)}{\pgfbox[center,top]{\begin{minipage}{11cm}
{\small
\noindent \textbf{Assertion d'intérêt} :  $\mathbf{H}_1:$ $\mu^{D}>0$ avec $\mu^{D}=\mbox{moyenne de }Y^{D}(=Y^{AV}-Y^{AP})$ \\
\textbf{Application numérique} :  puisqu'au vu des données, 
  \begin{eqnarray*}
\Est{\delta_{\mu^{D},0}}{y^{D}} &\NotR&\mathtt{(mean(AV-AP)-0)/seMean(AV-AP)}\simeq 5.25\\& >  & \delta^+_{lim,5\%} \NotR \mathtt{qt(1-.05,9)}\simeq1.833113
\end{eqnarray*}
  
on peut plutôt penser (avec un risque de 5\%) que le régime alimentaire du diététicien permet une perte de poids. }
\end{minipage}}}}
\only<2-2
>{
\pgfputat{\pgfxy(0.5,1.0)}{\pgfbox[center,top]{\begin{minipage}{11cm}
{\small
\noindent \textbf{Assertion d'intérêt} :  $\mathbf{H}_1:$ $\mu^{D}>0$ avec $\mu^{D}=\mbox{moyenne de }Y^{D}(=Y^{AV}-Y^{AP})$ \\
\textbf{Application numérique} :  puisqu'au vu des données, 
  \[
p-valeur\NotR\mathtt{1-pt((mean(AV-AP)-0)/seMean(AV-AP),9)} \simeq 0.03\% < 5\%,
\]
on peut plutôt penser (avec un risque de 5\%) que le régime alimentaire du diététicien permet une perte de poids. }
\end{minipage}}}}
\only<3-13>{
\pgfputat{\pgfxy(0.5,1.0)}{\pgfbox[center,top]{\begin{minipage}{11cm}\only<3>{\textbf{Indications \texttt{R}}~:\\\texttt{  > length(yA)\\
$[1]$ 20\\
> var(yA)\\
$[1]$ 0.07325571\\
  > deltaEst.H0 \# instruction R à fournir\\
$[1]$ 13.91858\\
> qchisq(0.95,19)\\
$[1]$ 30.14353}}\visible<4-13>{\noindent\textbf{Hypothèses de test~:}}\visible<6-13>{ {\small $\mathbf{H}_0:$ $\sigma^2_{A}=0.1$} vs } \visible<4-13>{$\mathbf{H}_1:$ $\sigma^2_{A}<0.1$}\\\only<4>{\\}\visible<8-13>{\noindent\textbf{Statistique de test sous $\mathbf{H_0}$~:}\\
\centerline{$\Est{\delta_{\sigma^2_{A},0.1}}{Y^{A}}= {\displaystyle (20-1)\frac{\Est{\sigma^2_{A}}{Y^{A}}}{0.1}} 
  \leadsto\chi^2(20-1) $}\newline}
\visible<10-13>{\noindent\textbf{Règle de Décision}~:\\
\centerline{Accepter $\mathbf{H_1}$ si   ${\color<12>{orange}\Est{\delta_{\sigma^2_{A},0.1}}{y^{A}}} < {\color<12>{darkgreen}\delta^-_{lim,5\%}}$}\newline}
\visible<12-13>{\small\noindent\textbf{Conclusion}~: puisqu'au vu des données,
\\$\Est{\delta_{\sigma^2_{A},0.1}}{y^{A}}\!\NotR\! {\color<12>{orange}\mbox{\scriptsize\texttt{(length(yA)-1)*var(yA)/0.1}}\!\simeq\! 13.92}$\\\phantom{$\!\Est{\delta_{\sigma^2_{A},0.1}}{y^{A}}\!$}$\!\nless  \delta^-_{lim,5\%}\!
\NotR\!{\color<12>{darkgreen}\mathtt{qchisq(.05,19)\!\simeq\!10.12}}\!$\\
 on ne peut pas plutôt penser (avec un risque de 5\%) que Alfred est compétent.}\end{minipage}}}}
\only<12>{
\pgfputat{\pgfxy(0.15,0.65)}{\pgfbox[left,bottom]{\pgfuseimage{pvalalfredn20sq}}}}
\only<14-24>{
\pgfputat{\pgfxy(0.5,1.0)}{\pgfbox[center,top]{\begin{minipage}{11cm}\only<14>{\textbf{Indications \texttt{R}}~:\\\texttt{  > length(yA)\\
$[1]$ 20\\
> var(yA)\\
$[1]$ 0.07325571\\
  > pchisq(deltaEst.H0,19)\\
$[1]$ 0.2115835}}\visible<15-24>{\noindent\textbf{Hypothèses de test~:}}\visible<17-24>{ {\small $\mathbf{H}_0:$ $\sigma^2_{A}=0.1$} vs } \visible<15-24>{$\mathbf{H}_1:$ $\sigma^2_{A}<0.1$}\\\only<15>{\\}\visible<19-24>{\noindent\textbf{Statistique de test sous $\mathbf{H_0}$~:}\\
\centerline{$\Est{\delta_{\sigma^2_{A},0.1}}{Y^{A}}= {\displaystyle (20-1)\frac{\Est{\sigma^2_{A}}{Y^{A}}}{0.1}} 
  \leadsto\chi^2(20-1) $}\newline}
\visible<21-24>{\noindent\textbf{Règle de Décision}~:\\
\centerline{Accepter $\mathbf{H_1}$ si   {\color<23>{orange}p-valeur (gauche)} < {\color<23>{blue}5\%}}\newline}
\visible<23-24>{\small\noindent\textbf{Conclusion}~: puisqu'au vu des données,
\\ \texttt{p-valeur}$\NotR{\color<23>{orange}\mathtt{pchisq((length(yA)-1)*var(yA)/0.1,19)}}$\\ \phantom{\texttt{p-valeur}}${\color<23>{orange}\simeq 21.16\%}\nless{\color<23>{blue}5\%}$\\
 on ne peut pas plutôt penser (avec un risque de 5\%) que Alfred est compétent.}\end{minipage}}}}
\only<23>{
\pgfputat{\pgfxy(0.15,0.65)}{\pgfbox[left,bottom]{\pgfuseimage{pvalalfredn20sp}}}}

\end{pgfpicture}

\end{beamerboxesrounded}
%\end{beamercolorbox}
\begin{tikzpicture}[remember picture,overlay]
  \node [rotate=30,scale=10,text opacity=0.05]
    at (current page.center) {CQLS};
\end{tikzpicture}
\end{frame}





\end{document}


