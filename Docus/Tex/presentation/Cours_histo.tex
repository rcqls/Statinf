
\documentclass[11pt]{beamer}
%Packages
\usepackage{multirow}
\usepackage{graphicx}
\usepackage[utf8]{inputenc}
\usepackage[T1]{fontenc}
\usepackage{aeguill}
\usepackage{amssymb}
\usepackage{pgf,pgfarrows,pgfnodes}
\usepackage[french]{babel}
\usepackage{float}
\usepackage{xkeyval,calc,listings,tikz}
\usepackage{fancyvrb}

%Preamble

\input /Users/remy/cqls/texinputs/Cours/cqlsInclude
\input /Users/remy/cqls/texinputs/Cours/testInclude


\mode<article>{\usepackage{fullpage}}
%\usetheme{Darmstadt}
%\usepackage{times}
%\usefonttheme{structurebold}
\usetheme{Antibes}
%\usecolortheme{lily}

\definecolor{VertFonce}{rgb}{0,.4,.0}

\mode<presentation>
{
  %\usetheme{Warsaw}
  % or ...
\usetheme{Boadilla}

  \setbeamercovered{transparent=5}
  % or whatever (possibly just delete it)
}
%\setbeamercovered{dynamic}


\subject{Talks}

\AtBeginSection[]
{
  \begin{frame}<beamer>
    \frametitle{Plan}
    \tableofcontents[currentsection,currentsubsection]
  \end{frame}
}


\setbeamercovered{invisible}
\newcommand{\Sim}{{\star}}
\newcommand{\ok}{ \textcolor{green}{\large$\surd$}}
\newcommand{\nok}{ \textcolor{red}{\large X}}


\usetikzlibrary{arrows,%
  calc,%
  fit,%
  patterns,%
  plotmarks,%
  shapes.geometric,%
  shapes.misc,%
  shapes.symbols,%
  shapes.arrows,%
  shapes.callouts,%
  shapes.multipart,%
  shapes.gates.logic.US,%
  shapes.gates.logic.IEC,%
  er,%
  automata,%
  backgrounds,%
  chains,%
  topaths,%
  trees,%
  petri,%
  mindmap,%
  matrix,%
  calendar,%
  folding,%
  fadings,%
  through,%
  positioning,%
  scopes,%
  decorations.fractals,%
  decorations.shapes,%
  decorations.text,%
  decorations.pathmorphing,%
  decorations.pathreplacing,%
  decorations.footprints,%
  decorations.markings,%
  shadows}
\tikzset{
  every plot/.style={prefix=plots/pgf-},
  shape example/.style={
    color=black!30,
    draw,
    fill=yellow!30,
    line width=.5cm,
    inner xsep=2.5cm,
    inner ysep=0.5cm}
}

%Styles

%Title


\title[Problématiques Produits A et B]
{Cours de Statistiques Inférentielles}
\author{CQLS~: cqls@upmf-grenoble.fr}
\date{\today}

\begin{document}
\maketitle


%\begin{frame}
%  \titlepage
%\end{frame}
\section[A.E.P.]{Approche Expérimentale des Probabilités}

\begin{frame}
\frametitle{Conseils pour l'industriel }
\begin{alertblock}{Expérience}
L'industriel demande conseil auprès de~:
\begin{enumerate}[<+-| alert@+>]
\item Un Expérimentateur~: naïvement, celui-ci se propose de reproduire l'expérience que se propose de faire l'industriel le \textbf{jour~J} en la répétant $m=10000$ fois pour autant de situations souhaitées par l'industriel.
\item Un Mathématicien~: de manière un peu arrogante, il prétend connaître tout ce qui peut se passer avant le \textbf{jour~J} dès lors qu'une population totale  lui est proposé.
\item[] Puisque $\mu^\bullet$  (resp. $\underline{\mathcal{Y}}^\bullet$) est inconnu, les deux conseillers s'accordent sur le fait de le remplacer par $\mu^\Sim$ (resp. $\underline{\mathcal{Y}}^\Sim$) fixé arbitrairement avant le \textbf{jour~J} parmi l'ensemble des valeurs possibles de $\mu^\bullet$. 
\end{enumerate}
\end{alertblock}
\end{frame}

\begin{frame}
\frametitle{Travail de l'expérimentateur}
\begin{alertblock}{Expérience}
Il construit des urnes contenant $N=2000000$ boules avec des répartitions en boules $0,1,2,3,\ldots$ spécifiques.
puis effectue \textbf{des} tirages (avec remise) de $n=1000$ boules au hasard au sein de cette urne.
\end{alertblock}

\begin{exampleblock}{Les urnes expérimentales}
\begin{center}
\begin{tabular}{|c|c|c|c|}
\hline 
situation & urne $U_{0.1}^A$ & urne $U_{0.15}^A$ & urne $U_{0.2}^A$ \\
\hline
Caract. & $N_1=200000$ &  $N_1=300000$ & $N_1=400000$  \\
\hline
$\mu^\Sim=$ & 0.1 & 0.15 & 0.2 \\
\hline
\end{tabular}
\end{center}
\begin{center}
\begin{tabular}{|c|c|c|c|}
\hline
& urne $U_{0.1}^B$ & urne $U_{0.15}^B$ & urne $U_{0.2}^B$ \\
\hline
Caract. & $N_1=100000$ &  $N_1=200000$ & $N_1=300000$  \\
& \multicolumn{3}{|c|}{$N_2=N_3=20000$} \\
\hline
$\mu^\Sim=$ & 0.1 & 0.15 & 0.2 \\
\hline
\end{tabular}
\end{center}
\end{exampleblock}
\end{frame}


\begin{frame}<1->[label=aep]
\frametitle<1->{\textbf{A}pproche \textbf{E}xpérimentale des \textbf{P}robabilités : l'Expérimentateur versus le Matheux}
\frametitle<11->{Histogramme discret}
\begin{onlyenv}<1-10>
\begin{columns}
\begin{column}{5cm}
      \begin{enumerate}
     \item[] \textbf{L'Expérimentateur ~:} 
        \item<1-10| alert@1-4>
         Réaliser $m$ expériences
        \item<5-10| alert@5-8>
          Répartition des $\Est{\mu^\Sim}{y^\Sim_{[j]}}$
          \hyperlink{aep<11>}{\beamergotobutton{Histo}}
  \item<9-10>[] \textbf{Le Matheux :}      
  \item<9-10| alert@9>
          Je le savais \textbf{à l'avance} pour $m\to+\infty$
           \item<10| alert@10>
           $\Est{\mu^\Sim}{Y^\Sim}\SuitApprox \mathcal{N}(\mu^\Sim,\frac{\sigma_\Sim}{\sqrt{n}})$
      \end{enumerate}
     
\end{column}    

\begin{column}{6.5cm}






\begin{pgfpicture}{-0.5cm}{0cm}{6cm}{6cm}
      \only<2-10>{\pgfputat{\pgfxy(0,4.4)}{\pgfbox[center,center]{$14.6\%$}}}
       \only<3-10>{\pgfputat{\pgfxy(2,4.4)}{\pgfbox[center,center]{$16.3\%$}}}
       \only<4-5>{\pgfputat{\pgfxy(5.5,4.4)}{\pgfbox[center,center]{$14.1\%$}}}
      \only<6>{\pgfputat{\pgfxy(5.5,4.4)}{\pgfbox[center,center]{$15.1\%$}}}
      \only<7>{\pgfputat{\pgfxy(5.5,4.4)}{\pgfbox[center,center]{$14.7\%$}}}
      \only<8>{\pgfputat{\pgfxy(5.5,4.4)}{\pgfbox[center,center]{$16.3\%$}}}
      \only<9-10>{\pgfputat{\pgfxy(5.5,4.4)}{\pgfbox[center,center]{$\cdots$}}}
     \pgfnodebox{EstY}[stroke]{\pgfxy(3,6.3)}{Future $\Est{\mu^\Sim}{Y^\Sim}$}{2pt}{2pt}
      \only<2-10>{\pgfnodebox{esty1}[stroke]{\pgfxy(0,5)}{$\Est{\mu^\Sim}{y^\Sim_{[1]}}$}{2pt}{2pt}}
       \only<3-10>{\pgfnodebox{esty2}[stroke]{\pgfxy(2,5)}{$\Est{\mu^\Sim}{y^\Sim_{[2]}}$}{2pt}{2pt}}
     \only<4-10>{\pgfputat{\pgfxy(3.75,5)}{\pgfbox[center,center]{$\cdots\cdots$}}
        \pgfnodebox{estym}[stroke]{\pgfxy(5.5,5)}{$\Est{\mu^\Sim}{y^\Sim_{[m]}}$}{2pt}{2pt}}
  \pgfsetendarrow{\pgfarrowto}
\only<2,5-10>{\pgfnodeconncurve{EstY}{esty1}{-90}{90}{0.5cm}{0.5cm}}
\only<2>{\pgfnodelabel{EstY}{esty1}[0.5][-3pt]{\pgfbox[right,base]{$1^{\grave eme}$ réalisation}}
}
\only<3,5-10>{\pgfnodeconncurve{EstY}{esty2}{-90}{90}{0.5cm}{0.5cm}}
\only<3>{\pgfnodelabel{EstY}{esty2}[0.5][-3pt]{\pgfbox[center,base]{$2^{\grave eme}$ réalisation}}
}
\only<4-10>{\pgfnodeconncurve{EstY}{estym}{-90}{90}{0.5cm}{0.5cm}}
\only<4>{\pgfnodelabel{EstY}{estym}[0.5][-3pt]{\pgfbox[center,base]{$m^{\grave eme}$ réalisation}}
}
\only<5>{
\pgfdeclareimage[interpolate=true,height=5cm]{image1}{img/hist100A15}
\pgfputat{\pgfxy(3,1.5)}{\pgfbox[center,center]{\pgfuseimage{image1}}}
\pgfputat{\pgfxy(3.5,3.75)}{\pgfbox[center,center]{$m=100$}}
}
\only<6>{
\pgfdeclareimage[interpolate=true,height=5cm]{image2}{img/hist1000A15}
\pgfputat{\pgfxy(3,1.5)}{\pgfbox[center,center]{\pgfuseimage{image2}}}
\pgfputat{\pgfxy(3.5,3.75)}{\pgfbox[center,center]{$m=1000$}}
}
\only<7>{
\pgfdeclareimage[interpolate=true,height=5cm]{image3}{img/hist5000A15}
\pgfputat{\pgfxy(3,1.5)}{\pgfbox[center,center]{\pgfuseimage{image3}}}
\pgfputat{\pgfxy(3.5,3.75)}{\pgfbox[center,center]{$m=5000$}}
}
\only<8>{
\pgfdeclareimage[interpolate=true,height=5cm]{image4}{img/hist10000A15}
\pgfputat{\pgfxy(3,1.5)}{\pgfbox[center,center]{\pgfuseimage{image4}}}
\pgfputat{\pgfxy(3.5,3.75)}{\pgfbox[center,center]{$m=10000$}}
}
\only<9>{
\pgfdeclareimage[interpolate=true,height=5cm]{image5}{img/hist10000A15Norm}
\pgfputat{\pgfxy(3,1.5)}{\pgfbox[center,center]{\pgfuseimage{image5}}}
\pgfputat{\pgfxy(3.5,3.75)}{\pgfbox[center,center]{$m=+\infty$}}
}
\only<10>{
\pgfdeclareimage[interpolate=true,height=5cm]{image6}{img/hist10000A15NormContour}
\pgfputat{\pgfxy(3,1.5)}{\pgfbox[center,center]{\pgfuseimage{image6}}}
\pgfputat{\pgfxy(3.5,3.75)}{\pgfbox[center,center]{$m=10000$ vs $m=+\infty$}}
}
\end{pgfpicture}
\end{column}
    
\end{columns}  
\end{onlyenv}
%\end{frame}

%\begin{frame}
\begin{onlyenv}<11->
\begin{exampleblock}{Répartition de $m$ réalisations de $\Est{\mu^\Sim}{Y}$}
\begin{itemize}
\item<11-| alert@11> Chaque réalisation $\Est{\mu^\Sim}{y_{[j]}}$ de $\Est{\mu^\Sim}{Y}$ est représentée par une brique de surface $1/m$ et de largeur $1/n$.
\item<12-| alert@12> Toutes les briques sont empilées l'une après l'autre en les centrant en abscisse en leur valeur associée.
\item<13-| alert@13> L'évolution de cet empilement laisse apparaître un ``mur" de briques de surface totale toujours égale à 1. 
\item <14-| alert@14> Cette représentation est appelée histogramme (discret) et permet de visualiser en seul coup d'oeil la répartition des $(\Est{\mu^\Sim}{y_{[j]}})_{j=1,\cdots,m}$.
\end{itemize}
\hyperlink{aep<5>}{\beamerreturnbutton{Retour}} \hyperlink{recap<1>}{\beamergotobutton{Suite}}
\end{exampleblock}
\end{onlyenv}
\end{frame}



\begin{frame}[label=recap]
\frametitle{\textbf{A}pproche \textbf{E}xpérimentale des \textbf{P}robabilités : Tableau récapitulatif}
\begin{block}{}
{\tiny
\begin{center}
\begin{tabular}{|c|c|c|c|}\hline
\multicolumn{4}{|l|}{$\downarrow$ \textbf{AVANT le jour J}}\\\hline\hline
\multicolumn{4}{|c|}{\textbf{Phase expérimentale}}\\
\multicolumn{4}{|c|}{\textit{Le paramètre à estimer est $\mu^\Sim$ fixé arbitrairement (par exemple,  à $0.15$)}}\\\hline
Avant simulation & $\cqlsbm{\mathcal{E}^\Sim}=(\mathcal{E}^\Sim_1,\mathcal{E}^\Sim_2,...,\mathcal{E}^\Sim_n)$ & ${\cqlsbm Y}^\Sim=(Y^\Sim_1,Y^\Sim_2,\ldots,Y^\Sim_n)$ & $\Est{\mu^\Sim}{Y^\Sim}$ \pause\\\hline
%\multicolumn{2}{|c|}{{Simulation en \texttt{R} d'une réalisation}} & \texttt{YY<-Sample(Y,n)} & \texttt{mean(YY)} \\ \hline
Après simulation & $1^{\grave ere}$ expérience $\cqlsbm{e}^\Sim_{[1]}$ & $\cqlsbm{y}^\Sim_{[1]}$ & $\Est{\mu^\Sim}{y^\Sim_{[1]}}$\\
 & $2^{\grave eme}$ expérience $\cqlsbm{e}^\Sim_{[2]}$ & $\cqlsbm{y}^\Sim_{[2]}$ & $\Est{\mu^\Sim}{y^\Sim_{[2]}}$\\
 & $\vdots$ & $\vdots$ & $\vdots$ \\
 &$m^{\grave eme}$ expérience $\cqlsbm{e}^\Sim_{[m]}$ & $\cqlsbm{y}^\Sim_{[m]}$ & $\Est{\mu^\Sim}{y^\Sim_{[m]}}$ \\
 & $\vdots$ & $\vdots$ & $\vdots$ \pause\\ \hline\hline
%\multicolumn{2}{|c|}{{Simulation en \texttt{R} de $m$ réalisations}} & \texttt{sim(YY,m)} & \texttt{sim(mean(YY),m)} \\ \hline
\multicolumn{4}{|c|}{\textbf{Phase pratique}}\\
\multicolumn{4}{|c|}{\textit{Le paramètre à estimer est $\mu^\bullet$ qui est inconnu}}\\\hline
Avant pratique & $\cqlsbm{\mathcal{E}^\bullet}=(\mathcal{E}^\bullet_1,\mathcal{E}^\bullet_2,...,\mathcal{E}^\bullet_n)$ & ${\cqlsbm Y^\bullet}=(Y^\bullet_1,Y^\bullet_2,\ldots,Y^\bullet_n)$ & $\Est{\mu^\bullet}{Y^\bullet}$ \pause\\\hline\hline
\multicolumn{4}{|l|}{$\downarrow$ \textbf{APRES le jour J}}\\\hline\hline
Après pratique & l'expérience réelle $\cqlsbm{e}^\bullet$ & $\cqlsbm{y}^\bullet$ & $\Est{\mu^\bullet}{y^\bullet}$ \\\hline
\end{tabular}
\end{center}
}
\end{block}
\end{frame}

\begin{frame}
\frametitle<1-3>{Réalisation d'une future estimation par l'\textbf{Expérimentateur}}
\frametitle<4-6>{Réalisation d'une future estimation par le \textbf{Matheux}}

\begin{columns}
\begin{column}{5cm}
\begin{center}
\only<1,4>{

\begin{tabular}{|c|c|}\hline
\multicolumn{2}{|c|}{ 
 
} \\\hline
j & $\Est{\mu^\bullet}{y^\bullet_{[j]}}$\\\hline
$\vdots$ & $\vdots$\\



4150

& 

14.4\%

\\



4151

& 

17.2\%

\\



4152

& 

15\%

\\


{\color[rgb]{1,0,0}
4153
}
& 
{\color[rgb]{1,0,0}
14.9\%
}
\\



4154

& 

13.7\%

\\



4155

& 

15.8\%

\\



4156

& 

14.6\%

\\

$\vdots$ & $\vdots$\\
\hline
\end{tabular}

}\only<2,5>{

\begin{tabular}{|c|c|}\hline
\multicolumn{2}{|c|}{ 
 
} \\\hline
j & $\Est{\mu^\bullet}{y^\bullet_{[j]}}$\\\hline
$\vdots$ & $\vdots$\\



2105

& 

15.3\%

\\



2106

& 

14.1\%

\\



2107

& 

13.2\%

\\


{\color[rgb]{1,0,0}
2108
}
& 
{\color[rgb]{1,0,0}
15.5\%
}
\\



2109

& 

16.7\%

\\



2110

& 

15.5\%

\\



2111

& 

14.5\%

\\

$\vdots$ & $\vdots$\\
\hline
\end{tabular}

}\only<3,6>{

\begin{tabular}{|c|c|}\hline
\multicolumn{2}{|c|}{ 
 
} \\\hline
j & $\Est{\mu^\bullet}{y^\bullet_{[j]}}$\\\hline
$\vdots$ & $\vdots$\\



3728

& 

14.9\%

\\



3729

& 

14.4\%

\\



3730

& 

14.8\%

\\


{\color[rgb]{1,0,0}
3731
}
& 
{\color[rgb]{1,0,0}
13.4\%
}
\\



3732

& 

14.9\%

\\



3733

& 

14.4\%

\\



3734

& 

16.4\%

\\

$\vdots$ & $\vdots$\\
\hline
\end{tabular}

}
\end{center}
\end{column}
\begin{column}{5cm}






\begin{pgfpicture}{0cm}{0cm}{6cm}{6cm}
\only<1>{
\pgfdeclareimage[interpolate=true,height=7.5cm]{brique1}{img/brique1}
\pgfputat{\pgfxy(2.5,3)}{\pgfbox[center,center]{\pgfuseimage{brique1}}}
}
\only<2>{
\pgfdeclareimage[interpolate=true,height=7.5cm]{brique2}{img/brique2}
\pgfputat{\pgfxy(2.5,3)}{\pgfbox[center,center]{\pgfuseimage{brique2}}}
}
\only<3>{
\pgfdeclareimage[interpolate=true,height=7.5cm]{brique3}{img/brique3}
\pgfputat{\pgfxy(2.5,3)}{\pgfbox[center,center]{\pgfuseimage{brique3}}}
}
\only<4>{
\pgfdeclareimage[interpolate=true,height=7.5cm]{point1}{img/point1}
\pgfputat{\pgfxy(2.5,3)}{\pgfbox[center,center]{\pgfuseimage{point1}}}
}
\only<5>{
\pgfdeclareimage[interpolate=true,height=7.5cm]{point2}{img/point2}
\pgfputat{\pgfxy(2.5,3)}{\pgfbox[center,center]{\pgfuseimage{point2}}}
}
\only<6>{
\pgfdeclareimage[interpolate=true,height=7.5cm]{point3}{img/point3}
\pgfputat{\pgfxy(2.5,3)}{\pgfbox[center,center]{\pgfuseimage{point3}}}
}
\end{pgfpicture}
\end{column}
\end{columns}

\end{frame}

\begin{frame}
\frametitle{Comment l'industriel doit-il utiliser ces informations~?}
\begin{exampleblock}{Réalisation d'une future estimation}
\begin{itemize}[<+-| alert@+>]
\item[$\to$] L'industriel s'imagine être le \textbf{jour~J} dans la situation où $\mu^\bullet=\mu^\Sim=0.15$ (juste pas le marché)
\item[$\to$] Il prend alors conscience que ce qui peut lui arriver \textbf{le jour~J}, c'est équivalent (ou presque) à~:
\begin{enumerate}
\item Choisir au hasard une brique (i.e un des $m$ $\Est{\mu^\Sim}{y_{[j]}}$)
\item Choisir au hasard un point sous la ``courbe $\mathcal{N}(\mu^\Sim,\frac{\sigma_\Sim}{\sqrt{n}})$'' associé à son abscisse représentant une réalisation au hasard de $\Est{\mu^\Sim}{Y}$ choisie parmi une infinité. 
\end{enumerate}
\item[$\Rightarrow$] Il voit clairement  la ``courbe $\mathcal{N}(\mu^\Sim,\frac{\sigma_\Sim}{\sqrt{n}})$'' comme un empilement d'une infinité de briques (``devenues des points'') associées à une infinité de réalisations possibles  de $\Est{\mu^\Sim}{Y}$.
\end{itemize}
\end{exampleblock}

\end{frame}

\begin{frame}
\frametitle{Approche Expérimentale versus Approche Classique}
\begin{exampleblock}{}
\begin{itemize}[<+-| alert@+>]
\item[] Moyenne des $m=10000$  réalisations $\Est{\mu^\Sim}{y_{[j]}}$ de $\Est{\mu^\Sim}{Y}$
\item[$\simeq$]  Moyenne d'une infinité de  réalisations $\Est{\mu^\Sim}{y_{[j]}}$ de $\Est{\mu^\Sim}{Y}$
\item[$=$] Espérance $\EEE{\Est{\mu^\Sim}{Y}}$
\item[$=$] Le paramètre $\mu^\Sim$
\end{itemize}
\end{exampleblock}

\begin{exampleblock}<5->{}
\begin{itemize}[<+-| alert@+>]
\item[] Variance des $m=10000$  réalisations $\Est{\mu^\Sim}{y_{[j]}}$ de $\Est{\mu^\Sim}{Y}$
\item[$\simeq$]  Variance d'une infinité de  réalisations $\Est{\mu^\Sim}{y_{[j]}}$ de $\Est{\mu^\Sim}{Y}$
\item[$=$] Variance $\VVV{\Est{\mu^\Sim}{Y}}$
\item[$=$] $\frac{\sigma^2_\Sim}{n}$
\end{itemize}
\end{exampleblock}
\end{frame}

\begin{frame}
\frametitle{Risque d'erreur de première espèce - Produit A ($\mu^\Sim=p^A$)}
\begin{exampleblock}{}
\begin{itemize}[<+-| alert@+>]
\item[] Proportion parmi les $m=10000$  estimations $\Est{p^A}{y_{[j]}}$ de $p^A=15\%$ qui sont supérieures à $p_{lim,5\%}^+=16.86\%$ ($=517/10000$)
\item[$=$] Proportion parmi les $m=10000$  estimations $\Est{\delta_{p^A,15\%}}{y_{[j]}}$ de $\delta_{p^A,15\%}=0$ qui sont supérieures à $\delta_{lim,5\%}^+\simeq 1.6449$ ($=517/10000$)
\item[$\simeq$] Proportion parmi une infinité d'estimations $\Est{p^A}{y_{[j]}}$ de $p^A=15\%$  qui sont supérieures à $16.86\%$
\item[$=$] $\mathbb{P}_{p^A=15\%} \left( \Est{p^A}{Y} > 16.86\%\right)$
\item[$=$] $\mathbb{P}_{\delta_{p^A,15\%}=0} \left( \Est{\delta_{p^A,15\%}}{Y} > 1.6449\right)$
\item[$=$] $\mathbb{P}_{H_0} \left( \mbox{ Accepter } H_1\right)$
\item[$\simeq$] $\alpha=5\%$.
\end{itemize}
\end{exampleblock}

\end{frame}

\begin{frame}
\frametitle{Risque d'erreur de première espèce - Produit B ($\mu^\Sim=\mu^B$)}
\begin{exampleblock}{}
\begin{itemize}[<+-| alert@+>]
\item[] Proportion parmi les $m=10000$ estimations $\Est{\delta_{\mu^B,0.15}}{y_{[j]}}$  de $\delta_{\mu^B,0.15}=0$ qui sont supérieures à $\delta_{lim,5\%}^+\simeq 1.6449$ ($=497/10000$). %inventé
\item[$\simeq$] Proportion parmi une infinité d'estimations $\Est{\delta_{\mu^B,0.15}}{y_{[j]}}$  de $\delta_{\mu^B,0.15}=0$  qui sont supérieures à $1.6449$.
\item[$=$] $\mathbb{P}_{\delta_{\mu^B,0.15}=0} \left( \Est{\delta_{\mu^B,0.15}}{Y} > 1.6449\right)$
\item[$=$] $\mathbb{P}_{H_0} \left( \mbox{ Accepter } H_1\right)$
\item[$\simeq$] $\alpha=5\%$.
\end{itemize}
\end{exampleblock}
\end{frame}




\end{document}


