
\documentclass[11pt]{beamer}
%Packages
\usepackage{multirow}
\usepackage{graphicx}
\usepackage[utf8x]{inputenc}
\usepackage{aeguill}
\usepackage{amssymb}
\usepackage[french]{babel}
\usepackage{pgf,pgfarrows,pgfnodes}
\usepackage{xmpmulti}
\usepackage{multimedia}
\usepackage{bbm}
\usepackage{bm}
\usepackage{mathrsfs,dsfont}
\usepackage{xcolor}
\usepackage{colortbl}
\usepackage{longtable}
\usepackage[T1]{fontenc}
\usepackage[scaled]{beramono}
\usepackage{float}
\usepackage{xkeyval,calc,listings,tikz}
\usepackage{slashbox}
\usepackage{fancyvrb}

%Preamble

\input Cours/cqlsInclude
\input Cours/testInclude


\mode<article>{\usepackage{fullpage}}
\usefonttheme{structureitalicserif}


\definecolor{VertFonce}{rgb}{0,.4,.0}

\mode<presentation>
{
  %\usetheme{Warsaw}
  % or ...
\usetheme{Boadilla}

  \setbeamercovered{transparent=5}
  % or whatever (possibly just delete it)
}
%\setbeamercovered{dynamic}


\subject{Talks}

\AtBeginSection[]
{
  
\begin{frame}<beamer>
	\frametitle{Plan}
    \tableofcontents[currentsection,currentsubsection]
\end{frame}

}


\definecolor{show}{rgb}{0.59,0.29,0.59}

\setbeamercovered{invisible}
\newcommand{\Sim}{{\star}}
\newcommand{\ok}{ \textcolor{green}{\large$\surd$}}
\newcommand{\nok}{ \textcolor{red}{\large X}}


\usetikzlibrary{arrows,%
  calc,%
  fit,%
  patterns,%
  plotmarks,%
  shapes.geometric,%
  shapes.misc,%
  shapes.symbols,%
  shapes.arrows,%
  shapes.callouts,%
  shapes.multipart,%
  shapes.gates.logic.US,%
  shapes.gates.logic.IEC,%
  er,%
  automata,%
  backgrounds,%
  chains,%
  topaths,%
  trees,%
  petri,%
  mindmap,%
  matrix,%
  calendar,%
  folding,%
  fadings,%
  through,%
  positioning,%
  scopes,%
  decorations.fractals,%
  decorations.shapes,%
  decorations.text,%
  decorations.pathmorphing,%
  decorations.pathreplacing,%
  decorations.footprints,%
  decorations.markings,%
  shadows}
\tikzset{
  every plot/.style={prefix=plots/pgf-},
  shape example/.style={
    color=black!30,
    draw,
    fill=yellow!30,
    line width=.5cm,
    inner xsep=2.5cm,
    inner ysep=0.5cm}
}

\definecolor{darkgreen}{rgb}{0,.4,.0}


\definecolor{darkgreen}{rgb}{0,.4,.0}

%Styles

%Title

\beamertemplateshadingbackground{green!50}{yellow!50}
\title[Problématiques Produits A et B]
{Cours de Statistiques Inférentielles}
\author{CQLS~: cqls@upmf-grenoble.fr}
\date{\today}

\begin{document}
\maketitle


%\begin{frame}
%  \titlepage
%\end{frame}



\pgfdeclareimage[width=11cm,height=5cm,interpolate=true]{alf2}{img/alf2}

\pgfdeclareimage[width=11cm,height=5cm,interpolate=true]{alf7}{img/alf7}

\pgfdeclareimage[width=11cm,height=5cm,interpolate=true]{alf8}{img/alf8}









\pgfdeclareimage[width=11cm,height=5cm,interpolate=true]{dict}{img/dict}

\pgfdeclareimage[width=11cm,height=5cm,interpolate=true]{dict2}{img/dict2}







\pgfdeclareimage[width=11cm,height=5cm,interpolate=true]{hetero2}{img/hetero2}

\pgfdeclareimage[width=11cm,height=5cm,interpolate=true]{hetero7}{img/hetero7}






%\beamertemplateshadingbackground{green!50}{yellow!50}
\begin{frame}<1->
\setbeamercolor{header}{fg=black,bg=blue!40!white}
 \hspace*{2.5cm}\begin{beamerboxesrounded}[width=6cm,shadow=true,lower=header]{}
  \pgfsetxvec{\pgfpoint{6cm}{0cm}}
\pgfsetyvec{\pgfpoint{0cm}{0.5cm}}
\begin{pgfpicture}{0cm}{0cm}{6cm}{0.5cm}

  \only<1-4>{
\pgfputat{\pgfxy(0.5,0.5)}{\pgfbox[center,center]{\textbf{\large Exemple compétence Alfred}}}}
\only<5-6>{
\pgfputat{\pgfxy(0.5,0.5)}{\pgfbox[center,center]{\textbf{\large Problématique de la dictée}}}}
\only<7-8>{
\pgfputat{\pgfxy(0.5,0.5)}{\pgfbox[center,center]{\textbf{\large Hétérogénéité des notes}}}}

  \end{pgfpicture}

\end{beamerboxesrounded}

\setbeamercolor{postit}{fg=black,bg=green!40!white}
%\begin{beamercolorbox}[sep=1em,wd=12cm]{postit}
\begin{beamerboxesrounded}[shadow=true,lower=postit]{}
\pgfsetxvec{\pgfpoint{11cm}{0cm}}
\pgfsetyvec{\pgfpoint{0cm}{2.1cm}}
\begin{pgfpicture}{0cm}{0cm}{11cm}{2.1cm}

\only<1-2>{
\pgfputat{\pgfxy(0.5,1.0)}{\pgfbox[center,top]{\begin{minipage}{11cm}
\textbf{Assertion d'intérêt}~: Alfred est compétent $\Leftrightarrow$ $\mathbf{H_1}:\sigma_A^2<0.1$.\\
\textbf{Décision} (au vu des $n=20$ données)~: \\
\hspace*{1cm}Accepter $\mathbf{H_1}$  si
\only<1>{${\color<1>{orange}\Est{\delta_{\sigma_A^2,0.1}}{y^A}}<{\color<1>{darkgreen}\delta_{lim,\alpha}^-}$}
\only<2>{${\color<2>{orange}p-valeur(gauche)}<{\color<2>{darkgreen}\alpha}$}\end{minipage}}}}
\only<3-4>{
\pgfputat{\pgfxy(0.5,1.0)}{\pgfbox[center,top]{\begin{minipage}{11cm}
\textbf{Question}~: Peut-on pour autant plutôt penser au vu de ce même jeu de données qu'Alfred n'est pas compétent (i.e. $\mathbf{H_1}:\sigma^2_A>0.1$)~?\\
\textbf{Réponse}~: p-valeur droite = \only<3>{?}\only<4>{1-(p-valeur gauche)=1-#R{round(pchisq(deltaEst.H0.alf,19)*100,2)}\%=#R{round(100*(1-pchisq(deltaEst.H0.alf,19)),2)}\%}\\
\visible<4>{car \textbf{la somme des p-valeurs droite et gauche est égale à 1!}}\end{minipage}}}}
\only<5-6>{
\pgfputat{\pgfxy(0.5,1.0)}{\pgfbox[center,top]{\begin{minipage}{11.3cm}
\textbf{Assertion d'intérêt}~: Il y a un effet sur le niveau des bacheliers en orthographe $\Leftrightarrow$ $\mathbf{H_1}:\mu^D \neq 6.3$.\\
\textbf{Décision} (au vu des $n=25$ données)~: \\
Accepter $\mathbf{H_1}$  si
{\small
\only<5>{${\color<5>{orange}\Est{\delta_{\mu^D,6.3}}{y^D}}<{\color<5>{darkgreen}\delta_{lim,\alpha/2}^-}$ ou ${\color<5>{orange}\Est{\delta_{\mu^D,6.3}}{y^D}}>{\color<5>{darkgreen}\delta_{lim,\alpha/2}^+}$ }
\only<6>{${\color<6>{orange}\mbox{p-valeur (bi)}\!=\!2\!\times\!\mbox{min(p-valeur gauche,p-valeur droite)}}\!<\!{\color<6>{darkgreen}\alpha}$}
}\end{minipage}}}}
\only<7-8>{
\pgfputat{\pgfxy(0.5,1.0)}{\pgfbox[center,top]{\begin{minipage}{11.3cm}
\textbf{Assertion d'intérêt}~: Les variances des notes entre les sections C et D sont différentes $\Leftrightarrow$ $\mathbf{H_1}:d_{\sigma^2}\neq 0.$\\
\textbf{Décision} (au vu des données)~: \\
Accepter $\mathbf{H_1}$  si
{\small
\only<7>{${\color<7>{orange}\Est{\delta_{d_{\sigma^2},0}}{y^C,y^D}}<{\color<7>{darkgreen}\delta_{lim,\alpha/2}^-}$ ou ${\color<7>{orange}\Est{\delta_{d_{\sigma^2},0}}{y^C,y^D}}>{\color<7>{darkgreen}\delta_{lim,\alpha/2}^+}$ }
\only<8>{${\color<8>{orange}\mbox{p-valeur (bi)}\!=\!2\!\times\!\mbox{min(p-valeur gauche,p-valeur droite)}}\!<\!{\color<8>{darkgreen}\alpha}$}
}\end{minipage}}}}

\end{pgfpicture}

\end{beamerboxesrounded}
%\end{beamercolorbox}

\setbeamercolor{postex}{fg=black,bg=yellow!50!white}
%\begin{beamercolorbox}[sep=1em,wd=12cm]{postex}
\begin{beamerboxesrounded}[shadow=true,lower=postex]{}
\pgfsetxvec{\pgfpoint{11cm}{0cm}}
\pgfsetyvec{\pgfpoint{0cm}{5cm}}
\begin{pgfpicture}{0cm}{0cm}{11cm}{5cm}

\only<1>{
\pgfputat{\pgfxy(0.05,0.0)}{\pgfbox[left,bottom]{\pgfuseimage{alf2}}}}
\only<2-3>{
\pgfputat{\pgfxy(0.05,0.0)}{\pgfbox[left,bottom]{\pgfuseimage{alf7}}}}
\only<4>{
\pgfputat{\pgfxy(0.05,0.0)}{\pgfbox[left,bottom]{\pgfuseimage{alf8}}}}
\only<5>{
\pgfputat{\pgfxy(0.05,0.0)}{\pgfbox[left,bottom]{\pgfuseimage{dict}}}}
\only<6>{
\pgfputat{\pgfxy(0.05,0.0)}{\pgfbox[left,bottom]{\pgfuseimage{dict2}}}}
\only<7>{
\pgfputat{\pgfxy(0.05,0.0)}{\pgfbox[left,bottom]{\pgfuseimage{hetero2}}}}
\only<8>{
\pgfputat{\pgfxy(0.05,0.0)}{\pgfbox[left,bottom]{\pgfuseimage{hetero7}}}}

\end{pgfpicture}

\end{beamerboxesrounded}
%\end{beamercolorbox}
\begin{tikzpicture}[remember picture,overlay]
  \node [rotate=30,scale=10,text opacity=0.05]
    at (current page.center) {CQLS};
\end{tikzpicture}
\end{frame}





\end{document}


