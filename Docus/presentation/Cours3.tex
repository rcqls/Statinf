
\documentclass[11pt]{beamer}
%Packages
\usepackage{multirow}
\usepackage{graphicx}
\usepackage[utf8x]{inputenc}
\usepackage{aeguill}
\usepackage{amssymb}
\usepackage[french]{babel}
\usepackage{pgf,pgfarrows,pgfnodes}
\usepackage{xmpmulti}
\usepackage{multimedia}
\usepackage{bbm}
\usepackage{bm}
\usepackage{mathrsfs,dsfont}
\usepackage{xcolor}
\usepackage{colortbl}
\usepackage{longtable}
\usepackage[T1]{fontenc}
\usepackage[scaled]{beramono}
\usepackage{float}
\usepackage{xkeyval,calc,listings,tikz}
\usepackage{fancyvrb}

%Preamble

\input Cours/cqlsInclude
\input Cours/testInclude


\mode<article>{\usepackage{fullpage}}
\usefonttheme{structureitalicserif}


\definecolor{VertFonce}{rgb}{0,.4,.0}

\mode<presentation>
{
  %\usetheme{Warsaw}
  % or ...
\usetheme{Boadilla}

  \setbeamercovered{transparent=5}
  % or whatever (possibly just delete it)
}
%\setbeamercovered{dynamic}


\subject{Talks}

\AtBeginSection[]
{
  
\begin{frame}<beamer>
	\frametitle{Plan}
    \tableofcontents[currentsection,currentsubsection]
\end{frame}

}


\definecolor{show}{rgb}{0.59,0.29,0.59}

\setbeamercovered{invisible}
\newcommand{\Sim}{{\star}}
\newcommand{\ok}{ \textcolor{green}{\large$\surd$}}
\newcommand{\nok}{ \textcolor{red}{\large X}}


\usetikzlibrary{arrows,%
  calc,%
  fit,%
  patterns,%
  plotmarks,%
  shapes.geometric,%
  shapes.misc,%
  shapes.symbols,%
  shapes.arrows,%
  shapes.callouts,%
  shapes.multipart,%
  shapes.gates.logic.US,%
  shapes.gates.logic.IEC,%
  er,%
  automata,%
  backgrounds,%
  chains,%
  topaths,%
  trees,%
  petri,%
  mindmap,%
  matrix,%
  calendar,%
  folding,%
  fadings,%
  through,%
  positioning,%
  scopes,%
  decorations.fractals,%
  decorations.shapes,%
  decorations.text,%
  decorations.pathmorphing,%
  decorations.pathreplacing,%
  decorations.footprints,%
  decorations.markings,%
  shadows}
\tikzset{
  every plot/.style={prefix=plots/pgf-},
  shape example/.style={
    color=black!30,
    draw,
    fill=yellow!30,
    line width=.5cm,
    inner xsep=2.5cm,
    inner ysep=0.5cm}
}

%Styles

%Title

\beamertemplateshadingbackground{green!50}{yellow!50}
\title[Problématiques Produits A et B]
{Cours de Statistiques Inférentielles}
\author{CQLS~: cqls@upmf-grenoble.fr}
\date{\today}

\begin{document}
\maketitle


%\begin{frame}
%  \titlepage
%\end{frame}

\section[I. C.]{Intervalle de confiance}

\begin{frame}{}
\begin{alertblock}{Motivation}
\pause
\begin{enumerate}[<+->]
\item Quelle confiance accordez-vous à deux estimations obtenues à partir de 2 echantillons de tailles respectives $n=5$ et $n=1000$~?
\item Plus généralement, quelle confiance doit-on accorder à une estimation $\Est{\theta^\bullet}{y}$ le \textbf{jour~J} selon son erreur standard $\Est{\sigma_{\widehat{\theta^\bullet}}}{y}$ plus ou moins grande. 
\item Interprétation des résultats d'un sondage avant le premier tour des élections présidentielles 2002~: Votre attitude aurait-elle été influencée si à la place d'une estimation $\Est{p^J}{y}$ (autour de $17\%$) pour le candidat Jospin, on vous avait fourni une ``fourchette" $[14.67\%,19.33\%]$.
Il paraît que cette information ne nous est pas fourni car les Français ne sauraient pas interpréter ce type de résultats. Qu'en pensez-vous~?    
\end{enumerate}
\end{alertblock}
\end{frame}

\pgfdeclareimage[width=8cm,height=4cm,interpolate=true]{normale}{img/normale}
\pgfdeclareimage[width=8cm,height=4cm,interpolate=true]{normale2}{img/normale2}

\begin{frame}[label=ic,t]
\frametitle{Construction Intervalle de Confiance et  \textbf{A. E. P.}}
 
\hspace*{0.7cm}\begin{pgfpictureboxed}{0cm}{0cm}{11cm}{5cm}
\pgfsetendarrow{\pgfarrowto}
  \pgfnodebox{Y}[stroke]{\pgfxy(0.5,4.5)}{\small$\Vect{Y}$}{2pt}{2pt}
%\onslide
\only<14->{ \pgfnodebox{y1}[stroke]{\pgfxy(0.5,3.5)}{\small$\Vect{y_{[1]}}$}{2pt}{2pt}}
%\onslide
\only<16->{ \pgfnodebox{y2}[stroke]{\pgfxy(0.5,2.5)}{\small$\Vect{y_{[2]}}$}{2pt}{2pt}}
%\onslide
\only<18->{
  \pgfputat{\pgfxy(0.5,2)}{\pgfbox[center,center]{$\vdots$}}
  \pgfnodebox{ym}[stroke]{\pgfxy(0.5,1.25)}{\small$\Vect{y_{[m]}}$}{2pt}{2pt}
}
%\onslide
\only<20>{ \pgfputat{\pgfxy(0.5,0.75)}{\pgfbox[center,center]{$\vdots$}}}
\only<14->{\pgfnodeconncurve{Y}{y1}{180}{180}{0.5cm}{0.5cm}}
\only<14,15>{\pgfnodelabel{Y}{y1}[0.75][-40pt]{\pgfbox[left,base]{$1^{\grave eme}$ réalisation}}
}
\only<16->{\pgfnodeconncurve{Y}{y2}{180}{180}{1cm}{0.5cm}}
\only<16,17>{\pgfnodelabel{Y}{y2}[0.9][-40pt]{\pgfbox[left,base]{$2^{\grave eme}$ réalisation}}
}
\only<18->{\pgfnodeconncurve{Y}{ym}{180}{180}{2cm}{0.5cm}}
\only<18,19>{\pgfnodelabel{Y}{ym}[0.95][-40pt]{\pgfbox[left,base]{$m^{\grave eme}$ réalisation}}
}
\only<2->{
\pgfnodebox{Est}[stroke]{\pgfxy(2.75,4.5)}{\small${\color<5>{blue}\Est{\theta^\bullet}{Y}}$,${\color<6>{blue}\Est{\sigma_{\widehat{\theta^\bullet}}}{Y}}$}{2pt}{2pt}
\pgfnodeconncurve{Y}{Est}{0}{180}{0cm}{0.5cm}
}
\only<15->{
\pgfnodebox{est1}[stroke]{\pgfxy(2.75,3.5)}{\small$\Est{\theta^\bullet}{y_{[1]}}$,$\Est{\sigma_{\widehat{\theta^\bullet}}}{y_{[1]}}$}{2pt}{2pt}
\pgfnodeconncurve{y1}{est1}{0}{180}{0cm}{0.5cm}
}
\only<17->{
\pgfnodebox{est2}[stroke]{\pgfxy(2.75,2.5)}{\small$\Est{\theta^\bullet}{y_{[2]}}$,$\Est{\sigma_{\widehat{\theta^\bullet}}}{y_{[2]}}$}{2pt}{2pt}
\pgfnodeconncurve{y2}{est2}{0}{180}{0cm}{0.5cm}
}
\only<19->{
\pgfputat{\pgfxy(2.75,2)}{\pgfbox[center,center]{$\vdots$}}
\pgfnodebox{estm}[stroke]{\pgfxy(2.75,1.25)}{\small$\Est{\theta^\bullet}{y_{[m]}}$,$\Est{\sigma_{\widehat{\theta^\bullet}}}{y_{[m]}}$}{2pt}{2pt}
\pgfnodeconncurve{ym}{estm}{0}{180}{0cm}{0.5cm}
}
\only<20>{
\pgfputat{\pgfxy(2.75,0.75)}{\pgfbox[center,center]{$\vdots$}}
}

\only<4->{
\pgfnodebox{Delta}[stroke]{\pgfxy(5.75,4.5)}{\small${\delta_{\widehat{\theta^\bullet},\theta^\bullet}}\From {Y}$}{2pt}{2pt}
\pgfnodeconncurve{Est}{Delta}{0}{180}{0cm}{0.5cm}
}
\only<15->{
\pgfnodebox{delta1}[stroke]{\pgfxy(5.75,3.5)}{ \small${\delta_{\widehat{\theta^\bullet},\theta^\bullet}}\From{y_{[1]}}$}{2pt}{2pt}
\pgfnodeconncurve{est1}{delta1}{0}{180}{0cm}{0.5cm}
}
\only<17->{
\pgfnodebox{delta2}[stroke]{\pgfxy(5.75,2.5)}{\small$ {\delta_{\widehat{\theta^\bullet},\theta^\bullet}}\From{y_{[2]}}$}{2pt}{2pt}
\pgfnodeconncurve{est2}{delta2}{0}{180}{0cm}{0.5cm}
}
\only<19->{
\pgfputat{\pgfxy(5.75,2)}{\pgfbox[center,center]{$\vdots$}}
\pgfnodebox{deltam}[stroke]{\pgfxy(5.75,1.25)}{\small$ {\delta_{\widehat{\theta^\bullet},\theta^\bullet}}\From{y_{[m]}}$}{2pt}{2pt}
\pgfnodeconncurve{estm}{deltam}{0}{180}{0cm}{0.5cm}
}
\only<20>{
\pgfputat{\pgfxy(5.75,0.75)}{\pgfbox[center,center]{$\vdots$}}
}
\only<8->{
\pgfnodebox{IC}[stroke]{\pgfxy(9,4.5)}{\small$[\Int{\theta^\bullet}{inf}{Y},\Int{\theta^\bullet}{sup}{Y}]$}{2pt}{2pt}
\pgfnodeconncurve{Est}{IC}{90}{90}{0.2cm}{0.2cm}
}
\only<15->{
\pgfnodebox{ic1}[stroke]{\pgfxy(9,3.5)}{\small$[\Int{\theta^\bullet}{inf}{y_{[1]}},\Int{\theta^\bullet}{sup}{y_{[1]}}]$}{2pt}{2pt}
\pgfnodeconncurve{est1}{ic1}{90}{90}{0.2cm}{0.2cm}
}
\only<17->{
\pgfnodebox{ic2}[stroke]{\pgfxy(9,2.5)}{\small$[\Int{\theta^\bullet}{inf}{y_{[2]}},\Int{\theta^\bullet}{sup}{y_{[2]}}]$}{2pt}{2pt}
\pgfnodeconncurve{est2}{ic2}{90}{90}{0.2cm}{0.2cm}
}
\only<19->{
\pgfputat{\pgfxy(9,2)}{\pgfbox[center,center]{$\vdots$}}
\pgfnodebox{icm}[stroke]{\pgfxy(9,1.25)}{\small$[\Int{\theta^\bullet}{inf} {y_{[m]}},\Int{\theta^\bullet}{sup}{y_{[m]}}]$ }{2pt}{2pt}
\pgfnodeconncurve{estm}{icm}{90}{90}{0.2cm}{0.2cm}
}
\only<20>{
\pgfputat{\pgfxy(9,0.75)}{\pgfbox[center,center]{$\vdots$}}
}

%%% images
\only<7,8>{
\pgfputat{\pgfxy(5.5,0.0)}{\pgfbox[center,bottom]{\pgfuseimage{normale}}}
}
\only<9,10>{
\pgfputat{\pgfxy(5.5,0.0)}{\pgfbox[center,bottom]{\pgfuseimage{normale2}}}
}
\end{pgfpictureboxed} 
%%%%%%%%%block
\begin{block}{} 
\only<1>{
Un Futur échantillon $\Vect{Y}$!
}\only<2,3>{
Une future estimation $\Est{\theta^\bullet}{Y}
\onslide<3>{\SuitApprox\mathcal{N}(\theta^\bullet,{\color{red}\sigma_{\widehat{\theta^\bullet}}})$ {\color{red}\textbf{inconnue}}}\\
Sa future erreur standard  $\Est{\sigma_{\widehat{\theta^\bullet}}}{Y}
\onslide<3>{\SuitApprox\mathcal{N}({\color{red}\sigma_{\widehat{\theta^\bullet}}},{\color{red}\sigma_{\widehat{\sigma_{\widehat{\theta^\bullet}}}}})$ {\color{red}\textbf{inconnue}}}
}\only<4-8>{
Future mesure d'écart standardisée~:\\


\centerline{$\displaystyle \delta_{\widehat{\theta^\bullet},\theta^\bullet}\From{Y}=\frac{{\color<5>{blue}\Est{\theta^\bullet}{Y}}-{\color<5-6>{red}\theta^\bullet}}{\color<6>{blue}\Est{\sigma_{\widehat{\theta^\bullet}}}{Y}} \onslide<7,8>{\SuitApprox \color{blue}\mathcal{N}(0,1) \mbox{ connue~!}} 
$}
\onslide<8>{$\to$ détermination  de $[\Int{\theta^\bullet}{inf} {Y},\Int{\theta^\bullet}{sup}{Y}]$~?}
}\only<9>{
\[
1-\alpha\simeq\Prob{-{\color{blue}\delta^+_{lim,\frac{\alpha}2}}\leq \delta_{\widehat{\theta^\bullet},\theta^\bullet}(\Vect{Y})
\leq{\color{blue}\delta^+_{lim,\frac{\alpha}2}}}\mbox{ avec }{\color{blue}\delta^+_{lim,\frac{\alpha}2}}\left\{\begin{array}{l}=q_{1-\frac\alpha2}(\mathcal{N}(0,1))\\ \simeq {\color{purple}1.96} \mbox{ si }{\color{purple}\alpha=5\%}\end{array}\right.
\]
}\only<10>{
\[
1-\alpha\simeq\Prob{{\color{blue}-\delta^+_{lim,\frac{\alpha}2}}\leq
\frac{{\color{blue}\Est{\theta^\bullet}{Y}}-{\color{red}\theta^\bullet}}{\color{blue}\Est{\sigma_{\widehat{\theta^\bullet}}}{Y}}
\leq{\color{blue}\delta^+_{lim,\frac{\alpha}2}}}
\]
}\only<11>{
\[
1-\alpha\simeq\Prob{-\delta^+_{lim,\frac{\alpha}2}\times {\Est{\sigma_{\widehat{\theta^\bullet}}}{Y}}\leq
\Est{\theta^\bullet}{Y}-{\color{red}\theta^\bullet}
\leq\delta^+_{lim,\frac{\alpha}2}\times {\Est{\sigma_{\widehat{\theta^\bullet}}}{Y}}
}
\]
}\only<12>{
\[
1-\alpha\simeq\Prob{-\delta^+_{lim,\frac{\alpha}2}\times {\Est{\sigma_{\widehat{\theta^\bullet}}}{Y}}\leq
{\color{red}\theta^\bullet}-\Est{\theta^\bullet}{Y}
\leq\delta^+_{lim,\frac{\alpha}2}\times {\Est{\sigma_{\widehat{\theta^\bullet}}}{Y}}
}
\]
}\only<13>{
\[
1-\alpha\simeq \mathbb{P} \Big(
\underbrace{{\color{blue}\Est{\theta^\bullet}{Y}}\!-\!{\color{blue}\delta^+_{lim,\frac{\alpha}2}}\!\times\!
{\color{blue}\Est{\sigma_{\widehat{\theta^\bullet}}}{Y}}}_{\color{purple}\Int{\theta^\bullet}{\inf}{Y}}\!\leq\!{\color{red}\theta^\bullet}\!\leq\!\underbrace{{\color{blue}\Est{\theta^\bullet}{Y}}\!+\!{\color{blue}\delta^+_{lim,\frac{\alpha}2}}\times
{\color{blue}\Est{\sigma_{\widehat{\theta^\bullet}}}{Y}}}_{\color{purple}\Int{\theta^\bullet}{\sup}{Y}} \Big)
\]
}\only<14>{
Construisons un premier échantillon $\Vect{y_{[1]}}$ !
}\only<15>{\small L'intervalle de confiance calculé à partir de $\Vect{y_{[1]}}$ est \textbf{``bon"} dans sa mission ssi\\
\centerline{\fbox{$\theta^\bullet\in[\Int{\theta^\bullet}{\inf}{y_{[1]}},\Int{\theta^\bullet}{\sup}{y_{[1]}}]$}
 $\Leftrightarrow$ \fbox{$\delta_{\widehat{\theta^\bullet},\theta^\bullet}(\Vect{y_{[1]}})\in [-\delta^+_{lim,\frac{\alpha}2},\delta^+_{lim,\frac{\alpha}2}]$}
}}\only<16>{
Construisons un deuxième échantillon $\Vect{y_{[2]}}$ !
}\only<17>{\small L'intervalle de confiance calculé à partir de $\Vect{y_{[2]}}$ est \textbf{``bon"} dans sa mission ssi\\
\centerline{\fbox{$\theta^\bullet\in[\Int{\theta^\bullet}{\inf}{y_{[2]}},\Int{\theta^\bullet}{\sup}{y_{[2]}}]$}
 $\Leftrightarrow$ \fbox{$\delta_{\widehat{\theta^\bullet},\theta^\bullet}(\Vect{y_{[2]}})\in [-\delta^+_{lim,\frac{\alpha}2},\delta^+_{lim,\frac{\alpha}2}]$}
}}\only<18>{
Construisons un $m^{\grave eme}$ échantillon $\Vect{y_{[m]}}$ !
}\only<19>{\small L'intervalle de confiance calculé à partir de $\Vect{y_{[m]}}$ est \textbf{``bon"} dans sa mission ssi\\
\centerline{\fbox{$\theta^\bullet\in[\Int{\theta^\bullet}{\inf}{y_{[m]}},\Int{\theta^\bullet}{\sup}{y_{[m]}}]$}
 $\Leftrightarrow$ \fbox{$\delta_{\widehat{\theta^\bullet},\theta^\bullet}(\Vect{y_{[m]}})\in [-\delta^+_{lim,\frac{\alpha}2},\delta^+_{lim,\frac{\alpha}2}]$}
}}\only<20>{
\textbf{\underline Interprétation par l'A.E.P.}~: nous imaginons disposer d'une {\color{red}infinité d'intervalles de confiance} dont une proportion ($\simeq$) {\color{purple}$1-\alpha$} (i.e. niveau de confiance) sont {\color{purple}``bons"}, i.e. contiennent le paramètre $\theta^\bullet$ inconnu.\\ 
\textbf{\underline{Obtention de $[\Int{\theta^\bullet}{\inf}{y},\Int{\theta^\bullet}{\sup}{y}]$ le jour~J}}~: équivalent à un choix au hasard d'{\color{blue}un unique intervalle de confiance} parmi cette infinité!!!
}

\end{block}
\end{frame}


\begin{frame}
\frametitle{Qualité d'un I.C.~: bon niveau de confiance $+$ assez petit }
\vspace*{-.2cm}
\begin{exampleblock}{}
\pause
\begin{enumerate}
\item<2-> {\small niveau de confiance $1-\alpha={\color{blue}100\%}$~: }
\only<2>{\small \begin{itemize}
\item $n=10$~: $[?,?]$, $[?,?]$, $[?,?]$,$\cdots$
\item $n=100$~: $[?,?]$, $[?,?]$, $[?,?]$,$\cdots$
\item $n=1000$~: $[?,?]$, $[?,?]$, $[?,?]$,$\cdots$
\end{itemize}
}\only<3->{\small \begin{itemize}
\item $n=10$~: $[{\color<5>{red}0,+\infty}]$, $[{\color<5>{red}0,+\infty}]$,$[{\color<5>{red}0,+\infty}]$ ,$\cdots$
\item $n=100$~:   $[{\color<5>{red}0,+\infty}]$, $[{\color<5>{red}0,+\infty}]$,$[{\color<5>{red}0,+\infty}]$ ,$\cdots$
\item $n=1000$~:  $[{\color<5>{red}0,+\infty}]$, $[{\color<5>{red}0,+\infty}]$,$[{\color<5>{red}0,+\infty}]$ ,$\cdots$
\end{itemize}
}
\item<4->  {\small niveau de confiance $1-\alpha={\color{blue}95\%}$~: 
\begin{itemize}
\item $n=10$~: $[{\color<5>{red}-0.118,0.718}]$, $[{\color<5>{red}-0.096,0.296}]$, $[{\color<5>{red}-0.199,0.999}]$,$\cdots$
\item $n=100$~: $[{\color<5>{red}0.129,0.311}]$, $[{\color<5>{red}0.077,0.263}]$, $[{\color<5>{red}0.101,0.319}]$,$\cdots$
\item $n=1000$~: $[{\color<5>{red}0.197,0.263}]$, $[{\color<5>{red}0.185,0.247}]$, $[{\color<5>{red}0.162,0.222}]$,$\cdots$
\end{itemize}
}
\item<4->  {\small niveau de confiance $1-\alpha={\color{blue}50\%}$~: 
\begin{itemize}
 \item $n=10$~: $[{\color<5>{red}0.156,0.444}]$, $[0.033,0.167]$, $[{\color<5>{red}0.194,0.606}]$,$\cdots$
\item $n=100$~: $[{\color<5>{red}0.189,0.251}]$, $[{\color<5>{red}0.138,0.202}]$, $[{\color<5>{red}0.173,0.247}]$,$\cdots$
\item $n=1000$~: $[0.219,0.241]$, $[0.205,0.227]$, $[{\color<5>{red}0.182,0.202}]$,$\cdots$
\end{itemize}
}
\item<4->  {\small niveau de confiance $1-\alpha={\color{blue}10\%}$~: 
\begin{itemize}
 \item $n=10$~: $[0.273,0.327]$, $[0.087,0.113]$, $[0.362,0.438]$,$\cdots$
\item $n=100$~: $[0.214,0.226]$, $[0.164,0.176]$, $[0.203,0.217]$,$\cdots$
\item $n=1000$~: $[0.228,0.232]$, $[0.214,0.218]$, $[0.19,0.194]$,$\cdots$
\end{itemize}
}
\end{enumerate}
\end{exampleblock}
\end{frame}




\begin{frame}
\frametitle{Vérification expérimentale}
\hspace*{-0.2cm}\only<1>{
\begin{tabular}{|c|c|c|c|c|}\hline
\multicolumn{5}{|c|}{
Echantillons générés avec
$\mu^\bullet=0.2$
 
[~(néanmoins inconnu du statisticien)
} \\\hline
j & $\Est{\mu^\bullet}{y_{[j]}}$ & $\Int{\mu^\bullet}{\inf}{y_{[j]}}$ & $\Int{\mu^\bullet}{\sup}{y_{[j]}}$ & {\scriptsize$\mu^\bullet \!\in\! [\!\Int{\mu^\bullet}{\inf}{y_{[j]}}\!,\!\Int{\mu^\bullet}{\sup}{y_{[j]}}\!]\!$?} \\\hline
$\vdots$ & $\vdots$ &$\vdots$ & $\vdots$ &$\vdots$ \\

 7
&
0.22
& 
0.187
& 
0.253
&
1
\\

 8
&
0.222
& 
0.19
& 
0.254
&
1
\\

 9
&
0.186
& 
0.157
& 
0.215
&
1
\\

 {\color{red}10}
&
{\color{red}0.216}
& 
{\color{red}0.184}
& 
{\color{red}0.248}
&
{\color{red}1}
\\

 11
&
0.198
& 
0.168
& 
0.228
&
1
\\

 12
&
0.228
& 
0.195
& 
0.261
&
1
\\

 13
&
0.198
& 
0.168
& 
0.228
&
1
\\

$\vdots$ & $\vdots$ & $\vdots$ & $\vdots$ & $\vdots$\\\hline
\multicolumn{4}{|r|}{${\scriptsize \Prob{\mu^\bullet\in[\Int{\mu^\bullet}{\inf}{Y},\Int{\mu^\bullet}{\sup}{Y}]}} \simeq $Taux de succès$=$}
&
95.14\%
\\\hline
\end{tabular}

}\only<2>{
\begin{tabular}{|c|c|c|c|c|}\hline
\multicolumn{5}{|c|}{
Echantillons générés avec
$\mu^\bullet=0.2$
 
[~(néanmoins inconnu du statisticien)
} \\\hline
j & $\Est{\mu^\bullet}{y_{[j]}}$ & $\Int{\mu^\bullet}{\inf}{y_{[j]}}$ & $\Int{\mu^\bullet}{\sup}{y_{[j]}}$ & {\scriptsize$\mu^\bullet \!\in\! [\!\Int{\mu^\bullet}{\inf}{y_{[j]}}\!,\!\Int{\mu^\bullet}{\sup}{y_{[j]}}\!]\!$?} \\\hline
$\vdots$ & $\vdots$ &$\vdots$ & $\vdots$ &$\vdots$ \\

 7231
&
0.235
& 
0.2
& 
0.27
&
0
\\

 7232
&
0.194
& 
0.165
& 
0.223
&
1
\\

 7233
&
0.192
& 
0.165
& 
0.219
&
1
\\

 {\color{red}7234}
&
{\color{red}0.195}
& 
{\color{red}0.165}
& 
{\color{red}0.225}
&
{\color{red}1}
\\

 7235
&
0.18
& 
0.151
& 
0.209
&
1
\\

 7236
&
0.181
& 
0.153
& 
0.209
&
1
\\

 7237
&
0.191
& 
0.161
& 
0.221
&
1
\\

$\vdots$ & $\vdots$ & $\vdots$ & $\vdots$ & $\vdots$\\\hline
\multicolumn{4}{|r|}{${\scriptsize \Prob{\mu^\bullet\in[\Int{\mu^\bullet}{\inf}{Y},\Int{\mu^\bullet}{\sup}{Y}]}} \simeq $Taux de succès$=$}
&
95.14\%
\\\hline
\end{tabular}

}\only<3->{
\begin{tabular}{|c|c|c|c|c|}\hline
\multicolumn{5}{|c|}{
Echantillons générés avec
$\mu^\bullet=0.2$
 
[~(néanmoins inconnu du statisticien)
} \\\hline
j & $\Est{\mu^\bullet}{y_{[j]}}$ & $\Int{\mu^\bullet}{\inf}{y_{[j]}}$ & $\Int{\mu^\bullet}{\sup}{y_{[j]}}$ & {\scriptsize$\mu^\bullet \!\in\! [\!\Int{\mu^\bullet}{\inf}{y_{[j]}}\!,\!\Int{\mu^\bullet}{\sup}{y_{[j]}}\!]\!$?} \\\hline
$\vdots$ & $\vdots$ &$\vdots$ & $\vdots$ &$\vdots$ \\

 1284
&
0.197
& 
0.165
& 
0.229
&
1
\\

 1285
&
0.224
& 
0.191
& 
0.257
&
1
\\

 1286
&
0.155
& 
0.128
& 
0.182
&
0
\\

 {\color{red}1287}
&
{\color{red}0.203}
& 
{\color{red}0.172}
& 
{\color{red}0.234}
&
{\color{red}1}
\\

 1288
&
0.218
& 
0.188
& 
0.248
&
1
\\

 1289
&
0.206
& 
0.175
& 
0.237
&
1
\\

 1290
&
0.178
& 
0.152
& 
0.204
&
1
\\

$\vdots$ & $\vdots$ & $\vdots$ & $\vdots$ & $\vdots$\\\hline
\multicolumn{4}{|r|}{${\scriptsize \Prob{\mu^\bullet\in[\Int{\mu^\bullet}{\inf}{Y},\Int{\mu^\bullet}{\sup}{Y}]}} \simeq $Taux de succès$=$}
&
95.14\%
\\\hline
\end{tabular}


\noindent \textbf{\only<3-4,6,8>{Question}\only<5,7,9>{Réponse}}~: $\mathbb{P}\Big(\mu^\bullet \in [\only<3-4>{{\color{red}0.172,0.234}}\only<5>{\!\underbrace{\Int{\mu^\bullet}{\inf}{\color{purple}y_{[1287]}}}_{\color{red}0.172}\!,\!\underbrace{\Int{\mu^\bullet}{\sup}{\color{purple}y_{[1287]}}}_{\color{red}0.234}\!}\only<6-7>{\!\Int{\mu^\bullet}{\inf}{\color<7>{purple}y}\!,\!\Int{\mu^\bullet}{\sup}{\color<7>{purple}y}\!}\only<8-9>{\!\Int{\mu^\bullet}{\inf}{\color<9>{purple}Y}\!,\!\Int{\mu^\bullet}{\sup}{\color<9>{purple}Y}\!}]\Big)=\only<3,6,8>{\mbox{ }?}\only<4>{95\%?}\only<5>{ {\color{red}100\%}}\only<7>{{\color{purple}0\%} \mbox{  ou } {\color{purple}100\%}}\only<9>{\color{purple}95\%}$
}
\end{frame}

\begin{frame}
\frametitle{Application: Salaire Juste}
Considérons (le \textbf{jour~J}) disposer d'un échantillon $\Vect{y}$ des Salaires Justes de $n$ individus du Pays. Ce jeu de données est noté \texttt{yJ} en \texttt{R}. Les formules des {\color<2-3>{purple}intervalle}s de {\color<2-3>{purple}confiance à $95\%$} pour {\color<2>{purple}$\mu^{J}$} et {\color<3>{purple}$\sigma^2_{J}$} s'obtiennent très facilement~:

\only<2>{\begin{eqnarray*}
\left\{ \begin{array}{c}\Int{\mu^{J}}{\inf}{y}\\\Int{\mu^{J}}{\sup}{y}\end{array}\right.
&=& {\color{blue}\Est{\mu^{J}}{y}}+{\color{purple}\left\{\begin{array}{c}-1\\1 \end{array}\right\}} 
\times{\color{red}\delta^+_{lim,\alpha/2}}{\color{blue}\Est{\sigma_{\widehat\mu^{J}}}{y}}\\
&\NotR&\mathtt{{\color{blue}mean(yJ)}+{\color{purple}c(-1,1)}*{\color{red}qnorm(.975)}*{\color{blue}seMean(yJ)}}\\
&\simeq& [100.06,101.36]
\end{eqnarray*}}
\only<3>{\begin{eqnarray*}
\left\{ \begin{array}{c}\Int{\sigma^2_{J}}{\inf}{y}\\\Int{\sigma^2_{J}}{\sup}{y}\end{array}\right.
&=& {\color{blue}\Est{\sigma^2_{J}}{y}}+{\color{purple}\left\{\begin{array}{c}-1\\1 \end{array}\right\}} 
\times{\color{red}\delta^+_{lim,\alpha/2}}{\color{blue}\Est{\sigma_{\widehat{\sigma^2_{J}}}}{y}}\\
&\NotR&\mathtt{{\color{blue}var(yJ)}+{\color{purple}c(-1,1)}*{\color{red}qnorm(.975)}*{\color{blue}seVar(yJ)}}\\
&\simeq& [86.53,134.06]
\end{eqnarray*}}
\end{frame}






\end{document}


