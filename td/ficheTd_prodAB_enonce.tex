
\documentclass[10pt]{report}
%Packages
\usepackage{multirow}
\usepackage{graphicx}
\usepackage[utf8]{inputenc}
\usepackage[T1]{fontenc}
\usepackage{aeguill}
\usepackage{amssymb}
\usepackage[french]{babel}
\usepackage{float}
\usepackage{fichetd}
\usepackage{mathenv}
\usepackage{tikz}
\usepackage[a4paper,pdftex=true]{geometry}
\usepackage{fancyvrb}

%Preamble

\input /Users/remy/cqls/texinputs/Cours/cqlsInclude
\input /Users/remy/cqls/texinputs/Cours/testInclude


\rightmargin -2cm
\leftmargin -1cm
\topmargin -2cm
\textheight 25cm

\newcommand{\redabr}{\textit{(rédaction abrégée) }}
\newcommand{\redstd}{\textit{(rédaction standard) }}

\setcounter{chapter}{3}
%Styles

%Title

\begin{document}




\chapter{Traitement\\ des problématiques \\des produits A et B}\label{TdProdAB}
\begin{IndicList}{Indications préliminaires} 
\item \textit{Objectif}~: En pratique, on peut être spécialement intéressé par une prise de décision qui dépend de la comparaison du \textbf{paramètre d'intérêt} $\theta$ inconnu par rapport à une \textbf{valeur de référence} $\theta_0$ (fixée selon la problématique). Cette comparaison sera par la suite appelée \textbf{assertion d'intérêt}.
Ne disposant que d'une estimation $\Est{\theta}{y}$ la décision conduisant à conclure que l'assertion d'intérêt est vraie à partir de l'échantillon $\Vect{y}$ du \textbf{jour J}  ne peut pas être complètement fiable. L'objectif est de construire un outil d'aide à la décision nous garantissant un risque d'erreur de se tromper dans notre décision de valider l'assertion d'intérêt n'excédant pas une valeur que nous nous sommes fixée (généralement autour des 5\%).
\item \textit{Paramètre d'écart standardisé}~: Comparer la paramètre d'intérêt $\theta$ à une valeur de référence $\theta_0$ est strictement équivalent à comparer leur différence ou leur rapport à 0 ou 1. Dans le cadre asymptotique, l'assertion d'intérêt pourra toujours se réécrire en fonction d'un paramètre d'écart standardisé $\delta_{\theta,\theta_0}:=\frac{\cqlsbm{\theta-\theta_0}} {\sigma_{\widehat{\theta}}}$. Il est important d'apprendre à lire cette expression où le numérateur $\theta-\theta_0$ a été mis en \textbf{gras} pour souligner son rôle plus important (en termes d'information pour l'utilisateur) par rapport au dénominateur $\sigma_{\widehat{\theta}}$ ayant été introduit principalement pour des raisons techniques (mais toutefois indispensables dans la construction de l'outil d'aide à la décision).  Il est alors direct de voir que~:
\[
\mbox{\textbf{Assertion d'intérêt}}\Longleftrightarrow \left\{\begin{array}{ccccc}
  \theta<\theta_0&\Longleftrightarrow &\theta-\theta_0<0 &\Longleftrightarrow &\delta_{\theta,\theta_0}<0\\ 
  \theta>\theta_0&\Longleftrightarrow&\theta-\theta_0>0 &\Longleftrightarrow  &\delta_{\theta,\theta_0}>0\\
  \theta\neq\theta_0&\Longleftrightarrow&\theta-\theta_0\neq0 &\Longleftrightarrow  &\delta_{\theta,\theta_0}\neq0\\
  \end{array}\right\}
\]
En commentaire non prioritaire, on peut tout de même remarquer que l'interprétation du $\sigma_{\widehat{\theta}}$ dans l'expression de $\delta_{\theta,\theta_0}$ est assez naturelle~: plus l'estimation de $\theta$ est fiable, se traduisant par un $\sigma_{\widehat{\theta}}$ d'autant plus faible, plus le paramètre d'écart standardisé $\delta_{\theta,\theta_0}$ est grand et ainsi plus facile à comparer à 0.
\item \textit{Estimation du paramètre d'écart standardisé}~: Dépendant du paramètre d'intérêt $\theta$ inconnu, le paramètre d'écart standardisé $\delta_{\theta,\theta_0}$ est lui-même inconnu (en fait doublement inconnu puisque dépendant aussi de $\sigma_{\widehat{\theta}}$ inconnu). Il est facilement estimable à partir de l'échantillon $\Vect{y}$ du \textbf{jour J}. Nous l'exprimons ci-dessous à partir du ``futur" échantillon $\Vect{Y}$~:
\[
\Est{\delta_{\theta,\theta_0}}{Y}:=\frac{\Est{\theta}{Y}-\theta_0}{\Est{\sigma_{\widehat{\theta}}}{Y}}
\]
Pour mesurer les risques d'erreur de décision, nous serons tout particulièrement intéressés par la loi de probabilité de  $\Est{\delta_{\theta,\theta_0}}{Y}$ lorsque $\theta=\theta_0$. Dans ce cas très particulier, nous remarquons que lorsque $n$ est suffisamment grand~:
\[
\Est{\delta_{\theta,\theta_0}}{Y}:=\frac{\Est{\theta}{Y}-\mbox{\fbox{$\theta_0$}}}{\Est{\sigma_{\widehat{\theta}}}{Y}}=\frac{\Est{\theta}{Y}-\mbox{\fbox{$\theta$}}}{\Est{\sigma_{\widehat{\theta}}}{Y}}=:\delta_{\widehat{\theta},\theta}(\Vect{Y})\SuitApprox \mathcal{N}(0,1)
\]
où $\delta_{\widehat{\theta},\theta}(\Vect{Y})$ a été introduit au début de la fiche T.D.~\ref{TdEst}.
\end{IndicList}




\begin{exercice}[Forme des Règles de Décision pour produits~A et B] $ $
\begin{enumerate}
\item Exprimer les assertions d'intérêt pour les produits~A et~B correspondant aux lancements des produits sur le marché en fonction des paramètres d'intérêts.
\item Reécrire ces assertions d'intérêt à partir des paramètres d'écart standardisé fournis dans le formulaire de cours.
\item Proposer les formes des Règles de décision associées aux expressions précédentes des assertions d'intérêts (via paramètres d'intérêt et d'écart standardisé).   
\item  Selon ces Règles de Décision, est-il possible pour l'industriel de ne pas se tromper dans sa décision quant au lancement de chaque produit~? 

\item Exprimez les erreurs de décision éventuelles en les illustrant par des exemples de situations réelles envisageables.

Il en ressort qu'il y a 2 risques d'erreurs de décision~: 
\begin{itemize}
\item Risque d'erreur I (ou première espèce)~: risque de décider à tort l'assertion d'intérêt.
\item Risque d'erreur II (ou deuxième expèce)~: risque de ne pas décider à tort l'assertion d'intérêt.
\end{itemize}
Comment les reformuleriez-vous dans les problématiques de l'industriel~? Lequel parmi ces 2 types d'erreurs est-il plus important de controler~?

\item Un statisticien informe l'industriel que la règle de décision (connue de tous les statisticiens) est de lancer le produit~A si $\Est{p^A}{y^A}>16.8573\%$. L'industriel pensant que son produit est tel que $p^A\geq 20\%$ se dit que cette règle de décision sera donc toujours acceptée. Qu'en pensez-vous~? 

\end{enumerate}
\end{exercice}


\begin{exercice}[Etudes expérimentales pour produits~A et B] $ $

L'expérimentateur désire mener une étude sur les outils d'aide à la décision pour les problématiques des produits~A et B. Il se propose alors de construire 6 urnes $U^A_p$ (avec $p=10\%$, $14\%$, $15\%$, $15.1\%$, $16\%$ et $20\%$) et 6 urnes $U^B_\mu$ (avec $\mu=0.1$, $0.14$, $0.15$, $0.151$, $0.16$ et $0.2$). Pour chacune des 12 urnes il construit $m=10000$ échantillons (notés $\Vect{y_{[k]}}$, $k=1,\cdots,m$) de taille $n=1000$. Voici les caractéristiques de ces urnes~:
une urne $U^\bullet_\mu$ ($\mu\geq0$ et $\bullet$=$A$ ou $B$) contient N=2000000 boules numérotées de 0 à 3. $N^\bullet_i$ désignant le nombre de boules dont le numéro est $i$ ($i=0,\cdots,3$), les répartitions de ces urnes sont fixées de la manière suivante~:
\begin{itemize}
\item $U^A_p$ ($\mu=p\in [0,1]$)~: $N^A_1=N\times p$ et $N^A_2=N^A_3=0$ de sorte qu'il y a une proportion $p$ de boules numérotées 1. La moyenne et la variance des numéros sont respectivement égaux à $\mu=p$ et $\sigma^2=p(1-p)$
\item $U^B_\mu$ ($\mu\geq0$)~: $N^B_1=N\times \mu -100000$, $N^B_2=N^B_3=20000$ de sorte que la moyenne des numéros est égale à $\mu$. La variance est égale à $\sigma^2=\mu(1-\mu)+\frac{2}{25}$.
\end{itemize}
Notons que $N^\bullet_0=N-(N^\bullet_1+N^\bullet_2+N^\bullet_3)$.

\noindent \textbf{Résultats expérimentaux pour le produit~A~:}
L'expérimentateur décide d'éprouver les Règles de Décision suivantes sur tous les $m=10000$ échantillons des 6 urnes~$U^A_p$~:
\begin{center}
Décider de lancer le produit~A si \fbox{$\Est{p^A}{y}>p^+_{lim}$}
\end{center} 
avec $p^+_{lim}$ pouvant prendre les 4 valeurs du tableau ci-dessous choisies sous le conseil du mathématicien. Voici les résultats fournis via la quantité $\gamma_m(p)=\overline{\Big(\Est{p^A}{y_{[.]}}>p^+_{lim}\Big)_m}$ correspondant à la proportion d'échantillons parmi les $m=10000$ conduisant au lancement du produit. Les valeurs entre parenthèses dans le tableau, fournies par le mathématicien, correspondent aux différentes valeurs de $\gamma_{+\infty}(p)$ (i.e. $m=+\infty$) qui, via la relation entre l'A.E.P. et l'A.M.P, est égal à $\gamma(p):=\PPP{\Est{p^A}{Y}>p^+_{lim}}$. 

\begin{center}
\begin{tabular}{|c|rl|rl|rl|rl|}\cline{1-9}
        \multirow{2}{*}{$p$}
         & 
    \multicolumn{8}{c|}{$p^+_{lim}$}
    
    
    
    
    
    
    
    \\ \cline{2-9}

    
        
         & 
    \multicolumn{2}{c|}{$15\%$}
     & 
    \multicolumn{2}{c|}{$16.4471\%$}
     & 
    \multicolumn{2}{c|}{$16.8573\%$}
     & 
    \multicolumn{2}{c|}{$17.6268\%$}
    
    \\ \cline{1-9}

    
        $10\%$
         & 
    
        $0\%$
         & 
    
        ($\simeq0\%$)
         & 
    
        $0\%$
         & 
    
        ($\simeq0\%$)
         & 
    
        $0\%$
         & 
    
        ($\simeq0\%$)
         & 
    
        $0\%$
         & 
    
        ($\simeq0\%$)
        
    \\ 

    
        $14\%$
         & 
    
        $16.57\%$
         & 
    
        ($\simeq18.11\%$)
         & 
    
        $1.45\%$
         & 
    
        ($\simeq1.29\%$)
         & 
    
        $0.52\%$
         & 
    
        ($\simeq0.46\%$)
         & 
    
        $0.05\%$
         & 
    
        ($\simeq0.05\%$)
        
    \\ 

    
        $15\%$
         & 
    
        $48.06\%$
         & 
    
        ($\simeq50\%$)
         & 
    
        $10.07\%$
         & 
    
        ($\simeq10\%$)
         & 
    
        $5.17\%$
         & 
    
        ($\simeq5\%$)
         & 
    
        $0.86\%$
         & 
    
        ($\simeq1\%$)
        
    \\ 

    
        $15.1\%$
         & 
    
        $51.52\%$
         & 
    
        ($\simeq53.52\%$)
         & 
    
        $11.93\%$
         & 
    
        ($\simeq11.71\%$)
         & 
    
        $6.18\%$
         & 
    
        ($\simeq6.03\%$)
         & 
    
        $1.37\%$
         & 
    
        ($\simeq1.28\%$)
        
    \\ 

    
        $16\%$
         & 
    
        $78.95\%$
         & 
    
        ($\simeq80.58\%$)
         & 
    
        $34.42\%$
         & 
    
        ($\simeq34.99\%$)
         & 
    
        $22.89\%$
         & 
    
        ($\simeq22.98\%$)
         & 
    
        $7.75\%$
         & 
    
        ($\simeq8.03\%$)
        
    \\ 

    
        $20\%$
         & 
    
        $99.99\%$
         & 
    
        ($\simeq100\%$)
         & 
    
        $99.84\%$
         & 
    
        ($\simeq99.75\%$)
         & 
    
        $99.35\%$
         & 
    
        ($\simeq99.35\%$)
         & 
    
        $96.67\%$
         & 
    
        ($\simeq96.97\%$)
        
    \\ \cline{1-9}

    \end{tabular}

\end{center}

\noindent Le tableau ci-dessous est l'équivalent du précédent pour les Règles de Décision de la forme~:
\begin{center}
Décider de lancer le produit~A si \fbox{$\Est{\delta_{p^A,15\%}}{y}>\delta^+_{lim}$}.
\end{center} 
Les valeurs (resp. entre parenthèses) du tableaux correspondent à $\gamma^\prime_m(p)=\overline{\Big(\Est{\delta_{p^A,15\%}}{y_{[.]}}>\delta^+_{lim}\Big)_m}$ (resp. à $\gamma^\prime_{+\infty}(p)=\gamma^\prime(p):=\PPP{\Est{\delta_{p^A,15\%}}{Y}>\delta^+_{lim}}$).
\begin{center}
\begin{tabular}{|c|rl|rl|rl|rl|}\cline{1-9}
        \multirow{2}{*}{$p$}
         & 
    \multicolumn{8}{c|}{$\delta^+_{lim}$}
    
    
    
    
    
    
    
    \\ \cline{2-9}

    
        
         & 
    \multicolumn{2}{c|}{$0$}
     & 
    \multicolumn{2}{c|}{$1.281552$}
     & 
    \multicolumn{2}{c|}{$1.644854$}
     & 
    \multicolumn{2}{c|}{$2.326348$}
    
    \\ \cline{1-9}

    
        $10\%$
         & 
    
        $0\%$
         & 
    
        ($\simeq0\%$)
         & 
    
        $0\%$
         & 
    
        ($\simeq0\%$)
         & 
    
        $0\%$
         & 
    
        ($\simeq0\%$)
         & 
    
        $0\%$
         & 
    
        ($\simeq0\%$)
        
    \\ 

    
        $14\%$
         & 
    
        $16.57\%$
         & 
    
        ($\simeq18.11\%$)
         & 
    
        $1.45\%$
         & 
    
        ($\simeq1.29\%$)
         & 
    
        $0.52\%$
         & 
    
        ($\simeq0.46\%$)
         & 
    
        $0.05\%$
         & 
    
        ($\simeq0.05\%$)
        
    \\ 

    
        $15\%$
         & 
    
        $48.06\%$
         & 
    
        ($\simeq50\%$)
         & 
    
        $10.07\%$
         & 
    
        ($\simeq10\%$)
         & 
    
        $5.17\%$
         & 
    
        ($\simeq5\%$)
         & 
    
        $0.86\%$
         & 
    
        ($\simeq1\%$)
        
    \\ 

    
        $15.1\%$
         & 
    
        $51.52\%$
         & 
    
        ($\simeq53.52\%$)
         & 
    
        $11.93\%$
         & 
    
        ($\simeq11.71\%$)
         & 
    
        $6.18\%$
         & 
    
        ($\simeq6.03\%$)
         & 
    
        $1.37\%$
         & 
    
        ($\simeq1.28\%$)
        
    \\ 

    
        $16\%$
         & 
    
        $78.95\%$
         & 
    
        ($\simeq80.58\%$)
         & 
    
        $34.42\%$
         & 
    
        ($\simeq34.99\%$)
         & 
    
        $22.89\%$
         & 
    
        ($\simeq22.98\%$)
         & 
    
        $7.75\%$
         & 
    
        ($\simeq8.03\%$)
        
    \\ 

    
        $20\%$
         & 
    
        $99.99\%$
         & 
    
        ($\simeq100\%$)
         & 
    
        $99.84\%$
         & 
    
        ($\simeq99.75\%$)
         & 
    
        $99.35\%$
         & 
    
        ($\simeq99.35\%$)
         & 
    
        $96.67\%$
         & 
    
        ($\simeq96.97\%$)
        
    \\ \cline{1-9}

    \end{tabular}

\end{center}

\noindent \textbf{Résultats expérimentaux pour le produit~B~:}
il expérimente les Règles de Décision suivantes sur tous les $m=10000$ échantillons des 6 urnes~$U^B_\mu$~:
\begin{center}
Décider de lancer le produit~B si \fbox{$\Est{\mu^B}{y}>\mu^+_{lim}$}
\end{center} 
avec $\mu^+_{lim}$ pouvant prendre les 4 valeurs du tableau ci-dessous fournissant les différentes quantités $\gamma_m(\mu)=\overline{\Big(\Est{\mu^B}{y_{[.]}}>\mu^+_{lim}\Big)_m}$ (et $\gamma_{+\infty}(\mu)=\gamma(\mu):=\overline{\Big(\Est{\mu^B}{Y}>\mu^+_{lim}\Big)_m}$).
\begin{center}
\begin{tabular}{|c|rl|rl|rl|rl|}\cline{1-9}
        \multirow{2}{*}{$\mu$}
         & 
    \multicolumn{8}{c|}{$\mu^+_{lim}$}
    
    
    
    
    
    
    
    \\ \cline{2-9}

    
        
         & 
    \multicolumn{2}{c|}{$0.15$}
     & 
    \multicolumn{2}{c|}{$0.168461$}
     & 
    \multicolumn{2}{c|}{$0.173694$}
     & 
    \multicolumn{2}{c|}{$0.183511$}
    
    \\ \cline{1-9}

    
        $0.1$
         & 
    
        $0.01\%$
         & 
    
        ($\simeq0.01\%$)
         & 
    
        $0\%$
         & 
    
        ($\simeq0\%$)
         & 
    
        $0\%$
         & 
    
        ($\simeq0\%$)
         & 
    
        $0\%$
         & 
    
        ($\simeq0\%$)
        
    \\ 

    
        $0.14$
         & 
    
        $23.39\%$
         & 
    
        ($\simeq24\%$)
         & 
    
        $2.55\%$
         & 
    
        ($\simeq2.22\%$)
         & 
    
        $1.13\%$
         & 
    
        ($\simeq0.87\%$)
         & 
    
        $0.14\%$
         & 
    
        ($\simeq0.11\%$)
        
    \\ 

    
        $0.15$
         & 
    
        $47.92\%$
         & 
    
        ($\simeq50\%$)
         & 
    
        $9.95\%$
         & 
    
        ($\simeq10\%$)
         & 
    
        $5.35\%$
         & 
    
        ($\simeq5\%$)
         & 
    
        $1.29\%$
         & 
    
        ($\simeq1\%$)
        
    \\ 

    
        $0.151$
         & 
    
        $50.72\%$
         & 
    
        ($\simeq52.77\%$)
         & 
    
        $10.95\%$
         & 
    
        ($\simeq11.27\%$)
         & 
    
        $5.61\%$
         & 
    
        ($\simeq5.76\%$)
         & 
    
        $1.25\%$
         & 
    
        ($\simeq1.2\%$)
        
    \\ 

    
        $0.16$
         & 
    
        $74.24\%$
         & 
    
        ($\simeq75.27\%$)
         & 
    
        $27.85\%$
         & 
    
        ($\simeq28.17\%$)
         & 
    
        $17.56\%$
         & 
    
        ($\simeq17.48\%$)
         & 
    
        $5.99\%$
         & 
    
        ($\simeq5.42\%$)
        
    \\ 

    
        $0.2$
         & 
    
        $99.96\%$
         & 
    
        ($\simeq99.94\%$)
         & 
    
        $98.43\%$
         & 
    
        ($\simeq97.91\%$)
         & 
    
        $96.45\%$
         & 
    
        ($\simeq95.53\%$)
         & 
    
        $86.35\%$
         & 
    
        ($\simeq85.64\%$)
        
    \\ \cline{1-9}

    \end{tabular}

\end{center}

\noindent Le tableau ci-dessous est l'équivalent du précédent pour les Règles de Décision de la forme~:
\begin{center}
Décider de lancer le produit~B si \fbox{$\Est{\delta_{\mu^B,0.15}}{y}>\delta^+_{lim}$}.
\end{center} 
Il fournit toutes les quantités $\gamma^\prime_m(\mu)=\overline{\Big(\Est{\delta_{\mu^A,0.15}}{y_{[.]}}>\delta^+_{lim}\Big)_m}$ (ainsi que $\gamma^\prime_{+\infty}(\mu)=\gamma^\prime(\mu):\overline{\Big(\Est{\delta_{\mu^A,0.15}}{Y}>\delta^+_{lim}\Big)_m}$).

\begin{center}
\begin{tabular}{|c|rl|rl|rl|rl|}\cline{1-9}
        \multirow{2}{*}{$\mu$}
         & 
    \multicolumn{8}{c|}{$\delta^+_{lim}$}
    
    
    
    
    
    
    
    \\ \cline{2-9}

    
        
         & 
    \multicolumn{2}{c|}{$0$}
     & 
    \multicolumn{2}{c|}{$1.281552$}
     & 
    \multicolumn{2}{c|}{$1.644854$}
     & 
    \multicolumn{2}{c|}{$2.326348$}
    
    \\ \cline{1-9}

    
        $0.1$
         & 
    
        $0.01\%$
         & 
    
        ($\simeq \mbox{ ????? }$)
         & 
    
        $0\%$
         & 
    
        ($\simeq \mbox{ ????? }$)
         & 
    
        $0\%$
         & 
    
        ($\simeq \mbox{ ????? }$)
         & 
    
        $0\%$
         & 
    
        ($\simeq \mbox{ ????? }$)
        
    \\ 

    
        $0.14$
         & 
    
        $23.39\%$
         & 
    
        ($\simeq \mbox{ ????? }$)
         & 
    
        $12.86\%$
         & 
    
        ($\simeq \mbox{ ????? }$)
         & 
    
        $10.15\%$
         & 
    
        ($\simeq \mbox{ ????? }$)
         & 
    
        $6.71\%$
         & 
    
        ($\simeq \mbox{ ????? }$)
        
    \\ 

    
        $0.15$
         & 
    
        $47.92\%$
         & 
    
        ($\simeq 50\%$)
         & 
    
        $8.31\%$
         & 
    
        ($\simeq 10\%$)
         & 
    
        $3.66\%$
         & 
    
        ($\simeq 5\%$)
         & 
    
        $0.51\%$
         & 
    
        ($\simeq 1\%$)
        
    \\ 

    
        $0.151$
         & 
    
        $50.72\%$
         & 
    
        ($\simeq \mbox{ ????? }$)
         & 
    
        $9.02\%$
         & 
    
        ($\simeq \mbox{ ????? }$)
         & 
    
        $3.81\%$
         & 
    
        ($\simeq \mbox{ ????? }$)
         & 
    
        $0.54\%$
         & 
    
        ($\simeq \mbox{ ????? }$)
        
    \\ 

    
        $0.16$
         & 
    
        $74.24\%$
         & 
    
        ($\simeq \mbox{ ????? }$)
         & 
    
        $59.05\%$
         & 
    
        ($\simeq \mbox{ ????? }$)
         & 
    
        $53.52\%$
         & 
    
        ($\simeq \mbox{ ????? }$)
         & 
    
        $45.56\%$
         & 
    
        ($\simeq \mbox{ ????? }$)
        
    \\ 

    
        $0.2$
         & 
    
        $99.96\%$
         & 
    
        ($\simeq \mbox{ ????? }$)
         & 
    
        $98.57\%$
         & 
    
        ($\simeq \mbox{ ????? }$)
         & 
    
        $96.46\%$
         & 
    
        ($\simeq \mbox{ ????? }$)
         & 
    
        $85.37\%$
         & 
    
        ($\simeq \mbox{ ????? }$)
        
    \\ \cline{1-9}

    \end{tabular}

\end{center}

\begin{enumerate}
\item En vous rappelant que pour un paramètre de moyenne $\mu$, la loi de probabilité de son estimateur $\Est{\mu}{Y}$ est $N(\mu,\frac{\sigma}{\sqrt{n}})$, donner les instructions~\texttt{R} qui ont permis au mathématicien de déterminer les valeurs de $\gamma(p)$ et $\gamma(\mu)$ dans les tableaux relatifs aux paramètres d'intérêt $p^A$ et $\mu^B$.
\item A quoi correspondent les instructions suivantes~:
\begin{Verbatim}[frame=leftline,fontfamily=tt,fontshape=n,numbers=left]
> p<-c(.1,.14,.15,.151,.16,.2)
> 100*pnorm(.169,p,sqrt(p*(1-p)/1000))
[1] 100.0000000  99.5890332  95.3780337  94.4054974  78.1221075   0.7127646
> mu<-p
> 100*pnorm(.169,mu,sqrt((mu*(1-mu)+2/25)/1000))
[1] 99.999994 97.974752 90.641532 89.388901 73.060810  2.269396
\end{Verbatim}

\item En vous appuyant sur les résultats du formulaire de cours, indiquer les instructions~\texttt{R} permettant de déterminer les valeurs de $\gamma(15\%)$ et $\gamma(0.15)$ pour les tableaux relatifs aux paramètres d'écart standardisé $\delta_{p^A,15\%}$ et $\delta_{\mu^B,0.15}$.
\item A quoi correspond l'instruction suivante~:
\begin{Verbatim}[frame=leftline,fontfamily=tt,fontshape=n,numbers=left]
> 100*pnorm(1.645)
[1] 95.00151
\end{Verbatim}

\end{enumerate}
\end{exercice}

\begin{exercice}[Finalisation des Règles de Décision] $ $

A partir des résultats expérimentaux de l'exercice précédent, on se propose de finaliser les Règles de Décision ($\mu$ pouvant être remplacé par $p$ pour le produit~A). Les quantités $p^+_{lim}$, $\mu^+_{lim}$ et $\delta^+_{lim}$ sont ici appelés \underline{seuils limites}.
\begin{enumerate}
\item Evaluer (approximativement en fonction de $\gamma_m(\mu)$) les risques d'erreurs de décision de type~I (notés $\alpha(\mu))$ et de type~II (notés $\beta(\mu)$). Exprimez-les en fonction de $\gamma(\mu)$ ou $\gamma^\prime(\mu)$.
\item Quelle est la plus grande valeur possible de la somme des deux risques de type~I~et~II~?
Peut-on proposer une Règle de Décision permettant de controler simultanément tous les risques d'erreurs~I~et~II~? Quel risque sera alors à controler~?
\item Quelles sont les situations (i.e. valeurs de $\mu$) à envisager pour générer une erreur de décision de type~I~? Elles seront appelées \textbf{Mauvaises situations} en opposition aux \textbf{Bonnes situations} qui correspondent aux valeurs de $\mu$ pour lesquelles l'assertion d'intérêt est vraie.
\item Pour quelle valeur de $\mu$ le risque de type~I est-il maximal~? 
Quelle est alors la \textbf{pire des (mauvaises) situations} qui permet de controler simultanément tous les risques de type~I~? Les urnes $U^\bullet_\mu$ correspondant à cette pire des situations sont-elles uniques~? 
\item Proposer des Règles de Décision associées à un risque maximal de type~I, noté $\alpha$, fixé à (ou n'excédant pas) $5\%$. Même question pour $\alpha=1\%$, $\alpha=10\%$.
\item Dans la pire des situations, combien en proportion d'estimations (du paramètre d'intérêt ou du paramètre d'écart standardisé selon le cas étudié) sont plus petit que les seuils limites.    
Comment peut-on alors définir directement ces seuils limites~? En déduire les instructions~\texttt{R} permettant de les obtenir.   
\item Dans le cas où la pire des situations correspond à plusieurs urnes, est-il possible de finaliser une unique Règle de Décision associée à $\alpha=5\%$ basée sur le paramètre d'intérêt~?Même question pour la Règle de Décision basée sur le paramètre d'écart standardisé. Quelles conclusions en tirez-vous sur les différentes Règles de Décision proposées précédemment~?    
\end{enumerate}

\end{exercice}


\begin{exercice}[Rédaction standard et abrégée]$ $\\
L'industriel est disposé à acheter deux échantillons $\Vect{y^A}$ et $\Vect{y^B}$ pour lesquels il obtient \texttt{mean(yA)=0.204}, \texttt{mean(yB)=0.172}, \texttt{sd(yB)=0.5610087}. Nous proposons les \textbf{rédactions standard} des corrections pour les questions~:
\begin{center}
Est-ce que le produit est rentable au risque maximal de type~I fixé à $\alpha=5\%$~?
\end{center}
En vous appuyant sur la construction des outils d'aide à la décision proposés dans les exercices précédents, expliquer les différents ingrédients de ces rédactions standard. Notamment, à quoi correspondent $\mathbf{H_1}$, non $\mathbf{H_1}$ et $\mathbf{H_0}$~?
Ces rédactions représentent-elles de bons résumés des principales informations relatives aux outils d'aide à la décision pour les produits~A et~B~?

\noindent \fbox{Rédaction Standard pour Produit~A}\hrulefill

\noindent \textbf{Hypothèses de test} : $\mathbf{H}_0:$ $p^{A}=15\%$ vs {\large $\mathbf{H}_1:$ $p^{A}>15\%$}\\
\textbf{Statistique de test sous $\mathbf{H}_0$} :
  $$
  \Est{\delta_{p^{A},15\%}}{Y^{A}}= {\displaystyle \frac{\Est{p^{A}}{Y^{A}}-15\%}{
\sqrt{\frac{15\%\times (1-15\%)} {1000}}
}} 
  \SuitApprox \mathcal{N}(0,1)
  $$
\textbf{Règle de décision} : Accepter $\mathbf{H}_1$ si 
  $\Est{\delta_{p^{A},15\%}}{y^{A}} > \delta^+_{lim,5\%}$\\
\noindent \textbf{Conclusion} :
puisqu'au vu des données, 
  \begin{eqnarray*}
\Est{\delta_{p^{A},15\%}}{y^{A}} &\NotR&\mathtt{(mean(yA)-0.15)/sqrt(0.15*(1-0.15)/length(yA))}\simeq 4.78232\\& >  & \delta^+_{lim,5\%} \NotR \mathtt{qnorm(1-.05)}\simeq1.644854
\end{eqnarray*}
  
on peut plutôt penser (avec un risque de 5\%) que le produit~A est rentable.

\noindent \fbox{Rédaction Standard pour Produit~B}\hrulefill

\noindent \textbf{Hypothèses de test} : $\mathbf{H}_0:$ $\mu^{B}=0.15$ vs {\large $\mathbf{H}_1:$ $\mu^{B}>0.15$}\\
\textbf{Statistique de test sous $\mathbf{H}_0$} :
  $$
  \Est{\delta_{\mu^{B},0.15}}{Y^{B}}= {\displaystyle \frac{\Est{\mu^{B}}{Y^{B}}-0.15}{
\Est{\sigma_{\cqlshat{\mu^{B}}}}{Y^{B}}
}} 
  \SuitApprox \mathcal{N}(0,1)
  $$
\textbf{Règle de décision} : Accepter $\mathbf{H}_1$ si 
  $\Est{\delta_{\mu^{B},0.15}}{y^{B}} > \delta^+_{lim,5\%}$\\
\noindent \textbf{Conclusion} :
puisqu'au vu des données, 
  \begin{eqnarray*}
\Est{\delta_{\mu^{B},0.15}}{y^{B}} &\NotR&\mathtt{(mean(yB)-0.15)/seMean(yB)}\simeq 1.24009\\&\ngtr & \delta^+_{lim,5\%} \NotR \mathtt{qnorm(1-.05)}\simeq1.644854
\end{eqnarray*}
  
on ne peut pas plutôt penser (avec un risque de 5\%) que le produit~B est rentable.

\noindent \textbf{Rédactions Abrégées~:} commenter les rédactions abrégées plus axées sur la pratique~:

\noindent \fbox{Rédaction Abrégée pour Produit~A}\hrulefill

\noindent \textbf{Assertion d'intérêt} :  $\mathbf{H}_1:$ $p^{A}>15\%$ \\
\textbf{Application numérique} :  puisqu'au vu des données, 
  \begin{eqnarray*}
\Est{\delta_{p^{A},15\%}}{y^{A}} &\NotR&\mathtt{(mean(yA)-0.15)/sqrt(0.15*(1-0.15)/length(yA))}\simeq 4.78232\\& >  & \delta^+_{lim,5\%} \NotR \mathtt{qnorm(1-.05)}\simeq1.644854
\end{eqnarray*}
  
on peut plutôt penser (avec un risque de 5\%) que le produit~A est rentable.

\noindent \fbox{Rédaction Abrégée pour Produit~B}\hrulefill

\noindent \textbf{Assertion d'intérêt} :  $\mathbf{H}_1:$ $\mu^{B}>0.15$ \\
\textbf{Application numérique} :  puisqu'au vu des données, 
  \begin{eqnarray*}
\Est{\delta_{\mu^{B},0.15}}{y^{B}} &\NotR&\mathtt{(mean(yB)-0.15)/seMean(yB)}\simeq 1.24009\\&\ngtr & \delta^+_{lim,5\%} \NotR \mathtt{qnorm(1-.05)}\simeq1.644854
\end{eqnarray*}
  
on ne peut pas plutôt penser (avec un risque de 5\%) que le produit~B est rentable.
\end{exercice}






\end{document}


